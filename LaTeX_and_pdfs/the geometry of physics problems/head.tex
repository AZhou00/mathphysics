%%% DOCUMENTCLASS %%%%%%%%%%%%%%%%%%%%%%%%%%%%%%%%%%%%%%%%%%%%%
%
\documentclass[10pt, a4paper]{scrartcl}
%
% Optionen f�r {}
%
% scrartcl - Artikel mit europ�ischer Layout-Anpassnug
% scrreprt - Artikel mit europ�ischer Layout-Anpassnug
% scrbook - Buch mit europ�ischer Layout-Anpassnug
% scrlettr - Briefe
%
% Kapitelstruktur: chapter (nur report/book) - section - subsection - subsubsection
%
%
% Optionen f�r []
% 10pt|12pt|14pt - Schriftgr��e
% a4paper - DIN A4
% fleqn - Linksb�ndige statt zentrierte Ausrichtung mathematischer Formeln
% leqno - Gleichungsnummerierungen links statt rechts
% titlepage|notitlepage - Eigene Seite f�r Titel/Zusammenfassung verwenden; Standardeinstellung f�r report/book: titlepage
% onecolumn|twocolumn - Ein- oder zweispaltiges Layout; nicht f�r letter/slides verf�gbar
% oneside|twoside - Ein- oder zweiseitiger Druck; Standard f�r alles au�er book: oneside
%
%
%
%%% DIVERSE PAKETE %%%%%%%%%%%%%%%%%%%%%%%%%%%%%%%%%%%%%%%%%%%%
%
\usepackage[a4paper]{geometry}		% Papierformat
\usepackage[OT2, T1]{fontenc}			% Vektorschriften werden hiermit seltsam rasterm��ig dargestellt ...
\usepackage{lmodern}				% ... jetzt nicht mehr
\usepackage[latin9]{inputenc}			% Codierung der Eingabe; hier: Latin9/ISO-8859-15
\usepackage[english]{babel}			% "ngerman, english": Silbentrennung englisch + neue deutsche Rechtschreibung
\usepackage{amsmath, amssymb, amstext}	% Verbesserte Formeldarstellung, erweiterte mathematischer Zeichenvorrat, \text{}
\usepackage{makeidx} \makeindex		% Index erstellen; Eintrag erstellen mit \index{Bezeichnung[!Unterbezeichnung[!Unterunterbezeichnung]]}; Index schreiben mit \printindex
%

\usepackage{tikz}
\usetikzlibrary{matrix,arrows}

\usepackage{hyperref}
\hypersetup{colorlinks=true, urlcolor=blue}

%
%
% 20110714 EY
\oddsidemargin-0.85cm
\evensidemargin-0.55cm
\topmargin-2.05cm     %I recommend adding these three lines to increase the 
\textwidth18.65cm   %amount of usable space on the page (and save trees)
\textheight25.45cm  
\parindent0.0em
%
%%% SILBENTRENNUNG %%%%%%%%%%%%%%%%%%%%%%%%%%%%%%%%%%%%%%%%%%%%
%
%\hyphenation{Bei-spiel} % Definiert eine selbstdefinierte kontextabh�ngige Silbentrennung im gesamten Dokument; F�r einzelne Flie�textw�rter: Bei\-spiel
%
%%% KOPF- UND FU�ZEILEN %%%%%%%%%%%%%%%%%%%%%%%%%%%%%%%%%%%%%%%
%
\usepackage{scrpage2}
\pagestyle{scrheadings}
\clearscrheadfoot
\usepackage{lastpage}

\automark[subsection]{section}
\ihead{\headmark}
\ohead{\pagemark}
\setheadsepline{1pt}

\pagestyle{plain}


%\theoremstyle{plain}
%\newtheorem{theorem}{Theorem}
%\newtheorem{axiom}{Axiom}
\newtheorem{lemma}{Lemma}
%\newtheorem{proposition}{Proposition}
%\newtheorem{corollary}{Corollary}

%\theoremstyle{definition}
%\newtheorem{definition}{Definition}



%\documentclass[twoside]{amsart}
%\usepackage{amssymb,latexsym}
%\usepackage{MnSymbol}
%\usepackage{times}
%\usepackage{graphics}
%\usepackage{tikz}
%\usepackage{hyperref}
%\hypersetup{colorlinks=true, urlcolor=blue}
%\usepackage{simpsons}
%\usepackage{epsdice}
%\usepackage{staves} 

%\usetikzlibrary{matrix,arrows}
%\usepackage{graphics}

%\oddsidemargin-0.85cm
%\evensidemargin-0.65cm
%\topmargin-2.05cm     %I recommend adding these three lines to increase the 
%\textwidth19.05cm   %amount of usable space on the page (and save trees)
%\textheight25.05cm  
%\parindent0.0em

%This next line (when uncommented) allow you to use encapsulated
%postscript files for figures in your document
%\usepackage{epsfig}

%plain makes sure that we have page numbers





%\chead{}
%\ihead{ssad}
%\cfoot{\pagemark/\pageref{LastPage}}
%\usepackage{fancyhdr}
%%\pagestyle{fancy}
%\fancyhead[L]{}
%\fancyhead[C]{--~\thepage~--}
%\fancyhead[R]{}
%\fancyfoot[L]{}
%\fancyfoot[C]{}
%\fancyfoot[R]{}
%
%%% SONSTIGES %%%%%%%%%%%%%%%%%%%%%%%%%%%%%%%%%%%%%%%%%%%%%%%%%
%

\setcounter{secnumdepth}{-1}
% -1 keine �berschrift wird nummeriert
% 0 Kapitel�berschriften (oder Abschnitts�berschriften im article.sty) werden nummeriert
% 1 die beiden h�chsten Ebenen werden nummeriert
% 5 alle �berschriften werden nummeriert. (Die Werte 2, 3, 4 ergeben die Zwischenwerte)
% \linespread{} % Zeilenabstand als Vielfaches von 1
