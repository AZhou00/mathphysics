%Nakahara_GTP-solutions.tex
%LaTeX
%\documentclass[twoside]{article}
\documentclass[twoside]{amsart}
%This makes the margins little smaller than the default
%\usepackage{fullpage}
%fullpage is not installed on andrew, so we'll just use these lines.
\oddsidemargin-0.80cm
\evensidemargin-0.10cm
\topmargin-1.6cm     %I recommend adding these three lines to increase the 
\textwidth17.55cm   %amount of usable space on the page (and save trees)
\textheight23.85cm  


%if you need more complicated math stuff, you should use the next line
\usepackage{amsmath}
%This next line defines a variety of special math symbols which you
%may need
\usepackage{amssymb}
\usepackage{graphics}
\usepackage{mathtools}
\usepackage{hyperref}

%This next line (when uncommented) allow you to use encapsulated
%postscript files for figures in your document
%\usepackage{epsfig}

\newtheorem{theorem}{Theorem}
\newtheorem{definition}{Definition}
\newtheorem{lemma}{Lemma}


%plain makes sure that we have page numbers
\pagestyle{plain}


\title{
Solutions to \emph{Geometry,Topology,Physics} by Mikio Nakahara, 2003.  
}
\author{
  Ernest Yeung
       }
\date{zima 2012}

%This defines a new command \questionhead which takes one argument and
%prints out Question #. with some space.
\newcommand{\questionhead}[1]
  {\bigskip\bigskip
   \noindent{\large\bf Question #1.}
   \bigskip}

\newcommand{\problemhead}[1]
  {\smallskip
   \noindent{\large\bf Problem #1.}
   }

\newcommand{\exercisehead}[1]
  {\smallskip
   \noindent{\large\bf Exercise #1.}
   }

\newcommand{\solutionhead}[1]
  {\smallskip
   \noindent{\large\bf Solution #1.}
   }


%-----------------------------------
\begin{document}
%-----------------------------------

\maketitle

\href{https://github.com/ernestyalumni/mathphysics/blob/master/LaTeX_and_pdfs/Nakahara_GTP-solutions.tex}{Permanent home} of this \verb|Nakahara GTP-solutions.tex| and \verb|.pdf| files on \verb|github|, \url{https://github.com/ernestyalumni/mathphysics/blob/master/LaTeX_and_pdfs/Nakahara_GTP-solutions.tex}.  \textbf{Go here} for the latest version; if you came here from anywhere else (such as Google Drive), then the version of solutions most likely will be outdated.  

From the beginning of 2016, I decided to cease all explicit crowdfunding for any of my materials on physics, math.  I failed to raise \emph{any} funds from previous crowdfunding efforts.  I decided that if I was going to live in \emph{abundance}, I must lose a scarcity attitude.  I am committed to keeping all of my material \textbf{open-sourced}.  I give all my stuff \emph{for free}.   

In the beginning of 2017, I received a very generous donation from a reader from Norway who found these notes useful, through \emph{PayPal}.  If you find these notes useful, feel free to donate directly and easily through \href{https://www.paypal.com/cgi-bin/webscr?cmd=_donations&business=ernestsaveschristmas%2bpaypal%40gmail%2ecom&lc=US&item_name=ernestyalumni&currency_code=USD&bn=PP%2dDonationsBF%3abtn_donateCC_LG%2egif%3aNonHosted}{PayPal}, which won't go through a 3rd. party such as indiegogo, kickstarter, patreon.  Otherwise, under the \emph{open-source MIT license}, feel free to copy, edit, paste, make your own versions, share, use as you wish.    

\noindent gmail        : ernestyalumni \\
linkedin     : ernestyalumni \\
tumblr       : ernestyalumni \\
twitter      : ernestyalumni \\
youtube      : ernestyalumni \\



%-----------------------------------%-----------------------------------%-----------------------------------%-----------------------------------%-----------------------------------
Solutions for \emph{Geometry, Topology, and Physics}.  Mikio Nakahara.  Institute of Physics Publishing.  2003.  ISBN 0 7503 0606 8
%-----------------------------------%-----------------------------------%-----------------------------------%-----------------------------------%-----------------------------------

\section{Quantum Physics}

\subsection{Analytical mechanics}

\subsubsection{Newtonian mechanics}

\subsubsection{Lagrangian formalism}

$\mathcal{L}$ independent of coordinate $q_k$; $q_k$ cyclic.  \\

$q_k(t) \to q_k(t) + \delta q_k(t)$
\begin{equation}
S[q(t), \dot{q}(t) ] = \int_{t_i}^{t_f} L(q,\dot{q}) dt \quad \quad \quad \, (1.3)
\end{equation}

\[
\delta S = \int_{t_i}^{t_f} \sum_k \delta q_k \left( \frac{ \partial L}{ \partial q_k} - \frac{d}{dt} \frac{ \partial L}{ \partial \dot{q}_k } \right) + \sum_k \left[ \delta_k \frac{ \partial L}{ \partial \dot{q}_k} \right]_{t_i}^{t_f} = 0 
\]
Note that 
\[
- \int \delta q_k \frac{d}{dt} \frac{ \partial L}{ \partial \dot{q}_k } + \delta q_k \left. \frac{ \partial L}{ \partial \dot{q}_k} \right|_{t_i}^{t_f} = \int \delta \dot{q}_k \frac{ \partial L}{ \partial \dot{q}_k }
\]
with $p_k = \frac{ \partial L}{ \partial \dot{q}_k}$.  
\begin{equation}
\Longrightarrow \delta q_k(t_i) p^k(t_i) = \delta q_k(t_f) p^k(t_f)
\end{equation}
since $t_i,t_f$ arbitrary, $\delta q_k(t) p^k(t)$ independent of $t$ and hence conserved.

\subsubsection{Hamiltonian formalism}

\exercisehead{1.1}  $A = A(q,p), B(q,p)$ defined on phase space of $H = H(q,p)$
\[
\begin{aligned}
  [A, c_1 B_1 + c_2 B_2 ] & = \partial_{q_k} A \partial_{p_k} (c_1 B_1 + c_2 B_2) - \partial_{p_k} A \partial_{q_k} ( c_1 B_1 + c_2 B_2) = \\
  & = c_1 \partial_{q_k} A \partial_{p_k} B_1 - c_1 \partial_{p_k} A \partial_{q_k} B_1 + c_2 \partial_{q_k} A \partial_{p_k} B_2 - c_2 \partial_{p_k} A \partial_{q_k} B_2 = \\
  & = c_1 [ A,B_1] + c_2 [A,B_2]
\end{aligned}
\]

\[
[A,B] = \partial_{q_k} A \partial_{p_k} B - \partial_{p_k} A \partial_{q_k} B = - (\partial_{q_k} B \partial_{p_k} A - \partial_{p_k} B \partial_{q_k} A ) = - [B,A]
\]

\[
\begin{aligned}
  [[A,B],C] & = \partial_{q_k} ( \partial_{q_l} A \partial_{p_l} B - \partial_{p_l} A \partial_{q_l} B ) \partial_{p_k} C - \partial_{p_k} ( \partial_{q_l} A \partial_{p_l} B - \partial_{p_l} A \partial_{q_l} B ) \partial_{q_k} C =  \\ 
  & = \partial^2_{q_k q_l} A \partial_{p_l} B \partial_{p_k} C - \partial^2_{q_k q_l} B \partial_{p_k} C \partial_{p_l} A + \\
  & \phantom{ = } + \partial^2_{q_k p_l} B \partial_{q_l} A \partial_{p_k} C - \partial^2_{q_k p_l} A \partial_{q_l} B \partial_{p_k} C + \\
  & \phantom{ = } + \partial^2_{p_k p_l} A \partial_{q_k} C \partial_{q_l} B - \partial^2_{p_k p_l } B \partial_{q_k} C \partial_{q_l} A + \\ 
  & \phantom{ = } + \partial^2_{q_l p_k} B \partial_{q_k} C \partial_{p_l}A - \partial^2_{q_l p_k} A \partial_{q_k} C \partial_{p_l} B
\end{aligned}
\]
\[
\Longrightarrow [[A,B], C ] + [[C,A], B] + [[B,C], A] = 0
\]

\begin{equation}
  \frac{dA}{dt} = \sum_k \left( \frac{ dA}{ dq_k } \frac{ dq}{ dt} + \frac{dA}{ dp_k } \frac{dp_k}{dt} \right) = \sum_k \left( \frac{dA}{ dq_k} \frac{ \partial H}{ \partial p_k} - \frac{ dA}{ d p_k} \frac{ \partial H}{ \partial q_k} \right) = [A,H] \quad \quad \quad \, (1.23)
\end{equation}


\subsection{ Canonical quantization }
\subsubsection{Hilbert space, bras, and kets}
\subsubsection{Axioms of canonical quantization}

A3. Poisson bracket in classical mechanics is replaced by commutator. 
\begin{equation}
  [ \widehat{A}, \widehat{B} ] \equiv \widehat{A} \widehat{B} - \widehat{B} \widehat{A} \quad \quad \quad \, (1.31)
\end{equation}



\subsection*{Problems}

\problemhead{1.1} $H = \int d^n x \left[ \frac{1}{2} \left( \frac{ \partial \phi }{ \partial t} \right)^2 + \frac{1}{2} ( \nabla \phi )^2 + V(\phi) \right]$  \\

If $\phi$ time independent, $H[\phi] = H_1[\phi] + H_2[\phi]$  
\[
\begin{aligned}
  & H_1[\phi] \equiv \frac{1}{2} \int d^n x ( \nabla x )^2 \\ 
  & H_2[\phi] \equiv \int d^n x V(\phi)
\end{aligned}
\]
\begin{enumerate}
\item[(1)] $\phi(x) \to \phi(\lambda x)$
\[
\begin{gathered}
  (\nabla \phi)^2 = \partial_j \phi \partial^j \phi = \lambda^{-2} \partial_i \phi \partial^i \phi = \lambda^{-2} (\nabla \phi)^2 \\ 
  \frac{ \partial }{ \partial x^i}  = \frac{ \partial y^i}{ \partial x^i } \frac{ \partial}{ \partial y^i } = \lambda \frac{ \partial }{ \partial y^i } \\ 
  d^n y = \lambda^n d^n x
\end{gathered}
\]
\item[(2)] $
  \begin{aligned}
    & H_1[\phi] \to \lambda^{n-2} H_1[\phi] \\
    & H_2[\phi] \to \lambda^n H_2[\phi] 
\end{aligned}$
\[
\Longrightarrow \partial_{\lambda} H = (n-2) H_1 + n H_2 = 0 \text{ or } (2-n) H_1 - nH_2 = 0 
\]
\item[(3)]
\end{enumerate}



\subsubsection{ Heisenberg equation, Heisenberg picture and Schr\"{o}dinger picture}

\subsubsection{Wavefunction}

\subsubsection{Harmonic oscillator}

\subsection{ Path integral quantization of a Bose particle}

\subsubsection{Path integral quantization}

\subsubsection{Imaginary time and partition function}

\subsubsection{Time-ordered product and generating functional }

\subsection{ Harmonic oscillator} 

\subsubsection{Transition amplitude}

\subsubsection{Partition function}

\subsection{ Path integral quantization of a Fermi particle }

\subsection{ Quantization of a scalar field}

\subsection{ Quantization of a Dirac field}

\subsection{ Gauge theories }

\subsubsection{ Abelian gauge theories}

\begin{align}
  & \nabla \cdot B = 0   & (1.241a) \\
  & \frac{ \partial B}{ \partial t } + \nabla \times E = 0  & (1.241b) \\ 
  & \nabla \cdot E = \rho & (1.241c) \\
  & \frac{ \partial E}{ \partial t}  - \nabla \times B = -j  & (1.241d) 
\end{align}

\[
A_{\mu} = ( - \phi, A )
\]

\[
\begin{aligned}
  & B = \nabla \times A \\ 
  & E = \frac{ \partial A}{ \partial t} - \nabla \phi \quad \quad \quad \, (1.242) 
\end{aligned}
\]

\exercisehead{1.11}

\[
\begin{gathered}
  \nabla \cdot B = 0 = \partial_i B_i = \partial_i (\nabla \times A)_i = \partial_i \epsilon_{ijk} \partial_j A_k = \epsilon_{ijk} \partial_i \partial_j A_k  = \\
  = \partial_1 \partial_2 A_3 - \partial_1 \partial_3 A_2 + \partial_2 \partial_3 A_1 - \partial_2 \partial_1 A_3 + \partial_3 \partial_1 A_2 - \partial_3 \partial_2 A_1 = \partial_1 F_{23} + \partial_2 F_{31} + \partial_3 F_{12} = 0 
\end{gathered}
\]

Watch out for convention.

\[
\partial_{\mu} = \left( \frac{ \partial }{ \partial t }, \nabla \right)
\]

\[
\begin{gathered}
  \frac{ \partial B}{ \partial t} + \nabla \times E = \partial_0 \epsilon_{ijk} \partial_j A_k + \epsilon_{ijk} \partial_j E_k = \epsilon_{ijk} \partial_0 \partial_j A_k + \epsilon_{ijk} \partial_j  \left(  \partial_0 A_k - \partial_k \phi  \right)  = \epsilon_{ikj} \partial_0 \partial_k A_j + \epsilon_{ijk} \partial_j \partial_0 A_k + \epsilon_{ijk} \partial_j \partial_k A_0 = \\
  = \epsilon_{ikj} \partial_0 \partial_k A_j + \epsilon_{ijk} \partial_j \partial_0 A_k + \epsilon_{ikj} \partial_j \partial_k A_0 = 
\end{gathered}
\]
Fix $i$.
\[
\begin{gathered}
  \Longrightarrow = \partial_0 \partial_k A_j - \partial_0 \partial_j A_k +  \partial_j \partial_0 A_k - \partial_k \partial_0 A_j + \partial_j \partial_k A_0   - \partial_k \partial_j A_0
\end{gathered}
\]


\exercisehead{1.12} 

\[
\begin{gathered}
  \mathcal{L} = (D^{\mu} \phi)^{\dag} ( D_{\mu} \phi ) + m^2 \phi^{\dag} \phi \\
  \begin{aligned}
    & \phi' = e^{- i e \alpha(x)} \phi \\ 
    & (\phi')^{\dag} = \phi^{\dag} e^{ i e \alpha(x) }
\end{aligned} \\ 
  A_{\mu}' = A_{\mu} -  \partial_{\mu} \alpha(x)  \quad \quad \, (1.257)
\end{gathered}
\]
Now easily
\[
\phi^{\dag} e^{i e \alpha(x) } e^{-i e \alpha(x) } \phi = \phi^{\dag} \phi 
\]
\[
\begin{gathered}
\begin{aligned}
  & D_{\mu} = \partial_{\mu} - ie A_{\mu} \\ 
  & D_{\mu} \phi = \partial_{\mu} \phi - ie A_{\mu} \phi
\end{aligned}  \\
\partial_{\mu} ( e^{-i e \alpha} \phi ) - i e( A_{\mu} - \partial_{\mu} \alpha ) e^{-i e \alpha } \phi = - e \partial_{\mu} \alpha \phi e^{-i e \alpha } + e^{-i e \alpha} \partial_{\mu} \phi - ie A_{\mu} e^{-i e \alpha} \phi + i e \partial_{\mu} \alpha e^{-ie \alpha } \phi = e^{-i e \alpha (x) } ( D_{\mu} \phi )
\end{gathered}
\]
\[
\begin{gathered}
  \begin{aligned}
    & D^{\mu} \phi = \partial^{\mu} \phi - i e A^{\mu} \phi \\ 
    & (D^{\mu} \phi)^{\dag} = \partial^{\mu} \phi^{\dag} + i e A^{\mu} \phi^{\dag}
\end{aligned} \\ 
(D^{\mu} \phi)^{\dag} \to \partial^{\mu} ( \phi^{\dag} e^{ie \alpha} ) + ie(A^{\mu} - \partial^{\mu} \alpha )\phi^{\dag} e^{ie \alpha} = \partial^{\mu} \phi^{\dag} e^{ie \alpha } + \phi^{\dag} ( ie \partial^{\mu} \alpha ) e^{ie \alpha} + ie A^{\mu} \phi^{\dag} e^{ie \alpha} - ie \partial^{\mu} \alpha \phi^{\dag} e^{ie \alpha} = (D^{\mu} \phi)^{\dag} e^{ie \alpha }
\end{gathered}
\]
Clearly $\mathcal{L}$ is invariant.  


\subsubsection{ Non-Abelian gauge theories }




\subsubsection{ Higgs fields }

\subsection{ Magnetic monopoles}

\subsection{ Instantons }

\subsubsection{ Introduction}

\subsubsection{ The (anti-)self-dual solution }

\section{Mathematical Preliminaries}

\subsection{Maps}

\subsubsection{Definitions}

\exercisehead{2.1}

$D = [-\pi/2, \pi/2]$, \, $R = [-1.1]$ \,  $f(x) = \sin{(x)}$ is bijective on $D$ to $R$

\exercisehead{2.2}

$f:x \to x^2$, $g: \to \exp{x}$

\[
f \circ g(x) = exp{(2x)}   \, \quad \, g \circ f(x) = \exp{x^2}
\]

\exercisehead{2.3} Consider $f(x) = f(y)$ 
\[
gf(x) = x = gf(y) = y \Longrightarrow x =y
\]
$f$ injective.
\[
\forall \, x \in X, \, x = g\circ f(x) = g(f(x)) = g(y) \text{ since } f:X\to Y. \text{ so } \, \exists \, y \in Y, \, \text{ s.t. } g(y) = x
\]
$g$ surjective.  

\exercisehead{2.4} 
\[
\begin{aligned}
  & a^{-1}: E^n \to E^n \\ 
  & a^{-1}(x) = x - a \\
  & a^{-1}a(x) = (x+a) - a = x \\
  & aa^{-1}(x) = (x-a) + a = x 
\end{aligned} \quad \quad \quad \, \begin{aligned}
  R^{-1}(x) = R^T x
  & R^{-1}R(x) = R^T R x = 1 x = x \\ 
  & RR^{-1}(x) = RR^Tx = 1x = x \\ 
  & (R^T R)_{ij} = R^T_{ik} R_{kj} = R_{ki} R_{kj} = \delta_{ij} \\ 
  & (RR^T)_{ij} = R_{ik} R^T_{kj} = R_{ik} R_{jk} = R^T_{ki} R^T_{kj} = \delta_{ij}
\end{aligned} 
\]
20120306 check above.

\[
\begin{aligned}
  & (R,a)(x) = Rx + a \\ 
  & (R,a)^{-1}(x) = R^T(x-a) \\ 
  & (Ra)^{-1} Ra(x) = R^T(Rx + a- a) = x \\ 
  & (Ra)(Ra)^{-1}(x) = R(R^T(x-a))  + a = x 
\end{aligned}
\]



\subsubsection{Equivalence relation and equivalence class}

\exercisehead{2.5} $m \sim m$ since $\frac{m}{2} = \frac{m}{2}$ \\
$m \sim n$. Then $n\sim m$ since remainder of $n$ by 2 is the same as the remainder of $m$ divided by 2. \\
$m \sim n$, $n\sim p$.  $m\sim p$ since remainder of $m$ by 2 is the same as the remainder of $m$ by 2 which is the same as the remainder of $p$ by 2.  

\exercisehead{2.6} $H = \lbrace \tau \in \mathcal{C} | Im{\tau} \geq 0 \rbrace$ \\
$SL(2,\mathbb{Z}) \equiv \lbrace \left( \begin{matrix} a & b \\ c & d \end{matrix} \right) | a,b, c, d \in \mathbb{Z}, \, ad - bc = 1 \rbrace$ \\
$\tau \sim \tau'$ if $\exists \, A = \left( \begin{matrix} a & b  \\ c & d \end{matrix} \right) \in SL(2,\mathbb{Z})$ s.t. $\tau' =  \frac{ a\tau + b}{ c\tau + d }$ \\

Note that 
\[
\left( \begin{matrix} a & b \\ c & d \end{matrix} \right) \left( \begin{matrix} \tau \\ 1 \end{matrix} \right) = \left( \begin{matrix} a\tau + b \\ c \tau + d \end{matrix} \right) 
\]
\[
\tau \sim \tau \text{ since } \left( \begin{matrix} 1 & \\ & 1 \end{matrix} \right) \left( \begin{matrix} \tau \\ 1 \end{matrix} \right) = \left( \begin{matrix} \tau \\ 1 \end{matrix} \right) 
\]

If $\tau \sim \tau'$ s.t. $\tau' = \frac{ a\tau + b}{ c\tau +d }$.  Consider $\left( \begin{matrix} d & - b \\ -c & a \end{matrix} \right)$   \\

$\tau' \sim \tau$ since 
\[
\left( \begin{matrix} d & -b \\ -c & a \end{matrix} \right) \left( \begin{matrix} \tau' \\ 1 \end{matrix} \right) = \left( \begin{matrix} d\tau' - b  \\ -c\tau' + a \end{matrix} \right) \quad \quad \quad \, \frac{ d \tau' - b }{ -c\tau' + a } = \frac{ d \left( \frac{ a \tau + b }{ c\tau + d } \right) - b }{ -c \left( \frac{ a\tau + b }{ c\tau + d } \right) + a } = \frac{ ad \tau - bc \tau }{ ad-bc } = \tau 
\]
Given $\tau \sim \tau'$, $\tau' \sim \tau''$
\[
\begin{aligned}
  & \left( \begin{matrix} a & b \\ c & d \end{matrix} \right) \left( \begin{matrix} \tau \\ 1 \end{matrix} \right) = \left( \begin{matrix} \tau' \\ 1 \end{matrix} \right) \\ 
  & \left( \begin{matrix} e & f \\ g & h \end{matrix} \right) \left( \begin{matrix} \tau' \\ 1 \end{matrix} \right) = \left( \begin{matrix} \tau'' \\ 1 \end{matrix} \right) 
\end{aligned} \quad \quad \Longrightarrow \left( \begin{matrix} e & f \\ g & h \end{matrix} \right) \left( \begin{matrix} a & b \\ c & d \end{matrix} \right) \left( \begin{matrix} \tau \\ 1 \end{matrix} \right) = \left( \begin{matrix} \tau'' \\ 1 \end{matrix} \right) = \left( \begin{matrix} a e + cf & be + df \\ ag + ch & bg + d h \end{matrix} \right) \left( \begin{matrix} \tau \\ 1 \end{matrix} \right) 
\]
\[
(ae + cf) (bg - dh) - (ag+ch) (be +df) = abeg + adeh + bcfg + cdfh - abeg - adfg - bceh - cdfh = 1 
\]

Example 2.6. \\
$g \sim g'$ if $\exists \, h \in H$ s.t. $g' = gh$ \\
$[g] = \lbrace gh | h \in H \rbrace \equiv gH$ (left) coset. \\
quotient space $\equiv G/H$ \\
if $H$ normal subgroup of $G$, $ghg^{-1} \in H$, \, $\forall \, g \in G$, $h \in H$, then $G/H$ quotient group.  \\
Since
\[
(g')^{-1} h g' = h'' \in H \quad \, \Longrightarrow hg' = g'h''
\]
\[
\Longrightarrow ghg'h' = gg' h'' h' = gg'h'''
\]
$[g][g'] = [gg']$ well-defined.  \\

Note $[e]$, $[g]^{-1} = [g^{-1}]$

\exercisehead{2.7} $a,b \in G$ conjugate to each other, $a\simeq b$, if $\exists \, g \in G$ s.t. $b = gag^{-1}$
$a\simeq a$ since $a = eae^{-1} = a$ \\
If $a\simeq b$, $g^{-1}bg = a$ and $g^{-1} \in G$, so $b\simeq a$. \\
If $a \simeq b$ and $b\simeq c$, then $c = hbh^{-1}$, so $c = hg a g^{-1} h^{-1}$ and $(hg)^{-1} = g^{-1}h^{-1}$.  


\subsection{Vector spaces}

\subsubsection{Vectors and vector spaces}

\subsubsection{Linear maps, images and kernels}

\begin{theorem}[2.1] If linear $f: V \to W$, 
\[
\text{dim}{V} = \text{dim}{ (\text{ker}{f}) } + \text{dim}{ (\text{im}{f} ) }
\]
\end{theorem}

\begin{proof}
  $\text{ker}{f}$, $\text{im}{f}$ vector spaces.  \\

Let basis of $\text{ker}{f}$ be $\lbrace g_1 \dots g_r \rbrace$ \\
\phantom{Let } basis of $\text{im}{f}$ be $\lbrace h_1' \dots h_s' \rbrace$ \\
$\forall \, i \, (1 \leq i \leq s)$, $h_i \in V$ s.t. $f(h_i) = h_i'$ \\

Consider $\lbrace g_1 \dots g_r, h_1 \dots h_s \rbrace$ \\

Let arbitrary $v\in V$ \\
$f(v) \in \text{im}{f}$, so $f(v) = c^i h_i' = c^i f(h_i)$ \\
by linearity $f(v- c^i h_i) = 0$, \\
so $v - c^i h_i \in \text{ker}{f}$ \\
so $\forall \, $ arbitrary $v\in V$, $v$ linear combination of $\lbrace g_1 \dots g_r, h_1 \dots h_s \rbrace$
\end{proof}

\exercisehead{2.8}

\begin{enumerate}
 \item[(1)] If $x_1, x_2 \in ker{f}$, then for $x_1 + x_2$
\[
\text{ by linearity of $f$, } f(x_1 + x_2) = f(x_1) + f(x_2) = 0 + 0 = 0 \quad \, \Longrightarrow x_1 + x_2 \in ker{f}
\]
\[
f(cx_1) = cf(x_1) = 0 \quad \, \Longrightarrow cx_1 \in ker{f}
\]
Under closure of addition and multiplication, $ker{f}$ is a vector space.   \\

If $w_1, \, w_2 \in im{f}$, 
\[
\begin{aligned} w_1 & = f(x_1) \\ w_2 & = f(x_2) \end{aligned} \text{ for some } x_1, x_2 \in V
\]
\[
w_1 + w_2 = f(x_1) + f(x_2) = f(x_1 + x_2) \in im{f} \text{ since $f$ linear and since } x_1 + x_2 \in V
\]

\[
cw_1 = cf(x_1) = f(cx_1) \in im{f} \text{ since } cx_1 \in V
\]
\item[(2)] If $f:V \to V$ isomorphism, $f$ bijective. $f(0)=0$ always for a linear map. Consider $x \in ker{f}$. So $f(x) =0$.  $f(x) = f(0)$, so $x=0$.  \\
Since $f$ linear, if $f(x) = f(y)$, $f(x) - f(y) = f(x-y) =0$.  Since $ker{f} =0$, $x-y =0$ so $x=y$.  $f$ bijective so $f$ an isomorphism.  
\end{enumerate}

\subsubsection{Dual vector space}

\[
f(\mathbf{v}) = f_i \alpha^i(v^j \mathbf{e}_j ) = f_i v^j \alpha^i(\mathbf{e}_j) = f_i v^i \quad \quad \, (2.12)
\]

Use notation $\langle, \rangle : V^* \times V \to K$ \\

$f: V \to W$ \\
$g: W \to K$ \\
$g\in W^*$ \\
$g\circ f : V \to K$ (Note $g\circ f$ on $V$, key observation) \\
$g\circ f \equiv h \in V^*$ 
\[
h(\mathbf{v}) \equiv g(f(\mathbf{v})) \quad \, \mathbf{v} \in V \quad \, (2.13)
\]

Given $g \in W^*$, $f:V \to W$ induces map $h\in V^*$ \\
$f^*: W^* \to V^*$ \\
$f^*:g \mapsto h = f^*(g) = g\circ f$ \quad \, $h$ is the pullback of $g$ by $f^*$


\exercisehead{2.9} 

Given $f_i = A_i^{\, k} e_k$, 
\[
\alpha^j f_i = A_i^{\, k} \alpha^j e_k = A_i^{\, j}
\]
\[
\beta^j A_j^{ \, i} = \beta^j \alpha^i f_j = \alpha^i \Longrightarrow \alpha^i = \beta^j A_j^{\, i }
\]

\subsubsection{Inner product and adjoint}

isomorphism $g:V \to V^*$, $g\in GL(m,K)$ 
\[
g:v^j \to g_{ij} v^j \quad \quad \, (2.14)
\]
$g(v_1, v_2) \equiv \langle gv_1, v_2 \rangle$ 
\[
g(v_1, v_2) = v_1^{\, i} g_{ji} v_2^{\, j} \quad \quad \, (2.16)
\]
$W = W(n,\mathbb{R})$, $\lbrace f_{\alpha} \rbrace$ basis $G:W \to W^*$ \\

Given $f:V \to W$ \\
adjoint of $f$, $\widetilde{f}$ 
\[
G(w,fv) = g(v,\widetilde{f}w) \quad \quad \, (2.17)
\]
where $v\in V$, $w\in W$ 
\[
w^{\alpha} G_{\alpha \beta} f^{\beta}_{\, i} v^i = v^i g_{ij} \widetilde{f}^j_{\, \alpha} w^{\alpha} \quad \quad \, (2.18)
\]
$G_{\alpha \beta} f^{\beta}_{\, i} = g_{ij} \widetilde{f}^j_{\, \alpha} $
\[
\widetilde{f} = g^{-1} f^t G^t \quad \quad \, (2.19)
\]

\exercisehead{2.10} (cf. wikipedia, ``Rank'') Consider a $m\times n$ matrix $A$ with column rank $A$ (maximum number of linearly independent column vectors of $A$).   \\

$dim$. of column space of $A = r$.  Then let $c_1 \dots c_r$ basis.  \\
Place $c_i$'s as column vectors to form $m \times r$ matrix $C = [ c_1 \dots c_r ]$ \\
$\exists \, r \times m$ matrix $R$ s.t. $A = CR$.  \quad \, $r_{ij}$ \, $\begin{aligned} & \quad  \\ i & = 1 \dots r \\ j & = 1 \dots m \end{aligned}$ \\

$A=CR$, so $\forall \, $ row vector of $A$ is a linear combination of row vectors of $R$, so row space of $A$ contained in row space of $R$.  \\

row rank $A \leq $ row rank $R$.  \\

$R$ has $r$ rows, $a_{ij} = c_{ik} r_{kj}$ \, $j= 1 \dots n$.  row rank $R \leq r = $ column rank $A$ \\
row rank $A \leq $ column rank $A$ \\
row rank $A^T \leq $ column rank $A^T$
$\Longrightarrow $ row rank $A = $ column rank $A$ or $\operatorname{rank}{(A)} = \operatorname{rank}{(A^T)}$ \\

Likewise for $NfM$ by following the same arguments.  

\exercisehead{2.11} \begin{enumerate}
\item[(a)] $g(v_1,v_2) = \overline{v}_1^{ \, i } g_{ij} v_2^{\, j }$ \\
\[
\overline{g(v_2,v_1)} = \overline{ \overline{v}_2^{ \, i} g_{ij} v_1^{ \, j} } = v_2^{ \, i } \overline{g}_{ij} \overline{v}_1^{\, j} = \overline{v}_1^{\, j} g_{ji} v_2^{\, i} = g(v_1,v_2)
\]
\[
g(v,\widetilde{f}w) = \overline{v}^i g_{ij} \widetilde{f}_{jk} w^k = \overline{G(w,fv)} = \overline{ \overline{w}^{\alpha} G_{\alpha \beta} f_{\beta \gamma} v^{\gamma} } = w^{\alpha} \overline{G}_{\alpha \beta} \overline{f}_{\beta \gamma} \overline{v}^{\gamma} = \overline{G}_{k\beta} \overline{f}_{\beta i} \overline{v}^i w^k
\]
\[
\Longrightarrow g_{ij} \widetilde{f}_{jk} = \overline{G}_{k \beta} \overline{f}_{\beta i }  = f^{\dag}_{i \beta} G_{\beta k}^{\dag} 
\]
\[
\Longrightarrow \widetilde{f} = g^{-1} f^{\dag} G^{\dag}
\]
\item[(b)]
\end{enumerate}

\subsubsection{Tensors} 

tensor $T$ of type $(p,q)$ maps $p$ dual vectors and $q$ vectors to $\mathbb{R}$

\begin{equation}
  T: \overset{p}{\bigotimes} V^* \overset{q}{\bigotimes} V \to \mathbb{R}
\end{equation}


\exercisehead{2.12} $\begin{aligned} & \quad \\ & f:V \to W \\ & f(v) = w \end{aligned}$, so $f(v) \in W$.  \\
Then tensor identified with dual vector of $W^*$.  Since $f:V$, Then $(1,1)$.  

\subsection{Topological spaces}

\subsubsection{Definitions}

\exercisehead{2.13}

$\tau_{\mathbb{R}} = \lbrace (a,b) | a,b \in \lbrace \mathbb{R}, \pm \infty \rbrace \rbrace$ \\

$\bigcap_{n=1}^{\infty} (a,b + \frac{1}{n} ) = (a,b]$.  Then $\lbrace b \rbrace \in \tau_{\mathbb{R}}$, $\forall \, b \in \mathbb{R}$.  \\

So then $\forall \, $ subset $Y \subset X$ is open.  Discrete topology.  


\subsubsection{Continuous maps}

\exercisehead{2.14} If $\begin{aligned} & \quad \\ 
  & f: \mathbb{R} \to \mathbb{R} \\ 
  & f(x) = x^2 \end{aligned}$ then $(-\epsilon, + \epsilon) \mapsto [0, \epsilon^2]$ 

\subsubsection{Neighborhoods and Hausdorff spaces}

\exercisehead{2.15} Let $X = \lbrace \text{John}, \text{Paul}, \text{Ringo}, \text{George}\rbrace$.   \\
\[
\begin{aligned} & \quad \\ 
& U_0 = \emptyset \\ 
& U_1  = \lbrace \text{John} \rbrace \\
& U_2 = \lbrace \text{John}, \text{Paul} \rbrace \\
& U_3 = X \end{aligned} \quad \quad \quad \, \begin{aligned} & U_1 \bigcup U_2 = U_2 \\ & U_1 U_2 = U_1 \end{aligned}
\]

Consider John and Ringo. Only neighborhood with open set is $X$ for Ringo.  Then $XU_{\text{John}} \neq \emptyset$


\exercisehead{2.16} $\forall \, a, b$, consider $\left( \frac{3a - b}{2}, \frac{a+b}{2} \right)$, $\left( \frac{b+a}{2}, \frac{3b - a}{2} \right)$



\subsubsection{Closed set}

\subsubsection{Compactness}

\subsubsection{Connectedness}

\subsection{Homeomorphisms and topological invariants}

\subsubsection{Homeomorphisms}

\subsubsection{Topological invariants}

\exercisehead{2.18} $f:S^1 \to E$ \\
$\begin{aligned} & \quad \\ 
  & f(x,y) = (ax,by) \\ 
  & f^{-1}(x,y) = \left( \frac{x}{a}, \frac{y}{b} \right) \end{aligned}$ \quad \, $ff^{-1} = f^{-1}f = 1$ bijective and cont. 

\subsubsection{Homotopy type}

\subsubsection{Euler characteristic: an example}


\section{Homology Groups} 

\subsection{Abelian groups}

\subsubsection{Elementary group theory}

e.g. $f: \mathbb{Z} \to \mathbb{Z}_2 = \lbrace 0 , 1 \rbrace$ \\
$\begin{aligned}
  & f(2n) = 0 \\ 
  & f(2n+1) = 1 \end{aligned}$  \\
homomorphism $f(2m+1 +2n) = f(2(m+n) + 1) = 1 = 1 + 0 = f(2m+1) + f(2n)$ \\
$k\mathbb{Z} \equiv \lbrace kn | n \in \mathbb{Z} \rbrace$, $k \in \mathbb{N}$ subgroup of $\mathbb{Z}$, $\mathbb{Z}_2 = \lbrace 0 ,1 \rbrace$ not a subgroup.  \\

Let $H$ subgroup of $G$, $\forall \, x, y \in G$, $x\sim y$ if $x- y \in H$ \\
group operation in $G/H$ naturally induced: $[x] + [y] = [x+y]$ \\
$G/H$ group since $H$ always a normal subgroup of $G$.  \\
\quad $aH = Ha \, \forall \, a \in G$ \\
\quad \, if $aH = Ha$ \, $\forall \, a \in G$, normal indeed.  \\
\quad $aH = a + x - y = x - y + a = Ha$ (since $G$ abelian) \\

If $H = G$, $0-x \in G$, \, $\forall \, x \in G$, $G/G = \lbrace  0 \rbrace$ \\
If $H = \lbrace 0 \rbrace$, $G/H = G$ since $x-y = 0$ iff $x=y$  \\

Ex. 3.1. Let us work out the quotient group $\mathbb{Z}/ k\mathbb{Z}$.  \\
$km - kn = k (m-n) \in k \mathbb{Z}$  \, $[km ] = [kn]$ \\

$\forall \, j$, \, $1 \leq j \leq k-1$, $(km + j) - (kn +j) = k(m-n) \in k \mathbb{Z}$.  $[km + j ] = [kn +j ]$ \\
$\forall \, j, l$, $0\leq j , l \leq k -1$, $j\neq l$, $(km+j) - (kn +l) = k(m-n) + (j-l ) \notin k\mathbb{Z}$.  Never belong to the same equivalence class.  \\

$\Longrightarrow \mathbb{Z}. k \mathbb{Z} = \lbrace [0], \dots, [k-1] \rbrace$.  \\
Define isomorphism $\begin{aligned} & \quad \\ 
  & \varphi : \mathbb{Z}/ k\mathbb{Z} \to \mathbb{Z}_k \\
  & \varphi([j]) = j \end{aligned}$, then $\mathbb{Z}/ k\mathbb{Z} \simeq \mathbb{Z}_k$  

\begin{lemma}[3.1] Let $f:G_1 \to G_2$ homomorphism.  Then
\begin{enumerate}
\item[(a)] $\ker{f} = \lbrace x | x \in G_1, f(x) = \rbrace$ subgroup of $G$.  Note: $\ker{f}$ \textbf{normal subgroup} of $G_1$, \\
$\begin{gathered}
  f(gxg^{-1}) = f(g)f(x) f(g^{-1}) = 1 \, \forall \, g \in G, \, x\in \ker{f}  \, \Longrightarrow \ker{f} \text{ normal subgroup }
\end{gathered}$
\item[(b)] $\text{im}{f} = \lbrace x | x \in f(G_1) \subset G_2 \rbrace$ subgroup of $G_2$
\end{enumerate}
\end{lemma}

\begin{proof}
\begin{enumerate}
\item[(a)] Let $x,y \in \ker{f}$ \\
$xy \in \ker{f}$ since $f(xy) = f(x) f(y) = 1\cdot 1 = 1$ \\
Note $1 \in \ker{f}$ since $f(1) = f(1\cdot 1 ) = f(1) f(1) \Longrightarrow f(1) = 1$ \\
$x^{-1} \in \ker{f}$ since $f(x^{-1}\cdot x) = f(x^{-1})f(x) = f(x^{-1}) 1 = f(1) = 1$ \, $f(x^{-1})= 1$
\item[(b)] Let $\begin{aligned} & \quad \\ 
  & y_1 = f(x_1) \\ 
  & y_2  = f(x_2) \end{aligned}$ \quad \, $\begin{aligned} & \quad \\ 
  & y_1,y_2 \in \text{im}(f) \\
  & x_1, x_2 \in G_1 \end{aligned}$ \quad \, \\
$\begin{gathered}
  y_1 y_2 = f(x_1) f(x_2) = f(x_1x_2) \quad x_1 x_2 \in G_1 \quad y_1 y_2 \in \text{im}{f} \\
  1 \in \text{im}{f} \text{ since } f(1) = 1 \, (\text{or } f(x_1) = f(x_1 \cdot 1) = f(x_1) f(1); \, \text{ so } f(1) = 1 \text{ and } 1 \in \text{im}{f}) \\ 
  1 = f(xx^{-1}) = f(x) f(x^{-1}) = yf(x^{-1}) \Longrightarrow f(x^{-1}) = y^{-1} \text{ and } y^{-1} \in \text{im}{f} \text{ since } x^{-1} \in G 
\end{gathered}$
\end{enumerate}
\end{proof}

\begin{theorem}[3.1] (Fundamental Thm. of homomorphism) 
\[
G_1/ \ker{f} \simeq \text{im}{f} 
\]
\end{theorem}

\begin{proof} By Lemma 3.1, both sides are groups. \end{proof}

Define.  $\begin{aligned} & \quad \\ 
  & \varphi: G_1/ \ker{f} \to \text{im}{f} \\
  & \varphi([x]) = f(x) \end{aligned}$. \quad \, $\varphi$ well-defined since $\forall \, x' \in [x]$, $\exists \, h \in \ker{f}$ s.t. $ \begin{gathered} x' = x + h \\ 
  f(x') = f(x) f(h) = f(x) \end{gathered}$ \\

\[
\varphi([x]+ [u]) = \varphi([x+y]) =  f(x+y) = f(x) + f(y) = \varphi([x]) + \varphi([y]) 
\]
$\varphi$ 1-to-1: if $\varphi([x]) = \varphi([y])$, then $f(x) = f(y)$ or $f(x) - f(y) = f(x-y) = 0$.  $x-y \in \ker{f}$ so $[x] = [y]$ \\
$\varphi$ onto: if $y \in \text{im}{f}$, $\exists \, x \in G_1$ s.t. $f(x) = y = \varphi([x])$ 

\subsubsection{Finitely generated Abelian groups and free Abelian groups}

\begin{lemma}[3.2] Let $G$ be free Abelian group of rank $r$, 
\[
G = \lbrace n_1 x_1 + \dots + n_r x_r | n_i \in \mathbb{Z}, \, 1 \leq i \leq r, \, n_1 x_1 + \dots + n_r x_r = 0 \text{ only if } n_1 = \dots = n_r = 0 \rbrace \equiv \text{ free Abelian group of rank $r$ } 
\]
Let subgroup $H$.  Choose $p$ generators $x_1 \dots x_p$ of $r$ generators of $G$ so generate $H$.  
\[
H \simeq k_1 \mathbb{Z} \oplus \dots \oplus k_p \mathbb{Z} \text{ and } H \text{ rank } p
\]
\end{lemma} 

\begin{proof}
\[
\begin{gathered}
  f: \underbrace{\mathbb{Z} \oplus \dots \oplus \mathbb{Z} }_{m} \to G \\ 
  f(n_1 \dots n_m) = n_1 x_1 + \dots + n_m x_m \quad \, (\text{surjective homomorphism})
\end{gathered}
\]
From Thm. 3.1. $ \underbrace{\mathbb{Z} \oplus \dots \oplus \mathbb{Z}}_{ m } / \ker{f} \simeq G$ \\

$\ker{f}$ subgroup of $\underbrace{ \mathbb{Z} \oplus \dots \oplus \mathbb{Z} }_{m}$, so Lemma 3.2, $\ker{f} \simeq k_1 \mathbb{Z} \mathbb{Z} \oplus \dots \oplus k_p \mathbb{Z}$  

\[
G \simeq \underbrace{ \mathbb{Z} \oplus \dots \oplus \mathbb{Z} }_{m} / \ker{f} \simeq \underbrace{ \mathbb{Z} \oplus \dots \oplus \mathbb{Z} }_{m} /(k_1 \mathbb{Z} \oplus \dots \oplus k_p \mathbb{Z} ) \simeq \underbrace{ \mathbb{Z} \oplus \dots \oplus \mathbb{Z} }_{m-p} \oplus Z_{k_1} \oplus \dots \oplus \mathbb{Z}_{k_p}
\]



\end{proof}



\subsection{Simplexes and simplicial complexes}

\subsubsection{Simplexes}

number of $q$-faces in $r$-simplex is $\binom{r+1}{q+1}$ \\

$r+1$ \, $p_0 \dots p_r$ pts., choose $p_{i_0} \dots p_{i_q}$ pts.  

\subsubsection{Simplicial complexes and polyhedra}

Example 3.5. Fig. 3.5(b) not a triangulation of a cylinder.  \\
$\begin{aligned} & \quad \\ 
  & \sigma_2 = \langle p_0 p_1 p_2 \rangle \\ 
  & \sigma_2' = \langle p_2 p_3 p_0 \rangle \end{aligned}$ \\
$\sigma_2 \sigma_2' \neq \emptyset$\\
$\sigma_2 \sigma_2' = \langle p_0 \rangle \bigcup \langle p_2 \rangle$ \quad \, $\sigma_2 \sigma_2'$ is not an actual face.  



\subsection{Homology groups of simplicial complexes}

\subsubsection{Oriented simplexes}

\subsubsection{Chain group, Cycle group and boundary group}

\begin{definition}[3.2] $r$-chain group $C_r(K)$ of simplicial complex $K$ is free Abelian group generated by oriented $r$-simplexes of $K$ element of $C_r(K)$ is $r$-chain.\end{definition}

Let $I_r$ $r$- simplexes in $K$, $\sigma_{r,i}$ ($1 \leq i \leq I_r$) 
\begin{equation}
c = \sum_{i=1}^{I_r} c_i \sigma_{r,i} \quad \, c_i \in \mathbb{Z}, \text{ coefficients of $c$ } \quad \quad \quad \, (3.15)
\end{equation}

addition of 2 $r$-chains $\begin{aligned} & \quad \\ 
  & c = \sum_i c_i \sigma_{r,i} \\ 
  & c' = \sum_i c_i' \sigma_{r,i} \end{aligned}$ \quad \quad \, $c + c' = \sum_i (c_i + c'_i) \sigma_{r,i} \quad \quad \, (3.16)$ \\

inverse $-c  =\sum_i ( - c_i) \sigma_{r,i}$ 

\begin{equation}
C_r(K) \simeq \underbrace{\mathbb{Z} \oplus \dots \oplus \mathbb{Z}}_{ I_r} \quad \quad \, (3.17) \text{ free Abelian group of rank $I_r$}
\end{equation}


$0$-simplex has no boundary: $\partial_0 p_0 = 0$ \quad \, (3.18) \\

Fig. 3.7(a) oriented 1 simplex.  
\[
\partial_1 (p_0 p_1) + \partial_1 (p_1 p_2) = p_1 - p_0 + p_2 - p_1 = p_2 - p_0 = \partial_1 (p_2 p_0)
\]

Fig. 3.7(b) triangle. 
\[
\partial_1 (p_0 p_1) + \partial_1 (p_1 p_2) + \partial_1 (p_2 p_0) = p_1 - p_0 + p_2 - p_1 + p_0 - p_2 = 0 
\]

Let $\sigma_r(p_0 \dots p_r)$ oriented $r$-simplex.
\begin{equation}
\partial_r \sigma_r \equiv \sum_{i=0}^r (-1)^i (p_0 p_1 \dots \widehat{p}_i \dots p_r ) \quad \quad \quad (3.20)
\end{equation}


$K \equiv n$-dim. simplicial complex. \\

chain complex $C(K)$.  \\
$i$ inclusion map $i: 0 \hookrightarrow C_n(K)$
\begin{equation}
0 \xrightarrow{i} C_n(K) \xrightarrow{\partial_n} C_{n-1}(K) \xrightarrow{\partial_{n-1}} \dots \xrightarrow{\partial_2} C_1(K) \xrightarrow{\partial_1} C_0(K) \xrightarrow{\partial_0} 0 \quad \quad \, (3.23)
\end{equation}

\begin{definition}[3.3] If $c \in C_r(K)$, $\partial_r c = 0$, $c$ \, $r$-cycle. 
\[
\boxed{ Z_r(K) = \lbrace c | \partial_r c =0 \rbrace  = \ker{\partial_r} \equiv r-\text{cycle group} }
\]
if $r =0$, $\partial_0 c =0$, $Z_0(K) = C_0(K)$
\end{definition}

\begin{definition}[3.4] If $\exists \, d \in C_{r+1}(K)$, $c = \partial_{r+1} d \quad \quad \quad \, (3.25)$, $c$ \, $r$-boundary
\[
\boxed{ B_r(K) = \text{im}{\partial_{r+1}} \equiv r-\text{boundary group} \quad \quad \quad \, B_n(K) =0  }
\]
\end{definition}

\begin{theorem}[3.3] 
\begin{equation}
  B_r(K) \subset Z_r(K) \quad \, (\subset C_r(K)) \quad \quad \quad \, (3.27)
\end{equation}
\end{theorem}

\begin{proof} 
\[  
\forall \, c \in B_r(K), \, \exists \, d \in c_{r+1}(K) \, \text{ s.t. } c = \partial_{r+1}d \quad \quad \, \partial_r c  = \partial_r \partial_{r+1} d = 0 \, (\text{Lemma } 3.3, \, \partial^2 d =0) \quad \, c \in Z_r(K)
\]
\end{proof}

\subsubsection{Homology groups}

\exercisehead{3.1}
  $K = \lbrace p_0, p_1 \rbrace$. $I_r = I_0 = 2$ \, $0-$simplexes.  $c=  c_1 p_0 + c_2 p_1$ \, $0-$chains.  \\
\[
\begin{gathered}
  c_0(K) \simeq \mathbb{Z} \oplus \mathbb{Z} \\
  \partial_0 c = 0 \quad \, \forall \, c \in C_0(K) \quad \, C_0(K) = Z_r(K) \\
  B_0(K) = \text{im}{\partial_1} =0 \\
  \Longrightarrow H_0(K) = \mathbb{Z} \oplus \mathbb{Z} 
\end{gathered}
\]
if $r\neq 0$, $Z_r(K) = 0$ since there are no other simplexes than $0-$simplexes.  $H_r(K) =0$
\begin{equation}
\Longrightarrow H_r(K) = \begin{cases} \mathbb{Z} \oplus \mathbb{Z} & (r=0) \\ 
  \lbrace 0 \rbrace & (r\neq 0) \end{cases} \quad \quad \quad \, (3.34)
\end{equation}

Ex. 3.7. $k = \lbrace p_0, p_1, (p_0 p_1) \rbrace$ \\
\[
\begin{aligned}
  & C_0(K)  = \lbrace i p_0 + j p_1 | i,j \in \mathbb{Z} \rbrace \\ 
  & C_1(K) = \lbrace k(p_0 p_1) | k \in \mathbb{Z}
\end{aligned}
\]

$B_1(K) = 0$, since $(p_0 p_1)$ not a boundary of any simplex in $K$ i.e. $\exists \, d \in K$ s.t. $\partial_2 d = (p_0 p_1)$, since $\nexists \, $ 2-simplex 
\[
H_1(K) = Z_1(K)/ B_1(K) = Z_1(K)
\]

If $z = m(p_0 p1_) \in \mathbb{Z}_1(K)$, 
\[
\partial_1 z = m (p_1-  p_0) = 0 \Longrightarrow m = 0 \quad \, Z_1(K) = 0 
\]
\begin{equation}
  H_1(K) = 0 \quad \quad \quad \, (3.35)
\end{equation}
$Z_0(K) = C_0(K)$ \\

Define surjective (onto) homomorphism.  $\begin{aligned} & \quad  \\ 
  & f: Z_0(K) \to \mathbb{Z} \\
  & f(ip_0 + jp_1) = i+ j \end{aligned}$ 
\[
\begin{gathered}
  \ker{f} = f^{-1}(0) = B_0(K) \\ 
  \partial_1(k(p_0 p_1)) = kp_1 - k p_0 \text{ and } f(kp_1 - k p_0) = 0
\end{gathered}
\]

Thm. 3.1.  $Z_0(K)/\ker{f} \simeq \text{im}{f} = \mathbb{Z}$.  
\begin{equation}
  H_0(K) = Z_0(K)/ B_0(K) \simeq \mathbb{Z} \quad \quad \quad \, (3.30)
\end{equation}

Ex. 3.8.  Triangulation of $S^1$ (\textbf{triangulation of circle}) 
\[
\begin{gathered}
  K = \lbrace p_0, p_1, p_2, (p_0 p_1), (p_1 p_2), (p_2 p_0) \rbrace \\ 
  B_1(K) = 0 \quad \, (\text{no 2-simplices in $K$}) \quad \quad \, H_1(K) = Z_1(K)/B_1(K) = Z_1(K) \\ 
  \partial_1 z = \partial_1( i (p_0 p_1) + j (p_1 p_2) + k (p_2 p_0)) = i (p_1 - p_0 + j (p_2 - p_1) + k(p_0 - p_2) = (k-i) p_0 + (i-j) p_1 + (j-k) p_2 = 0 \\
 \Longrightarrow i =j = k \\
  Z_1(K) = \lbrace i ((p_0 p_1) + (p_1p_2) + (p_2 p_0) ) | i \in \mathbb{Z} \rbrace \simeq \mathbb{Z} \Longrightarrow H_1(K) = Z_1(K) \simeq \mathbb{Z} \quad \quad \, (3.37)
\end{gathered}
\]

\[
\begin{gathered}
  Z_0(K) = C_0(K) \\ 
  B_0(K) = \lbrace \partial_1 [ l (p_0 p_1) + m (p_1 p_2) + n (p_2 p_0) ] | l,m,n \in \mathbb{Z} \rbrace = \lbrace (n-l) p_0 + (l-m) p_1 + (m-n) p_2 | l, m ,n \in \mathbb{Z} \rbrace
\end{gathered}
\]

\exercisehead{3.2} Let $K = \lbrace p_0, p_1, p_2, p_3, (p_0 p_1), (p_1 p_2), (p_2 p_3), (p_3 p_0) \rbrace$ \\
$B_1(K) = 0 \quad \, (KC_2(K) = \emptyset)$ \\
 $H_1(K) = Z_1(K)$ 
\[
\partial_1 ( i (p_0 p_1 ) + j (p_1 p_2) + k (p_2 p_0) + l (p_3 p_0) ) = (l- i) p_0 + (i-j) p_1 + (j-k) p_2 + (k-l) p_3
\]
$\partial_1 c = 0$ if $i = l =j =k$, $Z_1(K) \simeq \mathbb{Z}$.  $H_1(K) \simeq \mathbb{Z}$ \\

$Z_0(K) = C_0(K)$ \quad $(\partial_0 p_i = 0)$ \\
Define surjective homomorphism $\begin{aligned} & \quad \\ 
  & f: C_0(K) \mapsto \mathbb{Z} \\ 
  & f(ip_0 + j p_1 + k p_2 + l p_3) = i + j + k + l \end{aligned}$ 
\[
B_0(K) = \lbrace ap_0 + bp_1 + cp_2 + dp_3 | d = -(a+b+c) \rbrace \quad \quad \quad \, \begin{aligned}
  & a = l - i \\
  & b = i - j \\
  & c = j-k \\ 
  & d = k- l = - (a+b+c) \end{aligned}
\]

\[
\begin{gathered}
  \Longrightarrow B_0(K) = \ker{f} \\
  H_0(K) = Z_0(K)/B_0(K) = C_0(K)/\ker{f} \simeq \text{im}{f} = \mathbb{Z}
\end{gathered}
\]

\exercisehead{3.3} $B_2(K) =0$ \quad $KC_2(K) = \emptyset$
\[
\begin{gathered}
   \partial_2 c_2 = \partial_2 ( i (p_0 p_1 p_2) + j (p_0 p_1 p_3)  + k (p_0 p_2 p_3) + l (p_1 p_2 p_3) ) = \\
   = ( i + l ) (p_1 p_2) +  ( -i + k ) (p_0 p_2) +  ( i + j ) (p_0 p_1) +  ( -j + -k ) (p_0 p_3) +  ( j -l ) (p_1 p_3) +  ( k + l ) (p_2 p_3) = 0 \\
   \Longrightarrow i =k = -l = -j \Longrightarrow Z_2(K) \simeq \mathbb{Z} \text{ so } \boxed{ H_2(K) \simeq \mathbb{Z} }
\end{gathered}
\]

\[
\begin{gathered}
  \partial_1 c_1 = \partial_1 (a(p_0p_1) + b(p_0p_2) + c(p_0p_3) + d(p_1p_2) + e(p_1p_3) + f(p_2p_3)) = \\
  = (-a-b-c) p_0 + (a-d-e) p_1 + (b+d-f) p_2 + (c+e+f) p_3 = 0 
\end{gathered}
\]
By linear algebra,
\[
\left[ \begin{matrix} -1 & - 1 & -1 &  & & \\ 
    1 & & & -1 & -1 & \\
    & 1 & & 1 & & -1 \\ 
    & & 1 &  & 1 & 1 \end{matrix} \right] \left[ \begin{matrix} a \\ b \\ c \\ d \\ e \\ f \end{matrix} \right] = 0 \Longrightarrow \left[ \begin{matrix} 1 & & & -1 & -1 & \\ 
    & 1 & 1 & 1 & 1 &  \\
    & 1 & & 1 & & -1 \\ 
    & & 1 & & 1 & 1 \end{matrix} \right] \Longrightarrow \begin{aligned} & a = d + e \\ 
  & b = -d + f \\ 
  & c = -e -f \end{aligned}
\]

If 
\[
\begin{aligned}
  & a = i + j \\ 
  & b = -i + k \\ 
  & c = -j -k \\ 
  & d = i + l \\
  & e = j - l \\
  & f = k + l 
\end{aligned}
\]

Then $B_1(K) = Z_1(K)$.  Then $Z_1(K)/B_1(K) = 0$ by algebra.  $\boxed{ H_1(K) = 0 }$ \\

Let $\begin{aligned} & \quad \\ 
  & f: Z_0(K) \to \mathbb{Z}  \\
  & f(ip_0 + jp_1 + k p_2 + l p_3) = i + j +k + l \end{aligned}$ \\

$\ker{f} = B_0(K)$ since if
\[
\begin{aligned}
  & i = - ( a + b+ c) \\ 
  & j = a-  d  -e \\ 
  & k = b + d - f \\ 
  & l = c + e+ f 
\end{aligned}
\]
then $i+j + k + l =0$ 
\[
Z_0(K)/ \ker{f} \simeq \text{im}{f} = \mathbb{Z} \Longrightarrow \boxed{ H_0 \simeq \mathbb{Z}}
\]

\subsubsection{Computation of $H_0(K)$}

\begin{theorem}[3.5] Let $K$ be \emph{connected} simplicial complex.  Then
\begin{equation}
  H_0(K) \simeq \mathbb{Z} \quad \quad \quad \, (3.43)
\end{equation}
\end{theorem}

\subsubsection{More homology computations}

Example 3.10.  Fig. 3.8. triangulation of M\"{o}bius strip. \\
$B_2(K) =0$ \, $\partial_2 z =0$ each $(p_0 p_2), (p_1 p_4), (p_2 p_3), (p_4p_5), (p_3 p_1), (p_5p_0)$ appear only once.  $Z_2(K)=0$ 
\begin{equation}
H_2(K)= 0 \quad \quad \quad \quad \, (3.44)
\end{equation}

$H_1(K) = Z_1(K)/B_1(K)$.  $\ker{\partial_1}$ consists of closed loops. (closes on itself so $\partial_1 c_1 =0$) \\

Note $z \sim z'$ if $z - z' = \partial_2 c_2 \in B_1(K)$ \\
easily verify, $H_1(K)$ generated by just $[z]$.  $H_1(K) = \lbrace i [z] | i \in \mathbb{Z} \rbrace \simeq \mathbb{Z} \quad \quad \, (3.45)$ \\

Example 3.11. projective plane $\mathbb{R}P^2$ \\
 
Example $H_2(K)$ from a slightly different view pt.  Add all 2-simplexes in $K$ with same coefficient
\[
z \equiv \sum_{i=1}^{10} m \sigma_{2,i}, \quad \quad \, m \in \mathbb{Z}
\]

Observe that each 1-simplex of $K$ is a common face of exactly 2 2-simplices.  
\begin{equation}
  \partial_2 z = 2m (p_3 p_5) + 2m (p_5 p_4) + 2m (p_4 p_3) \quad \quad \quad \, (3.47)
\end{equation}

$Z_2(K) =0$ since $\partial_2 z = 0$ if $m=0$ \\
Note, any 1-cycle homologous to a multiple of $z = (p_3 p_5) + (p_5 p_4) + (p_4 p_3)$ \\
even multiples of $z$ is a boundary of 2-chain by Eq. 3.4.7.  
\begin{equation}
  H_1(K) = \lbrace [z] | [z] + [z] \sim [0] \rbrace \simeq \mathbb{Z}_2 \quad \quad \quad \, (3.48)
\end{equation}

This example shows that a homology group is not necessarily free Abelian.   \\

Example 3.12. surface of the torus has no boundary, but it's not a boundary of some 3-chain.  \\
Thus, $H_2(T^2)$ freely generated by 1 generator, surface itself, $H_2(T^2) \simeq \mathbb{Z}$ \\
closed loops $a,a'$ homologous since $a'-a = \partial d$ bounds shaded area Fig. 3.10 (could think of $a$ running backwards) \\
See Figure 3.12, triangulation of the Klein bottle (clear from there). \\
inner 1 simplices cancel out to leave only outer 1-simplexes (1 side of the ``square'')
\[
\begin{gathered}
  \begin{aligned}
    z & = \sum m \sigma_{2,i} \\ 
    \partial_2 z & = -2ma  \end{aligned} \\ 
  a = (p_0 p_1 ) + (p_1 p_2 ) + ( p_2 p_0)
\end{gathered}
\]
$\partial_2z =0$ if $m=0$ 
\begin{equation}
  H_2(K) = Z_2(K) \simeq 0 \quad \quad \quad \, (3.50)
\end{equation}
\[
b = (p_0 p_3) + (p_3 p_4) + (p_4p_0)
\]
1-cycle \, $\begin{aligned} & \quad \\ 
  c_1 & = ia + ib \\
  \partial_1 c_1 & = 0 \text{ closed } \end{aligned}$ \\
Now $\partial_2 z = + 2 m a$ so 
\[
2ma \sim 0
\]
Thus, $H_1(K)$ generated by 2 cycles $a,b$ s.t. $a+a =0$ (remember $a$ is ``glued backwards'')
\begin{equation}
  H_1(K) = \lbrace i [a] + j[b] | i,j \in \mathbb{Z} \rbrace \simeq \mathbb{Z}_2 \oplus \mathbb{Z} \quad \quad \, (3.51) 
\end{equation}
\subsection{General properties of homology groups}

\subsubsection{Connectedness and homology groups}

\begin{theorem}[3.6] Let $K  = \bigcup_{k=1}^N K_i$, $K_i K_j \neq \emptyset$.  $K_i$ connected components.  \\
Then 
\[
H_r(K) = \bigoplus_{i=1}^N H_r(K_i)
\]
\end{theorem}

\begin{proof}
  $C_r(K) = \bigoplus_{i=1}^N C_r(K_i)$ (rearrange cycles) \\
  since $Z_r(K_i) \supset B_r(K_i)$, $H_r(K_i)$ well-defined.  
\end{proof}

\subsubsection{Structure of homology groups}

\begin{definition}[3.6] Let $K$ simplicial complex.  
\begin{equation}
  b_r(K) \equiv \dim{H_r(K;\mathbb{R})} \quad (\text{$i$th Betti number}) \quad \quad \quad \, (3.56)
\end{equation}
\end{definition}

\begin{theorem}[3.7] (The Euler-Poincar\`{e} theorem) Let $K$ $n-$dim. simplicial complex, $I_r$ number of $r$-simplexes in $K$.  
\begin{equation}
  \chi(K) \equiv \sum_{i=0}^n (-1)^r I_r = \sum_{r=0}^n (-1)^r b_r(K) \quad \quad \quad \, (3.57)
\end{equation}
\end{theorem}

\problemhead{3.1}

Let $S^2$ with $h$ handles and $q$ holes $=X$ 
\[
C_3(X) \bigcap X= \emptyset \text{ so } B_2(X) = 0 \quad \, Z_2(X) \simeq \mathbb{Z} \, (1 \text{ surface }) \, \boxed{ H_2(X) \simeq \mathbb{Z} \quad \, b_2(X) = 1 }
\]

$h$ handles.  Think of $T^2$.  \\
$q$ holes.  For orientable case, the first hole does not change the homology, i.e.
\[
\partial c_1 =0 \text{ but for } d = \sum_{\sigma_2 \in C_2(X)} m\sigma_2, \, \partial d = c_1 \quad \, c_1 = 0
\]

\[
\boxed{ H_1(X) = \mathbb{Z}^{2h + q -1} }
\]

The sphere is connected (and note compact).  $\boxed{ H_0(X) = \mathbb{Z} }$

\problemhead{3.2}

Each cross cap hole contributes to $H_1(X)$ a 1 cycle $z_q$ s.t. $2z_q \sim 0$.  Thus, 
\[
\boxed{ H_1(X) = \mathbb{Z}_2^q }
\]
Otherwise, 
\[
\boxed{ \begin{aligned}
    & H_2(X) = \mathbb{Z} \\ 
    & H_0(X) = \mathbb{Z} \end{aligned} }
\]








\section{Homotopy Groups}





\section{Manifolds}

\subsection{Manifolds}

\subsubsection{Heuristic introduction}

\subsubsection{Definitions}

\subsubsection{Examples}

\exercisehead{5.1} 

\[
S^n = \lbrace ( x_1 \dots x_{n+1} \rbrace \in \mathbb{R}^{n+1} | \sum_{i=1}^{n+1} x_i^2 = 1 \rbrace \subset \mathbb{R}^{n+1}
\]

Let $\begin{aligned} & \quad \\
  & N = (0 \dots 0, 1) \\ 
  & S = (0 \dots 0, -1) \end{aligned}$ \quad \, $x\in S^n$.  Consider $t(x-N) + N = tx + (1-t)N$ when $x_{n+1} =0$ 
\[
\begin{gathered}
  tx_{n+1} + (1-t) = 0 \text{ or } tx_{n+1} + -1 + t = 0 \\ 
  \Longrightarrow \frac{ 1}{ 1- x_{n+1} } = t \quad \left( \text{ or } \frac{1}{ 1 + x_{n+1} } \right)
\end{gathered}
\]

\[
\begin{aligned}
  & \pi_1: S^n - N \to \mathbb{R}^n \\ 
  & \pi_1(x_1 \dots x_{n+1}) = \left( \frac{x_1}{ 1 - x_{n+1} } \dots \frac{x_n}{ 1 - x_{n+1} } , 0 \right) \\ 
  & \pi_2 : S^n -S \to \mathbb{R}^n \\ 
  & \pi_2(x_1 \dots x_{n+1}) = \left( \frac{x_1}{ 1 + x_{n+1}} \dots \frac{x_n}{ 1 + x_{n+1} }, 0 \right)
\end{aligned}
\]
Note, for $y_i = \frac{x_i}{ 1 - x_{n+1}}$
\[
\begin{gathered}
  y_1^2 + \dots +y_n^2 = |y|^2 = \frac{1-  x_{n+1}^2}{ (1-x_{n+1})^2 } = \frac{1+ x_{n+1}}{ 1- x_{n+1}} \text{ or } x_{n+1}  = \frac{ |y|^2 - 1 }{ |y|^2 + 1 }  \\
  x_i = y _i (1- x_{n+1}) = \frac{2y_i}{ 1 + |y|^2 }
\end{gathered}
\]

\[
\begin{aligned}
  & \pi_1^{-1}: \mathbb{R}^n \to S^n - N \\ 
  & \pi_1^{-1}(y_1 \dots y_n) = \left( \frac{2y_1}{ 1 + |y|^2 } \dots \frac{2y_n}{1+ |y|^2 } , \frac{ |y|^2 - 1 }{ |y|^2 + 1 } \right) \\ 
  & \pi_2^{-1}(y_1 \dots y_n) = \left( \frac{2y_1}{ 1 + |y|^2 } \dots \frac{2y_n}{1+ |y|^2 } , \frac{ 1 - |y|^2  }{ |y|^2 + 1 } \right)
\end{aligned}
\]

$\pi_1,\pi_2$ diff. injective, and $(S^n - N ) \bigcup (S^n-S) = S^n$ \\

Consider $(S^n - N)(S^n-S) = S^n - N \bigcup S$ 
\[
\begin{aligned}
  & \pi_1 \pi_2^{-1}(y_1 \dots y_n) = \left( \frac{y_1 }{ |y|^2} \dots \frac{y_n}{|y|^2} , 0 \right) \\ 
  & \pi_2 \pi_1^{-1}(y_1 \dots y_n) = \left( \frac{y_1 }{ |y|^2} \dots \frac{y_n}{|y|^2} , 0 \right) 
\end{aligned}
\]
since, for example, 
\[
\begin{gathered}
\frac{   \frac{2y_i }{ 1 + |y|^2} }{ 1 - \frac{ 1 - |y|^2}{ 1 + |y|^2 }} = \frac{y_i }{ |y|^2 }
\end{gathered}
\]



$\pi_1 \pi_2^{-1}$ bijective and $C^{\infty}$, $\pi_1 \pi_2^{-1}$ diffeomorphism.   \\

$\lbrace (S^n-N, \pi_1), (S^n - S, \pi_2) \rbrace$ \, $C^{\infty}$ atlas or differentiable structure.  


\subsection{The calculus on manifolds}



\subsubsection{Differentiable maps}

\exercisehead{5.2} If $f:M\to N$ smooth, \\
Consider $\begin{aligned} & \quad \\ 
& (U'_a, \varphi'_a) \in \mathcal{A} \\ 
& (V'_b, \psi'_b)\in\mathcal{B}\end{aligned}$ \, s.t. $f(U_a') \subseteq V_b'$ \\

Consider 
\[
\psi'_b f(\varphi_a')^{-1} = \psi_b'(\psi_{\beta}^{-1} \psi_{\beta}) f (\varphi_{\alpha}^{-1} \varphi_{\alpha}) (\varphi'_a)^{-1} = (\psi_b'\psi_{\beta}^{-1})(\psi_{\beta} f \varphi_{\alpha}^{-1})(\varphi_{\alpha} \varphi'_a)^{-1}
\]

$\begin{aligned} & \quad \\ 
  & U_{\alpha} U_a' \neq \emptyset \\ 
  & V_{\beta}V'_b \neq \emptyset \end{aligned}$ since $\begin{gathered} \quad \\ 
  x \in U_{\alpha}, U_a' \\ 
  f(x) \in V_{\beta} V'_{\beta} \end{gathered}$ \\

Then $(\psi_b' \psi_{\beta}^{-1}), (\varphi_{\alpha} \varphi'_a)^{-1}$ , \, $C^{\infty}$ (def. of atlas) and $(\psi_{\beta} f \varphi_{\alpha}^{-1})$ \, $C^{\infty}$ (given).  \\
So $\psi_b' f(\varphi_a')^{-1}$ \, $C^{\infty}$

\subsubsection{Vectors}

curve $c:I \to M$ \\
\phantom{curve } $f:M \to \mathbb{R}$ 

\[
\frac{d}{dt} \left. f(c(t)) \right|_{t=0} \quad \quad \quad \, (5.18)
\]

for $p = c(0) \in (U, \varphi) = (U, x^1 \dots x^n)$, \, $fc = f\varphi^{-1}\varphi c$
\[
\frac{d}{dt} \left. f(c(t)) \right|_{t=0} =   \frac{ \partial (f\varphi^{-1}) }{ \partial x^{\mu} }(x) \left. \frac{ d (x^{\mu}(c)) }{ dt}(t) \right|_{t=0} = \frac{ \partial f}{ \partial x^{\mu}} \left. \frac{dx^{\mu}(c(t)) }{dt} \right|_{t=0} = \left. \frac{ \partial f}{ \partial x^{\mu} } \frac{d c^{\mu}}{dt} \right|_{t=0} \quad \quad \, (\text{abuse of notation})
\]

Note that abuse of notation: $ \left. \frac{ dx^{\mu} }{dt}(c(t)) \right|_{t=0 } \equiv \left. \frac{d x^{\mu} ( \varphi{ (c(t))  }) }{dt} \right|_{t=0}$

$ \left. \frac{df(c(t))}{dt} \right|_{t=0}$ obtained by differential operator $X$ to $f$
\[
X = X^{\mu}\left( \frac{ \partial }{ \partial x^{\mu} } \right) \quad \quad \, \left( X^{\mu} = \left. \frac{dx^{\mu}(c(t))}{dt} \right|_{t=0} \right) \quad \quad \quad \, (5.20)
\]
\[
\left. \frac{df(c(t))}{ dt} \right|_{t=0} = X^{\mu} \left( \frac{ \partial f}{ \partial x^{\mu} } \right) = X[f] \quad \quad \quad \, (5.21)
\]



Introduce equivalence class of curves in $M$.  \\
curves $c_1(t) \sim c_2(t)$ if 
\begin{enumerate}
\item[(1)] $c_1(0) = c_2(0) = p$ 
\item[(2)] $
\left. \frac{dx^{\mu}{ (c_1(t)) } }{ dt} \right|_{t=0} = \left. \frac{dx^{\mu}(c_2(t)) }{ dt} \right|_{t=0}
$
\end{enumerate}

Identify tangent vector $X$ with $[c(t)]$.  

$T_p M$, tangent space of $M$ at $p$, all $[c]$ at $p \in M$ 

Use Sec. 2.2's theory of vector spaces to analyze $T_pM$ 

\hrulefill

Recall curve $c(t) : \mathbb{R} \to M$,  \\
\phantom{Recall curve }  $f: M \to \mathbb{R}$

Now tangent vector at $c(0)$ was, recall
\[
\left. \frac{df(c(t))}{ dt} \right|_{t=0} = \frac{ \partial f}{ \partial x^{\mu}}  \left. \frac{ dx^{\mu}(c(t))}{ dt} \right|_{t=0} = \left. \frac{dx^{\mu}(c(t))}{ dt} \right|_{t=0} \frac{ \partial f}{ \partial x^{\mu} }
\]
Let $X^{\mu} = \left. \frac{dx^{\mu}(c(t))}{ dt} \right|_{t=0}$

\[
\begin{gathered}
X[f] \equiv \left. \frac{df(c(t))}{ dt} \right|_{t=0} \\
\Longrightarrow X = x^{\mu} \frac{ \partial}{ \partial x^{\mu}} \end{gathered}
\]

So tangent vector $X$ is spanned by $\frac{ \partial }{ \partial x^{\mu} }$

Suppose $a^{\mu} \frac{ \partial }{ \partial x^{\mu} } =0$ 
\[
a^{\mu} \frac{ \partial }{ \partial x^{\mu}} x^{\nu} = a^{\mu} \delta_{\mu}^{ \, \, \nu } = a^{\nu} = 0 
\]


So $\lbrace \frac{ \partial }{ \partial x^{\mu}} \rbrace$ linearly independent.  

$\lbrace \frac{ \partial }{ \partial x^{m}} \rbrace$ a basis.  

\hrulefill

(cf. wikipedia) Consider short $\varphi : U \to \mathbb{R}^n$, $p\in U$, define map  $ \begin{aligned} & \quad \\ 
  & ( d\varphi)_p : T_x M \to \mathbb{R}^n \\ 
  & (d\varphi)_p( [c(0 ) ]) = \frac{d}{dt} ( \varphi ( c(0 )) ) \end{aligned}$ 

Consider $c(0) = p = c(t)$ \, $\forall \, t $ s.t. $c(t) \in U$.   \\
Surely $\varphi(c(t)) = $ const. (for $\varphi$ is a well-defined function)  \\
$\frac{d}{dt} \varphi(c(t)) = 0$.  For $[c(0)] = [p]$, $ (d\varphi)_p ([p ]) = 0$ \, $\forall \, $ chart $\varphi$ \, $(d\varphi)_p$ injective.  \\
Consider $a^i e_i \in \mathbb{R}^n$ 
\[
a^i = \frac{d}{dt} ( \varphi^i(c(0)) ) \equiv \left. \frac{d}{dt} x^i(c(t)) \right|_{t=0}
\]

$f: M \to \mathbb{R} \in C^{\infty}(M)$ if $f\varphi^{-1} \in C^{\infty}$.  $\forall \, $ chart $\varphi : U \to \mathbb{R}^n$ 

derivation at $p$ is linear $D : C^{\infty}(M) \to \mathbb{R}$ \, ( or $\begin{aligned} & \quad \\ 
  & X: C^{\infty}(M) \to \mathbb{R} \\ 
  & X[f] =\left.  \frac{df(c(t)) }{ dt} \right|_{t=0} \end{aligned}$ ).  These derivations form a vector space, tangent space $T_pM$. 

\[
\left. \frac{df(c(t))}{dt}  \right|_{t=0} = \frac{ \partial g}{ \partial x^{\mu} }(x^{\nu}) \frac{dx^{\mu}(c(t)) }{ dt} = \frac{dx^{\mu}(c(t)) }{dt} \frac{ \partial}{ \partial x^{\mu} } f
\]
$g(x^{\nu}) = (f\cdot \varphi^{-1})(x^{\nu})$

Consider $p\in U_i U_j$, \, $\begin{aligned} & \quad \\
  & x = \varphi_i(p) \\ 
  & y = \varphi_j(p) \end{aligned}$ 

\[
\begin{gathered}
  X = X^{\mu} \frac{ \partial }{ \partial x^{\mu} } = Y^{\nu} \frac{ \partial }{ \partial y^{\nu}} \\ 
  Y^{\nu} = X^{\mu} \frac{ \partial y^{\nu} }{ \partial x^{\mu} } \quad \quad \quad \, (5.23)
\end{gathered}
\]



\subsubsection{One-forms}

$\begin{aligned}
  & \omega \in T_p^*M \\ 
  & \omega : T_p M \to \mathbb{R} \end{aligned}$ \quad \textbf{cotangent vector or 1-form}.  \\

\textbf{ differential } $df \in T_p^*M$ on $V \in T_pM$
\[
\langle df , V \rangle = V[f] = V^{\mu} \frac{ \partial f}{ \partial x^{\mu} } \in \mathbb{R} \quad \quad \quad \, (5.24)
\]

arbitrary 1-form $\omega$
\[
\omega = \omega_{\mu} dx^{\mu} \quad \quad \quad \, (5.26)
\]

inner product defined
\[
\begin{aligned}
  & \langle \, , \, \rangle : T_p^*M \times T_pM \to \mathbb{R} \\
  & \langle \omega, V \rangle = \omega_{\mu} V^{\mu} \langle dx^{\mu}, \frac{ \partial}{ \partial x^{\nu} } \rangle = \omega_{\mu} V^{\mu} \quad \quad \, (5.27) 
\end{aligned}
\]

For $p \in U_iU_j$, $\begin{aligned} & \quad \\ 
  & x = \varphi_i(p) \\ 
  & y = \varphi_j(p) \end{aligned}$ \\

$\begin{aligned}
  & \quad \\
  & dy^{\nu} \in T_p^*M \text{ so } \\
  & dy^{\nu} = \omega_{\mu} dx^{\mu} \end{aligned}$ so
\[
\begin{gathered}
\langle dy^{\nu} , \frac{ \partial }{ \partial x^{\gamma} } \rangle = \delta^{\mu \gamma} \frac{ \partial y^{\nu} }{ \partial x^{\mu} } = \frac{ \partial y^{\nu} }{ \partial x^{\gamma} } = \omega_{\mu} \langle dx^{\mu}, \partial_{x^{\gamma}} \rangle = \omega_{\gamma} \\
\Longrightarrow dy^{\nu} = \frac{ \partial y^{\nu} }{ \partial x^{\mu} } dx^{\mu} 
\end{gathered}
\]

\[
\omega = \omega_{\mu} dx^{\mu} = \psi_{\nu} dy^{\nu} = \psi_{\nu} \frac{ \partial y^{\nu}}{ \partial x^{\mu}  } dx^{\mu} \text{ or } \omega_{\mu} = \psi_{\nu} \frac{ \partial y^{\nu }}{ \partial x^{\mu}}
\]


\subsubsection{Tensors}

$\tau^q_{r, p }(M)$ set of type $(q,r)$ tensors at $p\in M$.  

element of $\tau^q_{r, p }(M)$
\[
T = T^{\mu_1 \dots \mu_q }_{\phantom{\mu_1 \dots \mu_q} \nu_1 \dots \nu_r } \frac{ \partial }{ \partial x^{\mu_1} } \dots \frac{ \partial}{ \partial x^{\mu_q} } dx^{\nu_1 } \dots dx^{\nu_r} \quad \quad \quad (5.29)
\]
$T: \otimes^q T_p^*M \otimes^r T_pM \to \mathbb{R}$

Let $\begin{aligned} & \quad \\ 
  & V_i = V_i^{\mu} \frac{ \partial }{ \partial x^{\mu }} \quad \quad ( 1 \leq i \leq r) \\ 
  & \omega_i = \omega_{i \mu} dx^{\mu} \quad \quad ( 1 \leq i \leq q ) \end{aligned}$

\[
T(\omega_1 \dots \omega_q, V_1 \dots V_r) = T^{\mu_1 \dots \mu_q }_{\phantom{\mu_1 \dots \mu_q} \nu_1 \dots \nu_r } \omega_{1 \mu_1} \dots \omega_{q \mu_q} V_1^{\nu_1} \dots V_r^{\nu_r}
\]

%\[
%T(\omega_1 \dots a_i \omega_i + b_i \psi_i \dots \omega_q; V_1 \dots c_j V_j + d_j W_j \dots V_r) = T^{\mu_1 \dots \mu_q}_{\phantom{\mu_1 \dots \mu_q} \nu_1 \dots nu_r } \omega_{1 \mu_1} \dots ( a_i \omega_{i \mu_i} + b_i \psi_{i \mu_i} ) \dots \omega_{q \mu_q} V_1^{\nu_1} \dots \omega_j V_j^{\nu_j} 
%\]

\[
\begin{gathered}
  T(\omega_1 \dots a_i \omega_i + b_i \psi_i \dots \omega_q, V_1 \dots V_r) = T^{\mu_1 \dots \mu_q}_{\phantom{\mu_1 \dots \mu_q} \nu_1 \dots \nu_r } \omega_{1 \mu_1} \dots (q_i \omega_{i \mu_i} + b_i \psi_{i \mu_i} ) \dots \omega_{q \mu_q} , V_1^{\nu_1} \dots V_r^{\nu_r} ) =  \\
  = a_i T^{\mu_1 \dots \mu_q}_{\phantom{\mu_1 \dots \mu_q} \nu_1 \dots \nu_r } \omega_{1 \mu_1} \dots \omega_{i \mu_i} \dots \omega_{q \mu_q} , V_1^{\nu_1} \dots V_r^{\nu_r} + b_i T^{\mu_1 \dots \mu_q}_{\phantom{\mu_1 \dots \mu_q} \nu_1 \dots \nu_r } \omega_{1 \mu_1} \dots \psi_{i \mu_i} \dots \omega_{q \mu_q} , V_1^{\nu_1} \dots V_r^{\nu_r} = \\
  = a_i T(\omega_1 \dots \omega_i \dots \omega_q, V_1 \dots V_r) + b_i T(\omega_1 \dots \psi_i \dots \omega_q, V_1 \dots V_r)
\end{gathered}
\]
\[
\begin{gathered}
  T(\omega_1 \dots \omega_q, V_1 \dots c_j V_j + d_j W_j \dots V_r) = T^{ \mu_1 \dots \mu_q}_{\phantom{\mu_1 \dots \mu_q} \nu_1 \dots \nu_r} \omega_{1\mu_1} \dots \omega_{q \mu_q} V_1^{\nu_1} \dots (c_j V_j^{\nu_j} + d_j W_j^{\nu_j} \dots V_r = \\
  = c_j T^{\mu_1 \dots \mu_q}_{\phantom{\mu_1 \dots \mu_q} \nu_1 \dots \nu_r } \omega_{1 \mu_1} \dots \omega_{q \mu_q}  V_1^{\nu_1} \dots V_j^{\nu_j} \dots V_r + d_j T^{\mu_1 \dots \mu_q}_{\phantom{\mu_1 \dots \mu_q} \nu_1 \dots \nu_r } \omega_{1 \mu_1} \dots \omega_{q \mu_q}  V_1^{\nu_1} \dots W_j^{\nu_j} \dots V_r = \\
      = c_j T(\omega_1 \dots \omega_q, V_1 \dots V_j \dots V_r) + d_j T(\omega_1 \dots \omega_q, V_1 \dots W_j \dots V_r)
\end{gathered}
\]

\subsubsection{Tensor fields}

vector field - vector assigned smooth $\forall \, p \in M$ \\
\quad i.e. $V$ vector field if $V[f] \in \mathcal{F}(M)$ \quad \, $\forall \, f \in \mathcal{F}(M)$ \\

tensor field of type $(q,r)$, \, $\tau^q_{r,p}(M)$ \, $\forall \, p \in M$ \\
$\tau_1^0(M)$ set of dual vector fields $\equiv \Omega^1(M)$ \\
$\tau_0^0(M) = \mathcal{F}(M)$

\subsubsection{ Induced maps}

smooth $f:M \to N$ \\

differential $f_*:T_pM \to T_{f(p)}N$ \\
By def. of tangent vector as directional derivative along a curve, \\
\quad if $g \in \mathcal{F}(N)$, $fg \in \mathcal{F}(M)$ \\
vector $V \in T_pM$ acts on $gf$ to give number $V[gf]$ \\

define $\begin{aligned} & \quad \\
  & f_* V \in T_{f(p)}N  \\
  & (f_*V)[g] = V[gf] \quad \quad \, (5.31) \end{aligned}$ \\

or for $\begin{aligned} & \quad \\ 
  & (U,\varphi) \subset M \\ 
  & (V, \psi) \subset N \end{aligned}$ 

\[
(f_*V)[g\psi^{-1}(y) ] \equiv V[gf\varphi^{-1}(x) ] \quad \quad \quad \, (5.32)
\]

Let $V = V^{\mu} \frac{ \partial }{ \partial x^{\mu} }$, \, $f_* V = W^{\alpha} \frac{ \partial }{ \partial y^{\alpha} }$
\[
W^{\alpha} \frac{ \partial }{ \partial y^{\alpha} } [ g\psi^{-1}(y) ] = V^{\mu} \frac{ \partial }{ \partial x^{\mu} }[gf\varphi^{-1}(x) ]
\]

With $y = \psi(f(p))$, \\
take $g= y^{\alpha}$ 
\[
W^{\alpha} = V^{\mu} \frac{ \partial y^{\alpha} }{ \partial x^{\mu} } \quad \quad \quad \, (5.33)
\]

$f(p) = f\varphi^{-1}(x)$
\[
V^{\mu} \frac{ \partial }{ \partial x^{\mu} }[ y^{\alpha} f \varphi^{-1}(x) ] = V^{\mu} \frac{ \partial }{ \partial x^{\mu} }[y^{\alpha} ]
\]

\hrulefill
20121030

Consider smooth $\begin{aligned} & \quad \\ 
  & f: M \to N \\
  & f_* : T_p M \to T_{f(p)}N \end{aligned}$  

\[
\begin{aligned}
  & (U , \varphi) \subset M \quad \quad \, \varphi = x^{\mu} \\ 
  & (V, \psi ) \subset N \quad \quad \, \psi = y^{\nu} \end{aligned}
\]

\[
\begin{aligned}
  & \psi f \varphi^{-1} : \varphi(U) \subseteq \mathbb{R}^m \to \mathbb{R}^n \\ 
  & \psi f\varphi^{-1} \equiv f^{\nu}(x^{\mu })
\end{aligned} \quad \quad \, \begin{aligned} 
  & X \in T_p M \\
  & Y \in T_{f(p)} N \end{aligned} \quad \quad \, \begin{aligned} 
  X & = X^{\mu} \frac{ \partial }{ \partial x^{\mu } } \\
  Y & = Y^{\nu} \frac{ \partial }{ \partial y^{\nu } } \end{aligned}
\]

\[
\begin{aligned}
  & f_* X \in T_{f(p)} N \\ 
  & f_* X   = Y^{\nu} \frac{ \partial }{ \partial y^{\nu }}
\end{aligned}
\]
\[
\begin{aligned}
  & g \in \mathcal{F}(N) \\ 
  & g: N \to \mathbb{R}
\end{aligned} \quad \quad \, 
\begin{aligned}
  & g\psi^{-1}: \psi(N) \to \mathbb{R} \\ 
  & g\psi^{-1} \equiv g(y^{\nu} )
\end{aligned} 
\]
\[
\begin{aligned}
  & gf : M \to \mathbb{R} \\ 
  & gf \in \mathcal{F}(M)
\end{aligned}
\]

\[
\begin{aligned}
  & (f_* X)[g] \equiv X[gf] \\ 
  & (f_* X)[g] = Y^{\nu} \frac{ \partial g}{ \partial y^{\nu }}(y) = X^{\mu} \frac{ \partial }{ \partial x^{\mu }}[gf]  = X^{\mu} \frac{ \partial }{ \partial x^{\mu } } g( f^{\nu}(x) ) = X^{\mu} \frac{ \partial g}{ \partial y^{\nu} }(y) \frac{ \partial f^{\nu} }{ \partial x^{\mu} }(x)
\end{aligned}
\]
\[
\Longrightarrow \boxed{ Y^{\nu} = X^{\mu } \frac{ \partial f^{\nu }}{ \partial x^{\mu }}  = X^{\mu} \frac{ \partial y^{\nu} }{ \partial x^{\mu }} }
\]
where
\[
\begin{aligned}
  & gf = g \psi^{-1} \psi f \varphi^{-1} \\ 
  & g\psi^{-1} \psi f \varphi^{-1} : \varphi(U) \to \mathbb{R} \\ 
  & g\psi^{-1} \psi f \varphi^{-1} = g(f^{\nu}(x) ) 
\end{aligned}
\]


\exercisehead{5.3}

$\begin{aligned} & \quad \\ 
  f : & M \to N \\
  g : & N \to P \\ 
  gf : & M \to P \end{aligned}$ \quad \, Consider $\begin{aligned} & \quad \\ 
  p \in M , & (U, \varphi) \subset M  \\
  q = f(p) \in N , & (V, \psi) \subset N \\
  r = g(q) \in P , & (W, \chi) \subset P \end{aligned}$ \quad \, $\begin{aligned} & \quad \\ 
  & V \in T_pM \\
  & W \in T_qN \\
  & X \in T_{g(q)}P \end{aligned}$ \quad \, $\begin{aligned} & \quad \\ 
  & V = V^{\alpha} \frac{ \partial }{ \partial x^{\alpha} } \\ 
  & W = W^{\beta} \frac{ \partial }{ \partial y^{\beta} } \\ 
  & X = X^{\gamma} \frac{ \partial }{ \partial z^{\gamma} } \end{aligned}$

\[
\begin{gathered}
\begin{aligned}
  & f_*V \in T_{f(p)}N \\ 
  & f_*V[h] = V[hf] \\ 
  & f_*V[h\psi^{-1}(y)] = V[hf\varphi^{-1}(x)]
\end{aligned} \quad \quad \, \begin{aligned}
  & g_*W \in T_{g(q)}P \\ 
  & g_*W[k] = W[kg] \\ 
  & g_*W[k\chi^{-1}(z)] = W[kg\psi^{-1}(y)]
\end{aligned}  \quad \quad \, \begin{aligned}
  & gf_*V \in T_{gf(p)}P \\ 
  & gf_*V[l] = V[lgf] \\ 
  & gf_*V[l\chi^{-1}(z)] = V[lgf\varphi^{-1}(x)]
\end{aligned} \\
\end{gathered}
\]

\[
\Longrightarrow g_*(f_*V)[l] = f_* V[lg] = V[lgf] = gf_*V[l]
\]

In coordinates, 
\[
\begin{gathered}
  g_*(f_*V)[l\chi^{-1}(z) ] = f_*V[lg\psi^{-1}(y)] = W^{\alpha} \frac{ \partial }{ \partial y^{\alpha}}[ lg\psi^{-1}(y)] = V^{\mu} \frac{ \partial y^{\alpha} }{ \partial x^{\mu} } \frac{ \partial }{ \partial y^{\alpha} } [lg\psi^{-1}(y)] = V^{\mu} \frac{ \partial (y^{\alpha}f\varphi^{-1}(x)) }{ \partial x^{\mu} } \frac{ \partial }{ \partial y^{\alpha} } [lg\psi^{-1}(y) ] \\
  gf_*V[l \chi^{-1}(z) ] = V[lgf\varphi^{-1}(x) ] = V^{\alpha} \frac{ \partial }{ \partial x^{\alpha} }( lgf\varphi^{-1}(x)) \\
  \Longrightarrow  \frac{ \partial }{ \partial x^{\mu} }( lgf\varphi^{-1}(x)) = \frac{ \partial (y^{\alpha}f\varphi^{-1}(x)) }{ \partial x^{\mu} } \frac{ \partial }{ \partial y^{\alpha} } [lg\psi^{-1}(y) ]
\end{gathered}
\]
Chain rule is reobtained. \\

\[
\begin{aligned}
  & f: M \to N \\ 
  & f^* : T^*_{f(p)}N \to T^*_p M \quad \quad \, \text{ \textbf{ pull back }}
  & \langle f^* \omega, V \rangle = \langle \omega, f_* V \rangle \\
  & \begin{aligned}
      & \omega \in T^*_{f(p)}N \\ 
      & V \in T_p M \end{aligned}
\end{aligned}
\]


tensor of type $(0,r)$ $f^*: \tau^0_{r, f(p)}(N) \to \tau^0_{r,p}(M)$

\hrulefill 

20121030

pullback

\[
\begin{gathered}
\begin{aligned}
  & f: M \to N \\ 
  & f^* : T^*_{f(p)} N \to T^*_p M \end{aligned} \quad \quad \, \begin{aligned} & X \in T_pM \\ 
  & X = X^{\mu} \frac{ \partial }{ \partial x^{\mu } } \end{aligned} \\ 
\begin{aligned}
  & (U , \varphi ) \subset M \quad \, \varphi = x^{\mu} \\ 
  & ( V, \psi ) \subset N  \quad \quad \, \psi = y^{\nu } \end{aligned} \quad \quad \, 
\begin{aligned}
  & \omega \in T^*_{f(p)} N \\ 
  & \omega = \omega_{\alpha}dy^{\alpha}
\end{aligned}
\end{gathered}
\]

\[
\begin{aligned}
  & f^* \omega \in T^*_pM \text{ so } \\ 
  & f^* \omega = \psi_{\beta} dx^{\beta} 
\end{aligned}
\]

\[
\begin{gathered}
  \langle f^* \omega, X \rangle = \langle \psi_{\beta} dx^{\beta}, X^{\mu} \frac{ \partial }{ \partial x^{\mu }} \rangle = \psi_{\mu} X^{\mu} = \langle \omega, f_* X \rangle = \langle \omega_{\alpha} dy^{\alpha}, Y^{\nu} \frac{ \partial }{ \partial y^{\nu }} \rangle = \\
  = \langle \omega_{\alpha} dy^{\alpha}, X^{\mu} \frac{ \partial y^{\nu }}{ \partial x^{\mu} } \frac{ \partial }{ \partial y^{\nu} } \rangle = \omega_{\nu} X^{\mu} \frac{ \partial y^{\nu }}{ \partial x^{\mu }}
\end{gathered}
\]
\[
\boxed{ \psi_{\mu} = \omega_{\nu} \frac{ \partial y^{\nu }}{ \partial x^{\mu }} = \frac{ \partial y^{\nu }}{ \partial x^{\mu }} \omega_{\nu} }
\]

\exercisehead{5.4} 
\[
\begin{aligned}
  & \omega = \omega_{\alpha} dy^{\alpha} \in T^*_{f(p)}N \\ 
  & f^*\omega = \xi_{\mu} dx^{\mu} \in T_p^* M
\end{aligned}
 \quad \quad \, \begin{aligned}
& V \in T_p M \\ 
 & V = V^{\mu} \frac{ \partial }{ \partial x^{\mu}}
\end{aligned}
\]

\[
\begin{gathered}
  \langle f^* \omega , V \rangle = \langle \xi_{\mu} dx^{\mu}, V^{\nu} \frac{ \partial }{ \partial x^{\nu}} \rangle = \xi_{\mu} V^{\mu} = \\
    = \langle \omega, f_* V \rangle = \langle \omega_{\alpha} dy^{\alpha} , W^{\beta} \frac{ \partial }{ \partial y^{\beta}} \rangle = \langle \omega_{\alpha} dy^{\alpha} , V^{\mu} \frac{ \partial y^{\beta}}{ dx^{\mu} } \frac{ \partial }{ \partial y^{\beta}} \rangle = \omega_{\alpha} V^{\mu} \frac{ \partial y^{\beta}}{ \partial x^{\mu} } \delta^{\alpha}_{ \, \beta} = \omega_{\alpha} V^{\mu} \frac{ \partial y^{ \alpha} }{ \partial x^{\mu} } = \\
    \Longrightarrow \boxed{  \xi_{\mu} = \omega_{\alpha} \frac{ \partial y^{\alpha} }{ \partial x^{\mu} } }
\end{gathered}
\]

\exercisehead{5.5} \[
(gf)^*: T^*_{ gf(p)}P \to T^*_p M
\]
\[
\begin{gathered}
  \langle (gf)^* \chi, V \rangle = \langle \chi, (gf)_* V \rangle = \langle \chi, g_* f_*V \rangle = \langle g^* \chi, f_* V \rangle = \langle f^* g^* \chi, V \rangle \\
  \Longrightarrow (gf)^* = f^* g^*
\end{gathered}
\]
with
\[
\begin{aligned}
  & \begin{aligned}
      & g^* : T^*_{g(q)}P \to T^*_q N \\ 
  & \langle g^* \omega , W \rangle = \langle \omega, g_* W \rangle 
\end{aligned} \quad \, 
  & \begin{aligned}
      & f^* : T^*_{f(p)}N \to T^*_p M \\ 
      & \langle f^* \nu , V \rangle = \langle \nu, f_* V \rangle
    \end{aligned}
\end{aligned}
\]

\subsubsection{ Submanifolds}

\begin{definition}[5.3] (Immersion, submanifold, embedding) \\
Let smooth $f: M \to N $ \\
$\text{dim}{M} \leq \text{dim}{N}$ 

\begin{enumerate}
\item[(a)] immersion $f$ if $f_* : T_pM \to T_{f(p)} N$ injection (1-to-1) i.e. 
\[
\text{rank}{f_*} = \text{dim}{M}
\]
\item[(b)] embedding $f$ if $f$ injection and immersion \\
$f(M)$ submanifold of $N$.  
\end{enumerate}
\end{definition}

\subsection{ Flow and Lie derivatives }


Let vector field $X$ in $M$.  \\
integral curve $x(t)$ of $X$ is a curve in $M$, tangent vector at $x(t)$ is $\left. X \right|_x$ 
\[
\frac{dx^{\mu}}{ dt} = X^{\mu}(x(t))
\]
where $x^{\mu}$ is $\mu$th component of $\varphi(x(t))$, $X = X^{\mu} \frac{ \partial }{ \partial x^{\mu }}$


Note the abus of notation: $x$ used to denote a pt. in $M$ as well as its coordinates.  

Let $\sigma(t,x_0)$ integral curve of $X$ which passes a pt. $x_0$ at $t=0$.  
\begin{gather}
  \frac{d}{dt} \sigma^{\mu}(t,x_0) = X^{\mu}(\sigma(t,x_0)) \quad \quad \, (5.40a) \\ 
  \sigma^{\mu}(0, x_0) = x_0^{\mu} \quad \quad \, (5.40b) \text{ initial condition }
\end{gather}

$\sigma : \mathbb{R} \times M \to M$.  Flow generated by $X \in \mathcal{X}(M)$ s.t. 
\[
\sigma(t, \sigma^{\mu}(s,x_0) ) = \sigma( t+ s ,x_0)
\]

The previous was true because of the following.  From uniqueness of ODEs.
\[
\begin{gathered}
  \begin{gathered} 
    \frac{d}{dt} \sigma^{\mu}(t, \sigma^{\mu}(s,x_0) ) = X^{\mu}( \sigma(t, \sigma^{\mu}(s,x_0)) ) \\ 
    \sigma^{\mu}(0 , \sigma(s,x_0) ) = \sigma(s,x_0) \end{gathered} \\ 
  \begin{gathered}
    \frac{d}{dt} \sigma^{\mu}( t+ s ,x_0) = \frac{d}{d (t+s) } \sigma^{\mu}(t+s , x_0 ) = X^{\mu}(\sigma(t+s, x_0)) \\ 
    \sigma( 0 + s, x_0) = \sigma(s,x_0)
\end{gathered}
\end{gathered}
\]

\begin{theorem}[5.1] $\forall \, x \in M$, $\exists \, $ differentiable $\sigma: \mathbb{R} \times M \to M$ s.t. 
\begin{enumerate}
\item[(i)] $\sigma(0,x) = x$ 
\item[(ii)] $t \mapsto \sigma(t,x)$ solution of (5.40a), (5.40b) 
\item[(iii)] $\sigma(t, \sigma^{\mu}(s,x)) = \sigma(t+s,x)$
\end{enumerate}
\end{theorem}

Example 5.9. Let $M = \mathbb{R}^2$, $X((x,y)) = -y \frac{ \partial}{\partial x} + x \frac{ \partial }{ \partial y }$
\[
X((x,y))[f] = -y \frac{ \partial f}{ \partial x} + x \frac{ \partial f}{ \partial y } = \frac{df(\sigma(t)) }{ dt} = \frac{ dx^{\mu}(c(t)) }{ dt} \frac{ \partial f}{ \partial x^{\mu} }
\]

\[
\begin{gathered}
  \begin{aligned}
    \frac{dx}{dt} & = -y \\ 
    \frac{dy}{dt} & = x 
\end{aligned} \quad \quad \, \frac{ d^2y }{dt^2} = -y \Longrightarrow \begin{aligned} y & = Ac(t) = Bs(t) \\ x & = - As(t) + Bc(t) \end{aligned} \quad \quad \, \Longrightarrow \begin{aligned} A & = y \\ B & = x \end{aligned} \\ 
  \begin{aligned}
    x & =  x \cos{(t) } -y \sin{(t) }  \\
    y & = x \sin{(t) } + y \cos{(t)}
\end{aligned}
\end{gathered}
\]

\exercisehead{5.7}

\[
\begin{gathered}
  \begin{gathered}
    X = y \frac{ \partial }{ \partial x } + x \frac{ \partial }{ \partial y} \\ 
    X[f] = y \frac{ \partial f}{ \partial x} + x \frac{ \partial f}{ \partial y} = \frac{ dx^{\mu} }{dt} \frac{ \partial f}{ \partial x^{\mu} }
\end{gathered} \\
\begin{aligned}
  \frac{dx}{dt} & = y \\ 
  \frac{dy}{dt} & = x 
\end{aligned} \quad \quad \, \begin{aligned}
  \ddot{x} & = x \\ 
  \ddot{y} & = y \end{aligned} \quad \quad \, \begin{aligned}
x & = Ae^t + Be^{-t} \\ 
y & = Ae^t - Be^{-t} \end{aligned}
\end{gathered}
\]



\subsubsection{ One-parameter group of transformations}

For fixed $t\in \mathbb{R}$, flow $\sigma(t,x)$ is a diffeomorphism from $M$ to $M$, $\sigma_t : M \to M$ \\
$\sigma_t$ made into a commutative group by 
\begin{enumerate}
\item[(i)] $\sigma_t(\sigma_s(X)) = \sigma_{t+s}(X)$ i.e. $\sigma_t \circ \sigma_s = \sigma_{t+s}$ 
\item[(ii)] $\sigma_0 = 1$ 
\item[(iii)] $\sigma_{-t} = (\sigma_t)^{-1}$
\end{enumerate}

Under action $\sigma_{\epsilon}$, infinitesimal $\epsilon$, from (5.40a), (5.40b)
\[
\sigma_{\epsilon}^{\mu}(x) = \sigma^{\mu}(\epsilon, x) = x^{\mu} + \epsilon X^{\mu}(x)
\]
So vector field $X$ is the infinitesimal generator of transformation $\sigma_{\epsilon}$

Given a vector field $X$, corresponding flow $\sigma$ of referred to as exponentiation of $X$
\[
\sigma^{\mu}(t,x) = \exp{ (tX) } x^{\mu} \quad \quad \quad \, (5.43)
\]

Since 
\[
\begin{gathered}
\sigma^{\mu}(t,x) = x^{\mu} + t \left. \frac{d}{ds} \sigma^{\mu}(s,x) \right|_{s=0} + \frac{t^2}{2!} \left. \left( \frac{d}{ds} \right)^2 \sigma^{\mu}(s,x) \right|_{s=0} + \dots = \left[ 1 + t \frac{d}{ds} + \frac{t^2}{2!} \left( \frac{d}{ds} \right)^2 + \dots \right] \left. \sigma^{\mu}(s,x) \right|_{s=0} = \\
= \exp{ \left( t \frac{d}{ds} \right) } \left. \sigma^{\mu}(s,x) \right|_{s =0 } \quad \quad \, (5.44) 
\end{gathered}
\]

Properties
\begin{enumerate}
\item[(i)] $\sigma(0,x) = x = \exp{ ( 0X) } x $ \quad \quad \quad \, (5.45a) 
\item[(ii)] $\frac{ d\sigma (t,x) }{ dt} = X\exp{(tX)}x = \frac{d}{dt} [ \exp{ (tX) } x ]$ \quad \quad \quad \, (5.45b)
\item[(iii)] $\sigma(t,\sigma(s,x) ) = \sigma(t, \exp{(sX)} x) = \exp{ (tX)} \exp{(sX) }x = \exp{ [ (t+s) X ] }x = \sigma(t+s,x)$
\end{enumerate}

\subsubsection{ Lie derivatives } 

Let $\sigma(t,x)$, $\tau(t,x)$ be 2 flows generated by vector fields $X,Y$ 
\[
\begin{aligned}
  & \frac{ d\sigma^{\mu}(s,x) }{ ds} = X^{\mu}(\sigma(s,x)) \quad \quad \quad \, (5.46a) \\ 
  & \frac{d\tau^{\mu}(t,x) }{ dt} = Y^{\mu}(\tau(t,x) ) \quad \quad \quad \, (5.46b)
\end{aligned}
\]

evaluate change of $Y$ along $\sigma(s,x)$  \\
Compare $Y$ at $x$ and at $x' = \sigma_{\epsilon}(x)$ nearby \\
But components of $Y$ at 2 pts. belong to different tangent spaces $T_pM, T_{\sigma_{\epsilon}(x) } M$ \\
map $\left. Y \right|_{\sigma_{\epsilon}(x) }$ to $T_xM$ by 
\[
( \sigma_{-\epsilon })_* : T_{\sigma_{ \epsilon(x)} } M \to T_x M
\]
Lie derivative of vector field $Y$ along flow $\sigma$ of $X$

\begin{equation}
  \mathcal{L}_x Y = \lim_{\epsilon \to 0 } \frac{1}{ \epsilon } \left[  ( \sigma_{-\epsilon })_* \left. Y \right|_{\sigma_{\epsilon(x)} } - \left. Y \right|_x \right] \quad \quad \quad \, (5.47)
\end{equation}


\[
\begin{gathered}
  (\sigma_{-\epsilon})_* : T_{\sigma_{ \epsilon }(x) } M \to T_x M \\ 
  ((\sigma_{-\epsilon})_* Y ) [g] = Y [ g(\sigma_{-\epsilon} ) ]
\end{gathered}
\]

\exercisehead{5.8 }

\[
\begin{gathered}
  \begin{gathered}
    (\sigma_{\epsilon})_* : T_{ \sigma_{-\epsilon }(x) } M \to T_x M \\
\mathcal{L}_X Y =     \lim_{ \epsilon \to 0 } \frac{1}{ \epsilon } \left[ \left. Y \right|_x -  \left. (\sigma_{\epsilon })_* Y \right|_{ \sigma_{ - \epsilon}(x) } \right]
\end{gathered} \\
    \begin{gathered}
    (\sigma_{\epsilon})_* : T_{ x } M \to T_{\sigma_{\epsilon}(x)} M \\
 \mathcal{L}_X Y =    \lim_{ \epsilon \to 0 } \frac{1}{ \epsilon } \left[ \left. Y \right|_{\sigma_{\epsilon}(x)} -  \left. (\sigma_{\epsilon })_* Y \right|_{ x } \right]
\end{gathered} 
\end{gathered}
\]


Let $(U, \varphi)$ be a chart with coordinates $x$ \\
Let $ \begin{aligned} & \quad \\ 
  & X = X^{\mu} \frac{ \partial }{ \partial x^{\mu } } \\
  & Y = Y^{\mu} \frac{ \partial }{ \partial x^{\mu} } \end{aligned}$   
\[
\begin{gathered}
  \sigma_{ \epsilon}(x) = x^{\mu} +  \epsilon X^{\mu}(x) \\ 
  \left. Y \right|_{ \sigma_{\epsilon }(x) } = \left. Y^{\mu}(x^{\nu} + \epsilon X^{\nu}(x)) \right|_{x + \epsilon X} \simeq \left. \left[ Y^{\mu}(x) + \epsilon X^{\nu}(x) \partial_{\nu} Y^{\mu }(x) \right] e_{\mu} \right|_{x + \epsilon X}
\end{gathered}
\]
$\lbrace e_{\mu} = \frac{ \partial }{ \partial x^{\mu} } \rbrace$ is the coordinate basis.  

Consider Fig. 5.12.  Note the need to ``pullback'' vector $Y$ to $x$.  


$(\sigma_{-\epsilon})_*$ at $\sigma_{\epsilon}(x)$ to $x$.  

Recall that 
\[
\begin{aligned}
  & \sigma_{-\epsilon} : M \to M \\ 
  & (\sigma_{- \epsilon})_* : T_{ \sigma_{\epsilon}(x) } M \to T_x M \\ 
  & ((\sigma_{-\epsilon})_* Y)[g] = Y[g(\sigma_{-\epsilon }) ] 
\end{aligned}
\]

\[
\begin{gathered}
  \left. Y \right|_{ \sigma_{\epsilon}(x) } = \left. Y^{\mu} (x^{\nu} + \epsilon X^{\nu}(x) ) \frac{ \partial }{ \partial x^{\mu} } \right|_{x + \epsilon X} \simeq \left. \left[ Y^{\mu}(x) + \epsilon X^{\nu} \partial_{\nu} Y^{\mu}(x) \right] \frac{ \partial }{ \partial x^{\mu} } \right|_{x + \epsilon X } \\ 
    \text{ Now } \left. (\sigma_{-\epsilon})_* Y \right|_{ \sigma_{\epsilon}(x) } = W^{\alpha} \frac{ \partial }{ \partial x^{\alpha }} = Y^{\mu} \frac{ \partial }{ \partial x^{\mu}} \sigma_{-\epsilon}^{\alpha} \frac{ \partial }{ \partial x^{\alpha }} \\
    \Longrightarrow W^{\alpha} =\left.  Y^{\mu}  \right|_{ \sigma_{\epsilon}(x) } \frac{ \partial }{ \partial x^{\mu }} \sigma^{\alpha}_{-\epsilon } = \\
    = (Y^{\mu}(x) + \epsilon X^{\nu} \partial_{\nu} Y^{\mu}(x)) \frac{ \partial }{ \partial x^{\mu} } (x^{\alpha}  - \epsilon X^{\alpha} ) = [ Y^{\mu}(x) + \epsilon X^{\nu} \partial_{\nu} Y^{\mu}(x)] ( \delta_{\mu}^{ \, \, \alpha} - \epsilon \partial_{\mu} X^{\alpha} ) = \\
    = Y^{\alpha}(x) + \epsilon X^{\nu} \partial_{\nu} Y^{\alpha}(x) - \epsilon Y^{\mu} \partial_{\mu} X^{\alpha} + \mathcal{O}(\epsilon^2 ) 
\end{gathered} 
\]
So the first order term is 
\[
\epsilon (X^{\nu} \partial_{\nu} Y^{ \alpha}(x) - Y^{\nu} \partial_{\nu} X^{\alpha} )
\]

\exercisehead{5.9}

Given $\begin{aligned}
  & \quad \\ 
  & X = X^{\mu} \frac{ \partial }{ \partial x^{\mu }} \\ 
  & Y = Y^{\mu} \frac{ \partial }{ \partial x^{\mu }}\end{aligned}$ \\

Lie bracket $[X,Y]$.  $[X,Y]f = X[Y[f]] - Y[X[f]]$.  

\[
\begin{gathered}
  [X,Y] f = \left[ X^{\nu} \frac{ \partial Y^{\mu}}{ \partial x^{\nu }} - Y^{\nu} \frac{ \partial X^{\mu} }{ \partial x^{\nu }} \right] \frac{ \partial f }{ \partial x^{\mu }} \\ 
  \Longrightarrow \boxed{ [X,Y] = \left[ X^{\mu} \frac{ \partial Y^{\nu }}{ \partial x^{\mu} } - Y^{\mu} \frac{ \partial X^{\nu }}{ \partial x^{\mu} } \right] \frac{ \partial }{ \partial x^{\nu }} }
\end{gathered}
\]
This is the \textbf{local form of the Lie bracket} 

$\Longrightarrow \mathcal{L}_XY = [X,Y]$  (5.49b).  $[X,Y]$ indeed 1st.-order derivative and indeed a vector field.  

\exercisehead{5.10}  
\begin{enumerate}
\item[(a)] bilinearity

Want: $\begin{aligned}
  & [X, c_1 Y_1 + c_2 Y_2 ] = c_1[X,Y_1] + c_2 [ X,Y_2 ] \\ 
  & [c_1 X_1 + c_2 X_2, Y ] = c_1 [X_1, Y] + c_2 [ X_2, Y ] 
  \end{aligned}$

\[
\begin{aligned}
 [X, c_1 Y_1 + c_2 Y_2 ] & = X^{\mu} \frac{ \partial}{ \partial x^{\mu} } ( c_1 Y_1 + c_2 Y_2)^{\nu} - ( c_1 Y_1 + c_2 Y_2)^{\mu}  \frac{ \partial X^{\nu} }{ \partial x^{\mu} } = \\
   & = c_1 \left( X^{\mu} \frac{ \partial Y_1^{\nu} }{ \partial x^{\mu}  } - Y_1^{\mu} \frac{ \partial X^{\nu }}{ \partial x^{\mu }} \right) + c_2 \left( X^{\mu} \frac{ \partial Y_2^{\nu} }{  \partial x^{\mu} } - Y_2^{\mu} \frac{ \partial X^{\nu }}{ \partial x^{\mu} } \right) = c_1 [ X,Y_1] + c_2[ X,Y_2]  \\
   [c_1 X_1 + c_2 X_2, Y ] & = (c_1 X_1 + c_2 X_2)^{\mu} \frac{ \partial Y^{\nu}}{ \partial x^{\mu} } - Y^{\mu} \frac{ \partial }{ \partial x^{\mu} }( c_1 X_1 + c_2 X_2)^{\nu} = \\
   & = c_1 \left( X_1^{\mu} \frac{ \partial Y^{\nu}}{ \partial x^{\mu} } - Y^{\mu} \frac{ \partial X_1^{\nu}}{ \partial x^{\mu }} \right) + c_2 \left( X_2^{\mu} \frac{ \partial Y^{\nu }}{ \partial x^{\mu }} - Y^{\mu} \frac{ \partial X_2^{\nu }}{ \partial x^{\mu}} \right) = c_1 [X_1, Y] + c_2 [ X_2, Y ] 
\end{aligned}
\]
\item[(b)] \[
[Y,X] = Y^{\mu} \frac{ \partial X^{\nu}}{ \partial x^{\mu} } - X^{\mu} \frac{ \partial Y^{\nu }}{ \partial x^{\mu }} = - \left( X^{\mu} \frac{ \partial Y^{\nu }}{ \partial x^{\mu} } - Y^{\mu} \frac{ \partial X^{\nu}}{ \partial x^{\mu }} \right) = - [ X,Y]
\]
\item[(c)] Want: 
\[
[[X,Y],Z] + [[Z,X],Y] + [[Y,Z], X] = 0 
\]
Now
\[
\begin{gathered}
  [[X,Y],Z] \\
  [X,Y]^{\mu} \frac{ \partial Z^{\nu}}{ \partial x^{\mu }} - Z^{\mu} \frac{ \partial}{ \partial x^{\mu}} [X,Y]^{\nu} = (X^a \partial_a Y^{\mu} - Y^a \partial_a X^{\mu} ) \partial_{\mu}Z^{\nu} - Z^{\mu} ( \partial_{\mu}X^a \partial_a  Y^{\nu} + X^a \partial^2_{\mu a} Y^{\nu} - \partial_{\mu} Y^a\partial_a X^{\nu} - Y^a \partial^2_{\mu a} X^{\nu} ) = \\
  = X^a \partial_a Y^{\mu} \partial_{\mu} Z^{\nu} - Y^a \partial_a X^{\mu} \partial_{\mu} Z^{\nu} - Z^{\mu} \partial_{\mu} X^a \partial_a Y^{\nu} -Z^{\mu}X^a \partial^2_{\mu a} Y^{\nu} + Z^{\mu} \partial_{\mu} Y^a \partial_a X^{\nu} + Z^{\mu} Y^a \partial^2_{\mu a} X^{\nu} 
\end{gathered}
\]
Likewise,
\[
\begin{gathered}
%  \begin{gathered}
    [[Z,X],Y]^{\nu}    = Z^a \partial_a X^{\mu} \partial_{\mu} Y^{\nu} - X^a \partial_a Z^{\mu} \partial_{\mu} Y^{\nu} - Y^{\mu} \partial_{\mu} Z^a \partial_a X^{\nu} - Y^{\mu} Z^a \partial^2_{\mu a} X^{\nu} + Y^{\mu} \partial_{\mu} X^a \partial_a Z^{\nu} + Y^{\mu} X^a \partial^2_{\mu a} Z^{\nu} \\
%\end{gathered}
    [[Y,Z],X]^{\nu}    = Y^a \partial_a Z^{\mu} \partial_{\mu} X^{\nu} - Z^a \partial_a Y^{\mu} \partial_{\mu} X^{\nu} - X^{\mu} \partial_{\mu} Y^a \partial_a Z^{\nu} - X^{\mu} Y^a \partial^2_{\mu a} Z^{\nu} + X^{\mu} \partial_{\mu} Z^a \partial_a Y^{\nu} + X^{\mu} Z^a \partial^2_{\mu a} Y^{\nu} 
\end{gathered}
\]
All the 18 terms cancel.  

\end{enumerate}






\subsection{ Differential forms }

symmetry operation on tensor \quad 

$\omega \in \tau^0_{r,p}(M)$

\begin{equation}
P \omega(v_1 \dots v_r) \equiv \omega(v_{P(1)} \dots v_{P(r)} ) \quad \quad \quad (5.59)
\end{equation}

$v_i \in T_p M$, $P \in S_r$, symmetry group of order $r$

\[
\begin{gathered}
  \omega(e_{\mu_1} \dots e_{\mu_r} ) = \omega_{\mu_1 \dots \mu_r} \\
  P\omega(e_{\mu_1} \dots e_{\mu_r} ) = \omega_{ \mu_{P(1)} \dots \mu_{ P(r)}  }
\end{gathered}
\]

symmetrizer $\mathcal{S}$, $\omega \in \tau_{r,p}^0(M)$

\[
S_{\omega} = \frac{1}{r!} \sum_{P \in S_r} P\omega \quad \quad \quad (5.60)
\]

anti-symmetrizer $\mathcal{A}$
\[
\mathcal{A} \omega = \frac{1}{r!} \sum_{ P \in S_r} \text{sgn}{(P)} P\omega
\]




\subsubsection{ Definitions}

\begin{definition}[5.4] diff. form or $r$-form, totally antisymmetric tensor of type $(0,r)$ \\
define wedge product $\wedge$ of $r$ \, 1-forms by the totally antisymm. tensor product
\begin{equation}
dx^{\mu_1} \wedge dx^{\mu_2} \wedge \dots \wedge dx^{\mu_r} = \sum_{P\in S_r} \text{sgn}{(P)} dx^{\mu} \quad \quad \quad (5.62) 
\end{equation}
\end{definition}




% \subsubsection{ Definitions }

%\begin{definition}[5.4] differential form of order $r$ or $r$-form is a totally antisymmetric tensor of type $(0,r)$
%\end{definition}

%define 



%\begin{equation}
%  dx^{\mu_1} \wedge \dots \wedge dx^{\mu_r} = \sum_{ p \in S_r} \text{sgn}{(p)} dx^{\mu_{p(1)}} \wedge \dots \wedge dx^{\mu_{p(r)}}  \quad \quad \quad (5.62)
%\end{equation}


e.g.

\[
\begin{gathered}
  dx^{\mu} \wedge dx^{\nu} = dx^{\mu} \otimes dx^{\nu} - dx^{\nu} \otimes dx^{\mu} \\ 
  dx^{\lambda} \wedge dx^{\mu} \wedge dx^{\nu} = dx^{\lambda} dx^{\mu} dx^{\nu} + dx^{\nu} dx^{\lambda} dx^{\mu} + dx^{\mu} dx^{\nu} dx^{\lambda} - dx^{\lambda} dx^{\nu} dx^{\mu} - dx^{\nu} dx^{\mu} dx^{\lambda} - dx^{\mu} dx^{\lambda} dx^{\nu}
\end{gathered}
\]


\begin{enumerate}
  \item[(i)]
  \item[(ii)] \[
    \begin{gathered}
      dx^{\mu_1} \wedge \dots \wedge dx^{\mu_r} = \text{sgn}{ (P)} dx^{\mu_{P(1)} } \wedge \dots \wedge dx^{\mu_{ P(r)} } \\ 
      \sum_{ Q \in S_r} \text{sgn}{ (Q)} dx^{\mu_{Q(1) } } dx^{\mu_{Q(2) } } \ldots dx^{\mu_{Q(r)}}  = \sum_{Q \in S_r} \text{sgn}{ (Q) } ( \text{sgn}(P))^2 dx^{\mu_{Q(P(1)) }} \dots dx^{\mu_{Q(P(r)) } } = \\
       = ( \text{sgn}{ P } ) \sum_{Q \in  S_r } \text{sgn}{(Q)} \text{sgn}{ P } dx^{\mu_{Q(P(1)) }} \ldots dx^{ \mu_{Q(P(r)) } }
\end{gathered}
\]
\end{enumerate}


vector space of $r$-forms at $p\in M$ by $\Omega_p^r(M)$ \\
set of $r$-forms (5.62) forms basis of $\Omega_p^r(M)$

\[
\omega \in \Omega_p^r(M)
\]


\begin{equation}
  \omega = \frac{1}{r!} \omega_{\mu_1 \dots \mu_r} dx^{\mu_1} \wedge \dots \wedge dx^{\mu_r} \quad \quad \quad (5.63)
\end{equation}




$\omega_{\mu_1 \dots \mu_r}$ totally antisymmetric, reflecting antisymmetry of basis

$\binom{m}{r}$ choices of $(\mu_1 \dots \mu_r)$ out of $(1 \dots n)$ in (5.62)

\[
\text{dim}{ \Omega_p^r(M) } = \binom{m}{r}
\]
since $\binom{m}{r} = \binom{m}{m-r}$, $\Omega_p^r(M) \simeq \Omega_p^{m-r}(M)$

Let $\begin{aligned} & \quad \\ & \omega \in \Omega_p^q(M)  \\ & \xi \in \Omega_p^r(M) \end{aligned}$

action of $(q+r)$-form $\omega \wedge \xi $ on $q+r$ vectors 

\begin{equation}
  (\omega \wedge \xi )(v_1 \dots v_{q+r} ) = \frac{1}{q! r!} \sum_{ p \in S_{q+r}} \text{sgn}{(P)} \omega( v_{p(1)} \dots v_{p(q)} ) \xi(v_{p(q+1)} \dots v_{ p(q+r) } ) \quad \quad \quad (5.65)
\end{equation}

with this product, define 
\begin{equation}
\Omega_p^*(M) \equiv \bigoplus_{k=0}^m \Omega_p^k(M) \quad \quad \quad (5.66)
\end{equation}

\exercisehead{5.13} \[
\begin{aligned}
  & x = r\cos{\theta} \\ 
  & y = r\sin{\theta}
\end{aligned} \quad \quad \quad \begin{aligned} & r = \sqrt{ x^2 + y^2 } \\ & \theta = \arctan{ \left( \frac{y}{x} \right) } \end{aligned}
\]

\[
\begin{gathered}
  \begin{aligned}
    & \partial_r x = c_{\theta} \\ 
    & \partial_r y = s_{\theta} 
  \end{aligned} \quad \quad \begin{aligned}
    & \partial_{\theta} x = -rs_{\theta} \\ 
    & \partial_{\theta} y = rc_{\theta}
\end{aligned} \quad \quad \begin{aligned}
    \partial_x r = \frac{x}{r} \quad \quad \partial_y r = \frac{y}{r} \\
    \begin{aligned}
      & \partial_x \theta = \frac{ \frac{-y}{x^2} }{ 1 + \frac{y^2}{x^2} } = \frac{-y}{ x^2 + y^2 } \\
      & \partial_y \theta = \frac{x}{ x^2 + y^2 }
\end{aligned}
\end{aligned} \\
\begin{aligned}
  dx \wedge dy = (c_{\theta} dr + - rs_{\theta} d\theta ) \wedge (s_{\theta}dr + rc_{\theta} d\theta ) = rc_{\theta}^2 dr \wedge d\theta + rs_{\theta}^2 dr \wedge d\theta = rdr \wedge d\theta
\end{aligned}
\end{gathered}
\]



\exercisehead{5.14} 
\[
\xi \wedge \xi = (v_1 \wedge \dots \wedge v_q) \wedge (v_1 \wedge \dots \wedge v_q) = (-1)^q v_1 \wedge ( v_1 \wedge \dots \wedge v_q) \wedge ( v_2 \wedge \dots \wedge v_q) = (-1)^{q^2} \xi \wedge \xi = -\xi \wedge \xi \text{ if $q$ odd }
\]

\subsubsection{ Exterior derivatives }

\begin{definition}[5.5] exterior derivatives 
\[
\begin{aligned}
  & d_r : \Omega^r(M) \to \Omega^{r+1}(M) \\ 
  & \omega = \frac{1}{r!} \omega_{ \mu_1 \dots \mu_r} dx^{\mu_1} \wedge \dots \wedge dx^{\mu_r} \\  
  & d_r \omega = \frac{1}{r!} \left( \frac{ \partial }{ \partial x^{\nu }} \omega_{\mu_1 \dots \mu_r} \right) dx^{\nu} \wedge dx^{\mu_1} \wedge \dots \wedge dx^{\mu_r} \quad \quad \quad (5.68)
\end{aligned}
\]
\end{definition}


Example 5.10.  in 3-dim. 
\[
\begin{aligned}
  & \omega_0 = f(x,y,z) \\ 
  & \omega_1 = \omega_x(x,y,z) dx + \omega_y(x,y,z)dy + \omega_z(x,y,z) dz \\ 
  & \omega_2 = \omega_{xy}(x,y,z) dx \wedge dy + \omega_{yz}(x,y,z) dy \wedge dz + \omega_{zx}(x,y,z) dz \wedge dx \\ 
  & \omega_3 = \omega_{xyz}(x,y,z) dx \wedge dy \wedge dz
\end{aligned}
\]

Recall

\[
\omega = \frac{1}{r!} \omega_{ \mu_1 \dots \mu_r} dx^{\mu_1} \wedge \dots \wedge dx^{\mu_r}
\]

e.g.

\[
\omega_{12} dx^1 \wedge dx^2 + \omega_{13} dx^1 \wedge dx^3 + \omega_{21} dx^2 \wedge dx^1 + \dots = (\omega_{12} - \omega_{21} ) dx^1 \wedge dx^2 + \dots 
\]
$\omega_{\mu_1 \dots \mu_r}$ itself must be antisymmetrized. 


\[
d\omega = \frac{1}{r!} \left( \frac{ \partial \omega_{ \mu_1 \dots \mu_r} }{ \partial x^{\nu} } \right) dx^{\nu} \wedge dx^{\mu_1} \wedge \dots \wedge dx^{\mu_r}
\]
\begin{enumerate}
\item[(i)]
\[
d\omega_0 = \frac{ \partial f}{ \partial x} dx + \frac{ \partial f}{ \partial y} dy + \frac{ \partial f}{ \partial z} dz
\]
\item[(ii)]
\[
\begin{aligned}
  d\omega_1 & = \frac{1}{1!} \left( \frac{ \partial \omega_x}{ \partial y} dy \wedge dx  + \frac{ \partial \omega_x}{ \partial z} dz \wedge dx + \frac{ \partial \omega_y}{ \partial x} dx \wedge dy + \frac{ \partial \omega_y}{ \partial z} dz \wedge dy + \frac{ \partial \omega_z}{ \partial x} dx \wedge dz + \frac{ \partial \omega_z}{ \partial y} dy \wedge dz \right)  =  \\
  & = \left( \frac{ \partial \omega_y }{ \partial x} - \frac{ \partial \omega_x}{ \partial y} \right) dx \wedge dy + \left( \frac{ \partial \omega_z}{ \partial y} - \frac{ \partial \omega_y}{ \partial z} \right) dy \wedge dz + \left( \frac{ \partial \omega_x}{ \partial z} - \frac{ \partial \omega_z}{ \partial x} \right) dz \wedge dx 
\end{aligned}
\]
\item[(iii)]
\[
\begin{aligned}
  d\omega_2 & = \frac{1}{2!} \left( \frac{ \partial \omega_{xy} }{ \partial z} dz \wedge dx \wedge dy + \frac{ \partial \omega_{yx} }{ \partial z} dz \wedge dy \wedge dx + \frac{ \partial \omega_{yz} }{ \partial x} dx \wedge dy \wedge dz + \frac{ \partial \omega_{zy} }{ \partial x} dx \wedge dz \wedge dy + \dots \right) = \\ 
& = \left( \frac{ \partial \omega_{yz} }{ \partial x} + \frac{ \partial \omega_{zx} }{ \partial y} + \frac{ \partial \omega_{xy} }{ \partial z} \right) dx \wedge dy \wedge dz
\end{aligned}
\]
\end{enumerate}

\exercisehead{5.15} 20130929

$\begin{aligned} & \quad \\ 
  & \xi \in \Omega^q(M) \\
  & \omega \in \Omega^r(M) \end{aligned}$ \quad \quad $\xi \wedge \omega = \xi_{i_1 \dots i_q} \omega_{j_1 \dots j_r} dx^{i_1} \wedge \dots \wedge dx^{i_q} \wedge dx^{j_1} \wedge \dots \wedge dx^{j_r}$


\[
\begin{gathered}
  d(\xi \wedge \omega) = \frac{1}{ (q+r)!} \frac{ \partial ( \xi_{i_1 \dots i_q} \omega_{j_1 \dots j_r} )}{ \partial x^{\nu} }dx^{\nu} \wedge dx^{i_1} \wedge \dots \wedge dx^{i_q} \wedge dx^{j_1} \wedge \dots \wedge dx^{j_r} = \\ 
    = \frac{1}{(q+r)!} ( \partial_{\nu} \xi_{i_1 \dots i_q} \omega_{j_1 \dots j_r} + \xi_{i_1 \dots i_q} \partial_{\nu} \omega_{j_1 \dots j_r} ) dx^{\nu} \wedge dx^{i_1} \wedge \dots = \\
    = \frac{1}{ (q+r)!} \lbrace \partial_{\nu} \xi_{i_1 \dots i_q} \omega_{j_1 \dots j_r} dx^{\nu} \wedge dx^{i_1} \wedge \dots \wedge dx^{i_q} \wedge dx^{j_1} \wedge \dots \wedge dx^{j_r} + \\
    + \xi_{i_1 \dots i_q} \partial_{\nu} \omega_{j_1 \dots j_r} dx^{i_1} \wedge \dots \wedge dx^{i_q} \wedge dx^{\nu} \wedge dx^{j_1} \wedge \dots \wedge dx^{j_r} (-1)^q \rbrace 
\end{gathered}
\]

\[
\begin{gathered}
  d\xi \wedge \omega + (-1)^q \xi \wedge d\omega = \\
  = \frac{1}{q!} \frac{ \partial \xi_{i_1 \dots i_q} }{ \partial x^{\nu} } dx^{\nu} \wedge dx^{i_1} \wedge \dots \wedge dx^{i_q} \wedge \omega_{j_1 \dots j_r} dx^{j_1} \wedge \dots \wedge dx^{j_r} 
\end{gathered}
\]

decomposable forms 
\[
\begin{aligned}
  & \xi = f dx_{i_1} \wedge \dots \wedge dx_{i_q} = fdx_I \\ 
  & \omega = gdx_{j_1} \wedge \dots \wedge dx_{j_q} = gdx_I
\end{aligned}
\]
\[
\begin{gathered}
  d(\xi \wedge \omega) = d(fdx_I \wedge g dx_J) = d(fg) \wedge dx_2 \wedge dx_J = (fdg = gdf ) \wedge dx_2 \wedge dx_J = \\
  = df \wedge dx_I \wedge g dx_J + (-1)^q fdx_I \wedge dg \wedge dx_J = d\xi \wedge \omega + \xi \wedge d\omega
\end{gathered}
\]

$\begin{aligned}
  & X = X^{\mu} \partial_{x^{\mu }} \in \mathcal{X}{(M)} \\ 
  & Y = Y^{\nu} \partial_{x^{\nu }}\end{aligned}$ \quad \quad \quad $\omega = \omega_{\mu} dx^{\mu} \in \Omega^1(M)$

\[
\begin{gathered}
  X[\omega(Y)] - Y[\omega(X)] - \omega([X,Y]) = \frac{ \partial \omega_{\mu} }{ \partial x^{\nu }} ( X^{\nu} Y^{\mu} - X^{\mu} Y^{\nu} ) \\ 
  [X,Y] = [X^{\mu} \partial_{x^{\mu}} Y^{\nu} - Y^{\mu} \partial_{x^{\mu}} X^{\nu} ] \partial_{x^{\nu} } \\ 
 \omega([X,Y]) = \omega_{\nu} ( X^{\mu} \partial_{\mu} Y^{\nu} - Y^{\mu} \partial_{\mu} X^{\nu} ) \\
 \begin{aligned}
   & X[\omega(Y)] = X^{\mu} \partial_{\mu} ( \omega_{\nu} Y^{\nu} ) = X^{\mu} \partial_{\mu} \omega_{\nu} Y^{\nu} + X^{\mu} \omega_{\nu} \partial_{\mu} Y^{\nu} \\ 
   & Y[\omega(X)] = Y^{\mu} \partial_{\mu} (\omega_{\nu} X^{\nu} ) = Y^{\mu} \partial_{\mu} \omega_{\nu} X^{\nu} + Y^{\mu} \omega_{\nu} \partial_{\mu} X^{\nu}
\end{aligned} \\
\Longrightarrow X[\omega(Y)] - Y[\omega(X) ] - \omega([X,Y]) = X^{\mu} \partial_{\mu} \omega_{\nu} Y^{\nu} - Y^{\mu} \partial_{\mu} \omega_{\nu} X^{\nu} = \partial_{\mu} \omega_{\nu} (X^{\mu} Y^{\nu}  - Y^{\mu} X^{\nu }) 
\end{gathered}
\]

Suppose 
\[
d\omega(X_1 \dots X_{p+1}) = \sum_{i=1}^r (-1)^{i+1} X_i \omega(X_1 \dots \widehat{X}_i \dots X_{i+1}) + \sum_{ i < j } (-1)^{i+1} \omega([X_i, X_j], X_1 \dots \widehat{X}_i \dots \widehat{X}_j \dots X_{i+1}) \quad \quad \quad (5.71)
\]

for $r$-form $\omega \in \Omega^r(M)$

\[
\begin{gathered}
  \omega = \frac{1}{r!} \omega_{\mu_1 \dots \mu_r} dx^{\mu_1} \wedge \dots \wedge dx^{\mu_r} \\ 
  d\omega = \frac{1}{r!} \left( \frac{ \partial}{ \partial x^{\nu} }\omega_{\mu_1 \dots \mu_r}  \right) dx^{\nu} \wedge dx^{\mu_1} \wedge \dots \wedge dx^{\mu_r}
\end{gathered}
\]

\subsubsection{ Interior product and Lie derivative of forms }

interior product $i_X : \Omega^r(M) \to \Omega^{r-1}(M)$ where $X \in \chi(M)$ \\

Let $\omega \in \Omega^r(M)$, define

\begin{equation}
  i_X \omega(X_1 \dots X_{r-1}) \equiv \omega(X,X_1 \dots X_{r-1} ) \quad \quad \quad \, (5.78)
\end{equation}

with
\[
\begin{aligned}
  & X = X^{\mu} \frac{ \partial}{ \partial x^{ \mu} } \\ 
  &  \omega = \frac{1}{r!} \omega_{\mu_1 \dots \mu_r} dx^{\mu_1} \wedge \dots \wedge dx^{\mu_r}
\end{aligned}
\]

\begin{equation}
  i_X \omega = \frac{1}{ (r-1)!} X^{\nu} \omega_{\nu \mu_2 \dots \mu_r} dx^{\mu_2} \wedge \dots \wedge dx^{\mu_r} = \frac{1}{r!} \sum_{s=1}^r X^{\mu_s} \omega_{ \mu_1 \dots \mu_s \dots \mu_r }(-1)^{s-1} dx^{\mu_1} \wedge \dots \wedge \widehat{ dx^{\mu_s} } \wedge \dots \wedge dx^{\mu_r} \quad \quad \quad \, (5.79)
\end{equation}

cf. wikipedia

$i_X: \Omega^p(M) \to \Omega^{p-1}(M)$ \quad \quad \, $i_X$ a map that sends a $p$-form $\omega$ to the $(p-1)$ form $i_X \omega$

\[
(i_X \omega)(X_1 \dots X_{p-1}) = \omega(X,X_1 \dots X_{p-1} )
\]

$\alpha$ 1-form $i_X\alpha = \alpha(X)$


for $\begin{aligned} & \quad \\ 
  & \beta \in \Omega^p(M) \\ 
  & \gamma \in \Omega^q(M) \end{aligned}$

\[
i_X(\beta \wedge \gamma) = (i_X \beta) \wedge \gamma + (-1)^p \beta \wedge (i_X \gamma )
\]

\[
i_X \omega = \frac{1}{ (p-1)!} X^i \omega_{ii_2 \dots i_p} dx^{i_2} \wedge \dots \wedge dx^{i_p} = \frac{1}{ p!} \sum_{s=1}^p X^{i_s} \omega_{ i_1 \dots i_s \dots i_p }(-1)^{s-1} dx^{i_1} \wedge \dots \wedge \widehat{dx^{i_s}} \wedge \dots \wedge dx^{i_r}
\]

\[
\begin{aligned}
  & i_{\partial_x}(dx \wedge dy) = \frac{1}{ (2-1)!} dy = dy \\ 
  & i_{\partial_x}(dy \wedge dz) = 0 \\  
  & i_{\partial_x}( dz \wedge dx) = \delta^i_{ \, \, 1 } \epsilon_{31} dx^3 = -dz
\end{aligned}
\]


%%%%%%%% interior product on 1-form

Let $\omega$ 1-form.  $i_X \omega = \omega(X)$

\[
\begin{gathered}
  (di_X + i_X d)\omega = d(X^{\mu} \omega_{\mu} ) + i_X [ \frac{1}{2} (\partial_{\mu} \omega_{\nu} - \partial_{\nu} \omega_{\mu} ) dx^{\mu} \wedge dx^{\nu} ] = \\
  = (\omega_{\mu} \partial_{\nu} X^{\mu} + X^{\mu} \partial_{\nu} \omega_{\mu} ) dx^{\nu} + \frac{1}{2} ( X^{\mu} \partial_{\mu} \omega_{\nu} dx^{\nu} - X^{\nu} \partial_{\nu} \omega_{\mu} dx^{\mu} )
\end{gathered}
\]
using, recall,
\[
i_X \omega = \frac{1}{ (p-1)!} X^i \omega_{i i_2 \dots i_p } dx^{i_2} \wedge \dots \wedge dx^{i_p}
\]

Recall (5.55): $\mathcal{L}_X\omega = (X^{\nu} \partial_{\nu} \omega_{\mu} + \partial_{\mu} X^{\nu} \omega_{\nu} ) dx^{\mu}$

\begin{equation}
  \mathcal{L}_X \omega = (di_X + i_X d) \omega \quad \quad \quad \, (5.80)
\end{equation}

for $r$-form, $\omega = \frac{1}{r!} \omega_{\mu_1 \dots \mu_r} dx^{\mu_1} \wedge \dots \wedge dx^{\mu_r}$

\begin{equation}
  \mathcal{L}_X\omega = \lim_{ \epsilon \to 0} \frac{1}{\epsilon} ( (\sigma_{\epsilon})^* \left. \omega \right|_{ \sigma_{\epsilon}(X) } - \left. \omega  \right|_X ) = X^{\nu} \frac{1}{r!} \partial_{\nu} \omega_{ \mu_1 \dots \mu_r} dx^{\mu_1} \wedge \dots \wedge dx^{\mu_r} + \sum_{s=1}^r \partial_{\mu_s} X^{\nu} \frac{1}{r!} \omega_{ \mu_1 \dots (s \to \nu ) \dots \mu_r} dx^{\mu_1} \wedge \dots \wedge dx^{\mu_r} \quad \quad \, (5.81)
\end{equation}








%%%%%%% Example 5.12 % Hamiltonian phase space

Ex.5.12.  $(q^{\mu}, p_{\mu})$ tangent bundle!!!   \\
symplectic 2 form 
\begin{equation}
  \omega = dp_{\mu} \wedge dq^{\mu} \quad \quad \quad \, (5.88)
\end{equation}


1-form $\theta = q^{\mu} dp_{\mu}$ \quad \quad \, or (???) $\theta = p_{\mu} dq^{\mu}$

\[
\omega = d\theta
\]
Given $f(q,p)$ in phase space

define Hamiltonian vector field 
\[
X_f = \frac{ \partial f}{ \partial p_{\mu}} \frac{ \partial }{ \partial q^{\mu} } - \frac{ \partial f}{ \partial q^{\mu} } \frac{ \partial }{ \partial p_{\mu} } \quad \quad \, (5.91)
\]

\[
i_{X_f} \omega = \frac{ - \partial f}{ \partial p_{\mu}} dp_{\mu} - \frac{ \partial f}{ \partial q^{\mu} } dq^{\mu} = -df
\]
\[
i_X \omega = \frac{1}{ (p-1)!} X^i \omega_{ ii_2 \dots i_p} dx^{i_2} \wedge \dots \wedge dx^{i_p}
\]

Consider vector field generated by Hamiltonian

\begin{equation}
  X_H = \frac{ \partial H }{ \partial p_{\mu}} \frac{ \partial }{ \partial q^{\mu} } - \frac{ \partial H}{ \partial q^{\mu} } \frac{ \partial }{ \partial p_{\mu} } \quad \quad \, (5.92)
\end{equation}

Hamilton's eqns. of motion.
\[
\begin{aligned}
  & \dot{q}^{\mu} = \frac{ \partial H}{ \partial p_{\mu} } \\
  & \dot{p}_{\mu} = \frac{ - \partial H}{ \partial q^{\mu }} 
\end{aligned} \quad \quad \, (5.93) \quad \quad \text{(Hamilton's eqn. of motion)}
\]

\begin{equation}
X_H = \dot{q}^{\mu} \frac{ \partial }{ \partial q^{\mu} } + \dot{p}_{\mu} \frac{ \partial }{ \partial p_{\mu} } = \frac{d}{dt} \quad \quad \quad (5.94)
\end{equation}

symplectic 2-form $\omega$ left-invariant along flow generated by $X_H$

\[
\mathcal{L}_{X_H}\omega = d(i_{X_H} \omega ) + i_{X_H}(d\omega) = d(i_{X_H} \omega ) = - d^2 H = 0 
\]

used $(di_X + i_X d) \omega = \mathcal{L}_X \omega \quad \quad (5.82)$ 

Conversely, if $X$ satisfies $\mathcal{L}_X \omega=0$, $\exists \, $ Hamiltonian $H$ s.t. Hamilton's eqn. of motion is satisfied along the flow generated by $X$

from $\mathcal{L}_X \omega = d(i_X \omega ) = 0$ and hence by Poincar\'{e}'s lemma, $\exists \,  H(q,p)$ s.t. 
\[
i_X \omega = -dH
\]


\subsection{ Integration of differential forms }

\subsubsection{ Orientation }

integration of differential form over manifold $M$ defined only when $M$ is ``orientable''

Let $M$ connected $m$-dim. differential manifold \\
$\forall \, p \in M$, $T_pM$ spanned by basis $\lbrace \frac{ \partial }{ \partial x^{\mu} } \rbrace$, $x^{\mu}$ local coordinate on chart $U_i \ni p$ \\
Let $U_j$ another chart s.t. $U_i \cap U_j \neq \emptyset$ with local cordinates $y^{\alpha}$. \\
If $p \in U_i \cap U_j$, $T_pM$ spanned by either $\lbrace e_{\mu} \rbrace = \lbrace \frac{ \partial }{ \partial x^{\mu} }$, or $\lbrace \widetilde{e}_{\alpha} \rbrace = \lbrace \frac{ \partial }{ \partial y^{\alpha} } \rbrace$
\begin{equation}
  \frac{ \partial }{ \partial y^{\alpha} } = \frac{ \partial x^{\mu} }{ \partial y^{\alpha}} \frac{ \partial }{ \partial x^{\mu} } \quad \quad \, (5.97)
\end{equation}

If $J = \text{det}{ \left( \frac{ \partial x^{\mu} }{ \partial y^{\alpha}} \right)} >0$ on $U_i \cap U_j$, $\lbrace e_{\mu} \rbrace$, $\lbrace \widetilde{e}_{\alpha} \rbrace$ define same orientation on $U_i \cap U_j$

If $J <0$, opposite orientation.  



\begin{definition}[5.6]
$M$ connected manifold covered by $\lbrace U_i \rbrace$ \\
manifold $M$ orientable if $\forall \, $ overlapping charts $U_i, \, U_j$, $\exists \, $ local coordinates $\begin{aligned} & \quad \\
  & \lbrace x^{\mu} \rbrace \text{ for } U_i \\ 
  & \lbrace y^{\alpha} \rbrace \text{ for } U_j \end{aligned}$ s.t. $J = \text{det}{ \left( \frac{ \partial x^{\mu} }{ \partial y^{\alpha}} \right) } > 0$
\end{definition}

If $M$ nonorientable, $J$ can't be positive in all interactions of charts.  

If $m$-dim. $M$ orientable, $\exists \, m$-form $\omega$ s.t. $\omega\neq 0$ \\
This $m$-form $\omega$ is volume element \\
2 vol. elements $\omega, \omega'$ equivalent if $\exists \, $ strictly positive $h \in \mathcal{F}(M)$ s.t. $\omega = h\omega'$ \\
take $m$-form
\begin{equation}
  \omega = h(p) dx^1 \wedge \dots \wedge dx^m \quad \quad \quad \, (5.98)
\end{equation}
with positive-definite $h(p)$ on chart $(U,\varphi)$, $x = \varphi(p)$

If $M$ orientable, extend $\omega$ throughout $M$ s.t. component $h$ positive definite on chart $U_i$ \\
If $M$ orientable, $\omega$ vol. element. 

Let $p\in U_i \cap U_j \neq \emptyset$, $\begin{aligned} & \quad \\
  & x^{\mu} \text{ coordinates of $U_i$ } \\ 
& y^{\alpha} \text{ coordinates of $U_j$ } \end{aligned}$

\[
\begin{gathered}
  \omega = h(p) \frac{ \partial x^1}{ \partial y^{\mu_1} } dy^{\mu_1} \wedge \dots \wedge \frac{ \partial x^m}{ \partial y^{\mu_m} }dy^{\mu_m} = h(p) \text{det}{ \left( \frac{ \partial x^{\mu} }{ \partial y^{\nu} } \right) } dy^1 \wedge \dots \wedge dy^m \\
 \text{det}{ \left( \frac{ \partial x^i }{ \partial y^j } \right) } = \sum_{ \sigma \in S_n} \text{sgn}{ (\sigma)} \frac{ \partial x^1}{ \partial y^{\sigma_1} } \dots \frac{ \partial x^m}{ \partial y^{\sigma_m}} = \sum^m_{ i_1\dots i_m = 1} \epsilon_{i_1 \dots i_m} \frac{ \partial x^1}{ \partial y^{i_1} } \dots \frac{ \partial x^m}{ \partial y^{i_m} } = \epsilon_{i_1 \dots i_m } \frac{ \partial x^1}{ \partial y^{i_1} } \dots \frac{ \partial x^m}{ \partial y^{i_m}} \\
\frac{ \partial x^1}{ \partial y^{i_1}} \dots \frac{ \partial x^m}{ \partial y^{i_m} } dy^{i_1} \wedge \dots \wedge dy^{i_m} = \frac{ \partial x^1}{ \partial y^{i_1}} \dots \frac{ \partial x^m}{ \partial y^{i_m}} \epsilon^{i_1 \dots i_m} dy^1 \wedge \dots \wedge dy^{i_m} = \text{det}{ \left( \frac{ \partial x^i}{ \partial y^j} \right) } dy^1 \wedge \dots \wedge dy^{i_m}
\end{gathered}
\]




\subsubsection{ Integration of forms }

integration of function $f:M \to \mathbb{R}$ over oriented $M$ \\
take vol. element $\omega$ \\
in coordinate neighborhood $U_i$, \, $x=\varphi(p)$, $p\in U_i$

\begin{equation}
  \int_{U_i} f\omega \equiv \int_{\varphi{ (U_i)} } f(\varphi_i^{-1}(x)) h(\varphi_i^{-1}(x)) dx^1 \dots dx^m \quad \quad \, (5.100)
\end{equation}

\begin{definition}[5.7]
  open covering $\lbrace U_i \rbrace$ of $M$ s.t. $\forall \, p \in M$, $p$ covered by a finite number of $U_i$.  $M$ paracompact.  
\end{definition}

If diff. $\epsilon_i(p)$ s.t. \begin{enumerate}
\item[(i)] $0\leq \epsilon_i(p) \leq 1$ 
\item[(ii)] $\epsilon_i(p) =0 $ if $p \notin U_i$ 
\item[(iii)] $\epsilon_1(p) + \epsilon_2(p) + \dots = 1$ \quad \, $\forall p \in M$
\end{enumerate}

$\lbrace \epsilon(p) \rbrace$ partition of unity subordinate to covering $\lbrace U_i \rbrace$

from (iii), 
\begin{equation}
  f(p) = \sum_i f(p) \epsilon_i(p) = \sum_i f_i(p) \quad \quad \quad \, (5.101)
\end{equation}

$f_i(p) \equiv f(p) \epsilon_i(p)$, \, $f_i(p) =0$ \, $\forall \, p \notin U_i$

 Hence, $\forall \, p \in M$, paracompactness ensures in (5.101),  $f_i(p) < \infty  $, finite, in sum $\sum_i$

define
\begin{equation}
\int_M f \omega \equiv \sum_i \int_{U_i} f_i \omega \quad \quad \quad \, (5.102)
\end{equation}

Although a different atlas $\lbrace (V_i, \psi_i)\rbrace$ gives different coordinates and different partition of unity, integral defined by (5.102) same.  

Ex. 5.13. $S^1$.  $\begin{aligned} & \quad \\ & U_1 = S^1 - \lbrace (1,0) \rbrace \\ & U_2 = S^1 - \lbrace (-1,0) \rbrace \end{aligned}$ \quad \quad \, $\begin{aligned} & \quad \\ & \epsilon_1(\theta) = \sin^2{ \left( \frac{ \theta}{2} \right) } \\ \epsilon_2(\theta) = \cos^2{ \left( \frac{\theta}{2} \right) } \end{aligned}$ \quad \quad \, $\epsilon_1 + \epsilon_2 = 1 $ on $S^1$

$f= \cos^2{\theta}$

\[
\begin{aligned}
  & \int_0^{2\pi} d\theta \cos^2{\theta} = \pi \\ 
  & \int_S^1 d\theta \cos^2{\theta} = \int_0^{2\pi } d\theta \sin^2{\frac{ \theta}{2} } \cos^2{\theta} + \int_{-\pi}^{\pi} d\theta \cos^2{  \frac{ \theta}{2} } \cos^2{\theta} = \frac{ \pi}{2} + \frac{\pi}{2} = \pi
\end{aligned}
\]

\subsection{ Lie groups and Lie algebras }

\subsubsection{ Lie groups } 

Take $x,y,z \in \mathbb{R} -0$ s.t. $xy=z$ \quad \quad $xy=z$ \quad \, $\frac{ \partial z}{ \partial x } = y \neq 0$

\exercisehead{5.19}\begin{enumerate}
\item[(a)] $\mathbb{R}^+ = \lbrace x \in \mathbb{R} | x > 0 \rbrace$ \quad \quad \, $\frac{ \partial }{ \partial x} x^{-1} = -x^{-2} \neq 0 $ \, diff.

\item[(b)] 
\[
\begin{gathered}
\partial_x z  = \partial_x ( x+y)  = 1   \\
\partial_x( x^{-1} ) = \partial_x (-x) = -1 
\end{gathered}
\] \quad diff. 
\item[(c)] \[
\begin{gathered}
  (a,b) + (x,y) = (a+x, b+y) \quad \quad \, Dg =Dg(x) = \left[ \begin{matrix} a & \\ & b \end{matrix} \right] \\
  (x,y)^{-1} = (-x, -y) \quad \quad \, D(x,y)^{-1} = \left[ \begin{matrix} -1 & \\ & -1 \end{matrix} \right]
\end{gathered}
\]
\end{enumerate}

Lorentz group 
\[
O(1,3) = \lbrace M \in GL(4,\mathbb{R}) | M \eta M^T = \eta \rbrace  \quad \quad \eta = \text{diag}{ ( -1, 1, 1, 1 ) }
\]
\exercisehead{5.20} 20130801
\[
\text{det}{ M \eta M^T } = (\text{det}{M})^2 \text{det}{\eta} = \text{det}{ \eta} \quad \quad \, (\text{det}{M} )^2 = 1 \quad \quad \, \text{det}{M} = \pm 1
\]

if $UMU^T = \text{diag}{ ( \lambda_1, \lambda_2, \lambda_3, \lambda_5) } = \lambda_1 \lambda_2 \lambda_3 \lambda_4 = (\text{det}{U})^2 \text{det}{M} = (\pm 1) (\text{det}{U} )^2$

$(0,0)$ entry of $M\eta M^T$
\[
-m_0^2 + m_1^2 + m_2^2 + m_3^2 = 1 
\]$M$ unbounded so $O(1,3)$ noncompact

\begin{theorem}{5.2} Every closed subgroup $M$ of a Lie group $G$ is a Lie subgroup
\end{theorem}

e.g. $O(n)$, $SL(n,\mathbb{R})$, $SO(n)$ Lie subgroups of $GL(n,\mathbb{R})$

$SL(n,\mathbb{R})$ closed subgroup, consider $ \begin{aligned} & f: GL(n, \mathbb{R}) \to \mathbb{R} \\ 
  & A \mapsto \text{det}{A} \end{aligned}$

$f$ cont. $\lbrace 1 \rbrace$ closed $f^{-1}(1) = SL(n,\mathbb{R})$ so $SL(n,\mathbb{R})$ closed.  By Thm. 5.2., $SL(n, \mathbb{R}) $ Lie subgroup. 

Let $G$ Lie group. \\
\phantom{Let } $H$ Lie subgroup.

Define $g\sim g'$ if $\exists \, h \in H$ s.t. $g' = gh$ 
\[
[g] = \lbrace gh | h \in G \rbrace
\]

coset space $G /H$ is a manifold (not necessarily a Lie group)

if $H$ normal subgroup of $G$, i.e. $ghg^{-1} \in H$, \, $\forall \, \begin{aligned} & \\ 
  & g \in G \\
  & h \in H \end{aligned}$, then $G/H$ Lie groupp \\

take $[g], [g'] \in G/H$ \\
Let $gh, g'h'$ representative of $[g], [g']$, resp. 

\[
\begin{gathered}
  [g][g'] = gh g'h' = gg' h'' h' \in [gg'] \\ 
  g^{-1} h \\
  [g][g^{-1} ] = gh g^{-1} h' = gg^{-1} h'' h' = e h'' h' = h'' h' \in [e]
\end{gathered}
\]





\subsubsection{ Lie algebras }

left-translation  \\
$L_g:G \to G$ \\
$L_gh = gh = x^{ik}{(g)} x^{kj}{(h)} = x^{ij}{(gh)}$ \\
$L_eh = eh = x^{ik}{ (e)} x^{kj}{ (h)} = \delta^{ik} x^{kj}{(h)} = x^{ij}{(h)} = 1h = h$ \\
$L_ge = ge = x^{ik}{(g)}x^{kj}{(e)} = x^{ik}{(g)}  \delta^{kj} = x^{ik}{(g)} = g1 = g $ \\
$L_{g*} : T_hG \to T_{gh}G$

Pushforward?  \\
Recall local coordinate form of pushforward 
\[
\begin{aligned}
  & X \equiv X_h \in T_h G \\ 
  & X_h = X_h^{ij} \left. \frac{ \partial }{ \partial x^{ij} } \right|_h 
\end{aligned}
\]
\[
L_{g*}X_h \equiv L_{g*} X = X_h^{kl} \frac{ \partial x^{ij}{ (gh) } }{ \partial x^{kl}{(h)} } \left. \frac{ \partial }{ \partial x^{ij} } \right|_{gh} = X^{ij}_{gh} \left. \frac{ \partial }{ \partial x^{ij} } \right|_{gh}
\]
Note that $x^{ik}{ (g)} x^{kj}{ (h)} = x^{ij}{ (gh)}$
\[
\Longrightarrow \frac{ \partial x^{ij}{(gh)} }{ \partial x^{kl}{ (h)} } = x^{im}{(g)} \delta^{km} \delta^{jl} = x^{ik}{ (g)} \delta^{jl} = \begin{cases} 0 & \text{ if } j\neq l \\ x^{ik} & \text{ if } j =l \end{cases}
\]

so
\[
\begin{gathered}
  L_{g*} X_h = X_h^{kl} \frac{ \partial x^{ij}{(gh)} }{ \partial x^{kl}{(h)} } \left. \frac{ \partial }{ \partial x^{ij} } \right|_{gh} = X^{kl}_h x^{ik}{(g)} \delta^{jl} \left. \frac{ \partial }{ \partial x^{ij} } \right|_{gh} = X^{kj}_h x^{ik}{(g)} \left. \frac{ \partial }{ \partial x^{ij}} \right|_{gh} = x^{ik}{(g)} X^{kj}_h \left. \frac{ \partial }{ \partial x^{ij}} \right|_{gh}
\end{gathered}
\]

santiy check: $L_{e*}X_h = eX_h = X_h$

Consider left-invariant vector fields  \\
\quad $L_{g*} X = X$ \, $\forall \, g$ (cf. wikipedia) \\
\quad $L_{g*} \left. X \right|_h = X_{gh}$ (cf. Nakahara) \\


\exercisehead{5.21} 

\begin{equation}
  L_{a^*} \left. X \right|_g = X^{\mu}(g) \frac{ \partial x^{\nu}(ag) }{ \partial x^{\mu}(g) } \left. \frac{ \partial }{ \partial x^{\nu} } \right|_{ag} = x^{\nu}(ag) \left. \frac{ \partial }{ \partial x^{\nu }} \right|_{ag} \quad \quad \, (5.110)
\end{equation}

\[
\begin{gathered}
  L_{a*} \left. X \right|_g = \left. X \right|_{ag}
\end{gathered} \quad \quad \quad \begin{gathered}
  L_a g = ag \\
  \begin{aligned} 
    y^i & = y^i(x^j ) = a_{ij} x^j \\
    \partial_j y^i & = a_{ij}
\end{aligned}
\end{gathered}
\]

\[
d(y(x)) X^{\mu}(g) \left. \frac{ \partial }{ \partial x^{\mu} } \right|_g = X^{\mu}(g) \left. \frac{ \partial }{ \partial x^{\mu} } \right|_g y^{\nu}
\]



Recall that 
\[
\begin{aligned}
  & V = V^{\mu} \frac{ \partial }{ \partial x^{\mu}} \\ 
  & f_* V = W^{\alpha} \frac{ \partial }{ \partial y^{\alpha }}
\end{aligned} \quad \quad \, W^{\alpha} = V^{\mu} \frac{ \partial y^{\alpha }}{ \partial x^{\mu}} 
\]


\[
L_{a*} \left. X \right|_g = X^{\mu}(g) \frac{ \partial y^{\nu}(ag) }{ \partial x^{\mu} } \left. \frac{ \partial }{ \partial y^{\nu} } \right|_{ag} = \left. X \right|_{ag} = Y^{\nu} \left. \frac{ \partial }{ \partial y^{\nu}} \right|_{ag}
\]


$V \in T_e G$ defines unique left invariant vector field $X_V$ 
\begin{equation}
\left. X_V \right|_g = L_{g*} V \quad \quad g \in G \quad \quad \quad (5.111)
\end{equation}



$\mathbb{g} \equiv $ set of left invariant vector fields on $G$ \\
$T_eG \to \mathbb{g}$ is an isomorphism \\
$V \mapsto X_V$ \\
$\mathbb{g} \subset \chi(G)$ \\


Lie Bracket (Sec. 5.3) also defined on $\mathbb{g}$ 

\[
\begin{gathered}
  g, ag = L_ag \in G \\ 
  X,Y \in \mathbb{g}
\end{gathered}
\]

\begin{equation}
\left. L_{a*}[ X,Y]  \right|_g = [ L_{a*} \left. X \right|_g, L_{a*} \left. Y \right|_g ] = \left. [X,Y] \right|_{ag} \quad \quad \quad (5.112)
\end{equation}
so $[X,Y] \in \mathbb{g}$ \\

e.g. $GL(n,\mathbb{R})$ coordinates given by $n^2$ entries $x^{ij}$ of the matrix.  

\[
\begin{aligned}
  & g = \lbrace x^{ij}(g) \rbrace 
  & a = \lbrace x^{ij}(a) \rbrace
\end{aligned} \, \in GL(n, \mathbb{R}) \quad \quad \, L_a g = ag = x^{ik}(a) x^{kj}(g)
\]

take $V = V^{ij} \left. \frac{ \partial }{ \partial x^{ij} } \right|_e \in T_eG$ 

\begin{equation}
\begin{aligned}
  \left. X_V \right|_g & = L_{g^*} V = V^{ij} \left. \frac{ \partial }{ \partial x^{ij} } \right|_e x^{kl}(g) x^{lm}(e) \left. \frac{ \partial }{ \partial x^{km}} \right|_g = v^{ij} x^{kl}(g) \delta_i^l \delta^m_j \left. \frac{ \partial }{ \partial x^{km}} \right|_g = V^{ij} x^{ki}(g) \left. \frac{ \partial }{ \partial x^{kj} } \right|_g = \\
  & = x^{ki}(g) V^{ij} \left. \frac{ \partial }{ \partial x^{kj} } \right|_g = (gV)^{kj} \left. \frac{ \partial }{ \partial x^{kj} } \right|_g \quad \quad \quad (5.113)
\end{aligned}
\end{equation}

\[
\begin{aligned}
  & V = V^{ij} \left. \frac{ \partial }{ \partial x^{ij} } \right|_e \\ 
  &  W = W^{ij} \left. \frac{ \partial }{ \partial x^{ij} } \right|_e 
\end{aligned} \quad \quad \quad 
\]
\[
\left. [X_V, X_W ] \right|_g = x^{ki}(g) V^{ij} \left. \frac{ \partial }{ \partial x^{kj} } \right|_g x^{ca}(g) W^{ab} \left. \frac{ \partial }{ \partial x^{cb} } \right|_g - (V \leftrightarrow W ) = 
\]
\begin{equation}
  =  x^{ij}(g) [ V^{jk} W^{kl} - W^{jk} V^{kl} ] \left. \frac{ \partial }{ \partial x^{il} } \right|_g = ( g[V,W[)^{ij} \left. \frac{ \partial }{ \partial x^{ij} } \right|_g \quad \quad \quad (5.114)
\end{equation}

\begin{equation}
  \Longrightarrow L_{g*}V = gV \quad \quad \quad (5.115) 
\end{equation}

\begin{equation}
  \left. [ X_V, X_W ] \right|_g = L_{g*}[V,W] = g[V,W] \quad \quad \quad (5.116)
\end{equation}


\begin{definition}[5.11]
  $\mathbb{g} \equiv $ set of left-invariant vector fields with Lie bracket $[ \, , \, ] : \mathbb{g} \times \mathbb{g} \to \mathbb{g}$  \quad \, Lie algebra of Lie group $G$
\end{definition}

\hrulefill

20141117 EY recap: Recapping,

\[
\begin{gathered}
  \begin{aligned}
    & L_a : G \to G \\
    & L_a g = ag \end{aligned} \quad \quad \, \begin{aligned}
    & R_a : G \to G \\ 
    & R_a g = ga \end{aligned}
\end{gathered}
\]
$L_a$, $R_a$ are diffeomorphisms; indeed $L_{a^{-1}} = (L_a)^{-1}$, $R_{a^{-1}} = (R_a)^{-1}$ \quad \, $\forall \, a \in G$

$\forall \, X_g = T_gG$, \\
locally $X_g = X_g^i \frac{ \partial }{ \partial g^i }$ \\

$L_{a*} : T_gG \to T_{ag}G$ \\
locally, $L_{a*} X_g = X^i_g \frac{ \partial (ag)^j}{ \partial g^i} \frac{ \partial}{ \partial (ag)^j}$ (because that's what pushforwards do) 

Note the abuse of notation above.

\textbf{if} $X_g$ is \textbf{left-invariant}, 
\[
L_{a*} X_g = X_{ag}
\] 
this implies
\[
X^i_g \frac{ \partial (ag)^j}{ \partial g^i} = X^j_{ag}
\]

Now isomorphisms can be shown so that
\[
T_1G = \mathfrak{g} = \lbrace X | X \in TG, \, \forall \, a,g \in G, L_{a*} X_g = X_{ag} \rbrace \text{ i.e. set of left-invariant vector fields}
\]
indeed, for instance, $\forall \, V \in T_1G$, locally $V  = V^i \left. \frac{ \partial }{ \partial x^i} \right|_1$, 
\[
\begin{gathered}
  L_{g*} V = X \\ 
  L_{a*}X = L_{a*} L_{g*} V = L_{ag*} V = X_{ag}
\end{gathered}
\]
uniqueness can be shown in either cases, note. 

\emph{Specialize} to the case of $G = GL(n)$

\[
\begin{aligned}
  & (L_ag)^{ij} = a^{ik} g^{kj} \\ 
  & X_g = X^{ij}_g \frac{ \partial }{ \partial g^{ij} } \\ 
  & L_{a*} X_g = X^{ij}_g \frac{ \partial (ag)^{kl} }{ \partial g^{ij}} \frac{ \partial }{ \partial (ag)^{kl} } = X^{ij}_g a^{ki} \frac{ \partial }{ \partial (ag)^{kj} } = (aX_g)^{kj} \frac{ \partial }{ \partial (ag)^{kj}}
\end{aligned}
\]
where I used the following calculation:
\[
(ag)^{kl} = a^{km} g^{ml} \Longrightarrow \frac{ \partial (ag)^{kl}}{ \partial g^{ij}} = a^{ki} \delta^{jl}
\]

If $X_g$ left invariant,
\[
aX_g=X_{ag}
\]



\hrulefill

e.g. $\mathbf{so}(n)$ Lie algebra of $SO(n)$

e.g. 5.15 
\begin{enumerate}
\item[(a)] $G = \mathbb{R}$  \quad \quad \, define $La : x \mapsto x + a$,  \quad \, left invariance field $X = \frac{ \partial }{ \partial x }$  

\[
L_{a*} \left. X \right|_x = \frac{ \partial (a+x) }{ \partial x} \frac{ \partial }{ \partial (a+ x) } = \frac{ \partial }{ \partial (x+a) } = \left. X \right|_{x+a}
\]

$X = \frac{ \partial }{ \partial \theta}$ \quad unique left vector field on $G = SO(2) = \lbrace e^{i \theta} | 0 \leq \theta \leq 2 \pi \rbrace$ 

\[
\frac{ \partial ( \phi + \theta ) }{ \partial \theta } \frac{ \partial }{ \partial (\phi + \theta ) } = \frac{ \partial }{ \partial ( \phi + \theta ) }
\]

\item[(b)] Let $\mathbf{gl}{ (n, \mathbb{R})}$   \quad curves $\begin{aligned} & \quad \\ 
  & c : (-\epsilon , \epsilon ) \\ 
  & c(0 ) = 1 \end{aligned}$ $ \, \to GL(n, \mathbb{R})$ \quad \quad $c(s) = 1 + sA + \mathcal{O}(s^2)$ \quad near $s=0$, $A$ \, $n\times n$ matrix of real entries 
\end{enumerate}










\subsubsection{The one-parameter subgroup }

\subsubsection{Frames and structure equation }

$\lbrace V_1 \dots V_n \rbrace$ basis of $T_1G = \mathfrak{g}$



\begin{equation}
[X_a, X_b] = c_{ab}^c X_c  \quad \quad \quad \, (5.133)
\end{equation}

From local form of the Lie bracket, cf. Exercise 5.9 
\[
[X,Y] = \left( X^{\mu} \frac{ \partial Y^{\nu}}{ \partial x^{\mu }} - Y^{\mu} \frac{ \partial X^{\nu}}{ \partial x^{\mu} } \right) \frac{ \partial }{ \partial x^{\nu}}
\]

\[
[V_a,V_b] = \left( V_a^i \frac{ \partial V_b^j}{ \partial x^i} - V_b^i \frac{ \partial V_a^j}{ \partial x^i} \right) \frac{ \partial }{ \partial x^j} = c^c_{ab}V_c = c^c_{ab} X_c^j \frac{ \partial }{ \partial x^j}
\]

dual basis to $\lbrace X_a \rbrace$, $\lbrace \theta^a \rbrace$ s.t. $\langle \theta^a, X_b \rangle = \delta^a_{ \, \, b}$ \\
dual basis satisfies \textbf{Maurer-Cartan's structure equation}
\[
d\theta^a(X_b,X_c) = X_b \delta^a_{ \, \, c} - X_c \delta^a_{ \, \, b} - \theta^a([X_b,X_c]) = X_b\delta^a_{ \, \, c} - X_c \delta^a_{ \, \, b} - \theta^a(c^d_{bc} X_d ) = X_b \delta^a_{ \, \, c} - X_c \delta^a_{ \, \, b} - c^a_{bc} = -c^a_{bc}
\]
where I used $d\omega(X,Y) = X\omega(Y) - Y\omega(X) - \omega([X,Y])$, cf. wikipedia ``exterior derivative'', etc.

\begin{equation}
  d\theta^a = \frac{-1}{2}c^a_{bc} \theta^b \wedge \theta^c \quad \quad \quad \, (5.136)
\end{equation}


%canonical 1-form, \textbf{Maurer-Cartan form on $G$ }

%\begin{equation}
%\begin{aligned}
%  & \theta : T_gG \to T_e G \\ 
%  & \theta : X \mapsto (L_{g^{-1}})_* X = (L_g)^{-1}_* X, \quad \, X \in \, T_gG
%\end{aligned} \quad \quad \quad \, (5.137)
%\end{equation}

define Lie algebra valued 1-form $\theta: T_gG \to T_1G$, \textbf{canonical 1-form} or \textbf{Maurer-Cartan form} on $G$ \\
\[
\theta \in \Omega^1(G;\mathfrak{g})
\]
\begin{equation}
  \theta : X \mapsto (L_{g^{-1}})_* X = (L_g)^{-1}_* X \text{ where } X \in T_gG \quad \quad \quad \, (5.137)
\end{equation}

\begin{theorem}[5.3] 
\begin{enumerate}
\item[(a)] canonical 1-form is $\theta = V_a \otimes \theta^a$, where $\lbrace V_a \rbrace$ basis of $T_1G = \mathfrak{g}$, $\lbrace \theta^a \rbrace$ dual basis of $T^*_gG$

EY : 20141117 here's where Nakahara has a mistake, $\theta$ isn't at $e$ but at $g$
\item[(b)] where \\
$d\theta = V_a\otimes d\theta^a$ and 
\begin{equation}
  [ \theta \wedge \theta ] \equiv [ V_{a} , V_{b} ] \otimes \theta^{a} \wedge \theta^{b} \quad \quad \quad \, (5.140)
\end{equation}
\end{enumerate}
\end{theorem}

\begin{proof}
  \begin{enumerate}
\item[(a)] $\forall \, Y = Y^a X_a \in T_gG$
\[
\theta(Y) = (L_{g^{-1}})_* Y = (L_g)^{-1}_* Y = Y^a(L_{g^{-1}})_* X_a = Y^a (L_{g^{-1}})_* (L_g)_* V_a = Y^a V_a
\]
On the other hand
\[
(V_a \otimes \theta^a)(Y) = V_a \otimes \theta^a(Y^b X_b) = V_a (Y^b(\theta^a(X_b))) = Y^a V_a
\]

EY : 20141117 note that Nakahara, I believe, made a mistake with thinking $\theta^a$ is a dual basis at $e$, not $g$

Thus
\[
\boxed{ \theta = V_a \otimes \theta^a }
\]


\item[(b)] Now $[V_a,V_b] = c^c_{ab}V_c$
\[
\begin{aligned}
  & d\theta + \frac{1}{2} [ \theta \wedge \theta ] = 0 \\
  & \frac{1}{2} [ \theta \wedge \theta ] = \frac{1}{2} [V_a,V_b] \otimes \theta^a \wedge \theta^b = \frac{1}{2} V_c \otimes c^c_{ab} \theta^a \wedge \theta^b \\ 
  & d\theta = V_c \otimes d\theta^c = V_c \otimes \frac{-1}{2} c^c_{ab} \theta^a \wedge \theta^b 
\end{aligned}
\]
\[
\Longrightarrow \boxed{ d\theta + \frac{1}{2} [ \theta \wedge \theta ] = 0  }
\]
\end{enumerate}
\end{proof}

\hrulefill

20141117 EY's recap:

$c^{c}_{ab}$ independent of $g \in G$

$\forall \, g\in G$, $\forall \, \theta^a$ in dual basis for $T_g^*G$ (also $\mathfrak{g}^*$), $\theta^a \in T_g^*G$, \\
$\forall \, d\theta^a \in \Omega^2_g(G)$, then 
\[
d\theta^a = \frac{-1}{2} c^a_{bc} \theta^b \wedge \theta^c
\]
is satisfied, \textbf{Maurer-Cartan's structure equation}.


\hrulefill


\subsection{ The action of Lie groups on manifolds }




\subsubsection{ Definitions} 


\subsubsection{ Orbits and isotropy groups }


\subsubsection{ Induced vector fields }

\subsubsection{ The adjoint representation }














\section{de Rham Cohomology Groups}

$r$-form $\omega$ in $\mathbb{R}^r$

\[
\omega = a(x) dx^1 \wedge dx^2 \wedge \dots \wedge dx^r
\]

define integration of $\omega$ over $\overline{\sigma}_r$ 

\begin{equation}
\int_{ \overline{\sigma}_r } \omega \equiv \int_{ \overline{\sigma}_r } a(x) dx^1 dx^2 \dots dx^r \quad \quad \quad \, (6.2)
\end{equation}

\[
\int_{ \overline{\sigma}_2} \omega = \int_{ \overline{\sigma}_2 } dx dy = \int_0^1 dx \int_0^{1-x} dy = \frac{1}{2}
\]

\[
\int_0^1 dx \int_0^{1-x}dy \int_0^{1-y-x} dz = \int_0^1 dx \int_0^{1-x} dy (1-y-x) = \frac{1}{6}
\]

Let smooth $f:\sigma_r \to M$

$s_r = f(\sigma_r) \subset M$ (singular) $r$-simplex in $M$

define integration of $r$-form $\omega$ over $r$-chain in $M$

\begin{equation}
\int_{s_r} \omega = \int_{ \overline{\sigma}_r } f^* \omega \quad \quad \quad \, (6.6)
\end{equation}


general $r$-chain $c = \sum_i a_i s_{r,i}  \in C_r(M)$

\begin{equation}
  \int_c \omega = \sum_i a_i \int_{s_{r,i} } \omega \quad \quad \quad \, (6.7)
\end{equation}

\subsection{ Stokes' theorem }

\begin{theorem}[Stokes' thm.]
$\begin{aligned}
& \omega \in \Omega^{r-1}(M) \\ 
& c\in C_r(M)
\end{aligned}$

then
\begin{equation}
\int_c d\omega = \int_{\partial c} \omega \quad \quad \quad \, (6.8)
\end{equation}
\end{theorem}

\begin{proof}
$c$ linear combination of $r$-simplexes \\
suffices to prove (6.8) for $r$-simplex $s_r$ in $M$ 

Let $f:\overline{\sigma}_r \to M$ s.t. $f(\overline{\sigma}_r) = s_r$

\[
\int_{s_r} d\omega = \int_{ \overline{\sigma}_r} f^*(d\omega) = \int_{\overline{\sigma}_r} d(f^* \omega)
\]

using (5.75)

Also we have 

\[
\int_{\partial s_r} \omega = \int_{ \partial \overline{\sigma}_r } f^* \omega
\]
\end{proof}



\subsubsection{ Preliminary consideration }


\subsubsection{ Stokes' theorem }

\subsection{ de Rham cohomology groups }





\section{Riemannian Geometry}

\subsection{ Riemannian manifolds and pseudo-Riemannian manifolds }

\subsubsection{ Metric tensor }

\begin{definition}[7.1]
\begin{enumerate}
  \item[(i)] $  g_p(U,V) = g_p(V,U) $
\end{enumerate}
\end{definition}

Since $g \in \tau_2^0(M)$ (2 covariant indices, type (0,2) tensor)

Recall from Ch. 5, 5.2.3, 1-forms, 
$df \in T_p^* M$ on $V \in T_p M$ defined. 
\[
\langle df, V \rangle = V[f] = V^{\mu} \frac{ \partial f}{ \partial x^{\mu} } \in \mathbb{R}
\]

If $\exists \, $ metric $g$
\[
g_p : T_p M \otimes T_p M \to \mathbb{R}
\]
define
\[
\begin{aligned}
  g_p(U, \, ) : & T_p M \to \mathbb{R} \\
  & V \mapsto g_p(U,V)
\end{aligned}
\]
Then $g_p(U, \, )$ identified with 1-form $\omega_U \in T_p^*M$ 

Similarly, $\omega \in T_p^*M$ induces $V_{\omega} \in T_pM$ by $\langle \omega, U \rangle = g(V_{\omega}, U)$ 

Thus $g_p$ isomorphism between $T_pM$ and $T_p^*M$

Consider $\begin{aligned} & \quad \\ 
  & V = v^{\alpha} \frac{ \partial }{ \partial x^{\alpha }} \quad \quad \, \text{(since $v \in T_pM$)} \\
  & \omega = \omega_{\alpha} dx^{\alpha} \quad \quad \, \text{(since $\omega \in T^*_pM$)} \\
\end{aligned}$

For arbitrary $U \in T_p M$
\[
g_p(V,U) = g_p( v^{\beta} \frac{ \partial }{ \partial x^{\beta} }, U ) = v^{\beta} g_p\left( \frac{ \partial }{ \partial x^{\beta} }, U \right) = \omega \cdot U  = \omega_{\alpha}dx^{\alpha}(U) 
\]

Let $U = U^{\lambda} \frac{ \partial }{ \partial x^{\lambda }}$

\[
\begin{gathered}
  \Longrightarrow v^{\beta} U^{\lambda} g_p\left( \frac{  \partial }{ \partial x^{\beta} }, \frac{ \partial }{ \partial x^{\lambda} } \right) = \omega_{\alpha} dx^{\alpha} U^{\lambda} \frac{ \partial }{ \partial x^{\lambda }} = \omega_{\alpha} U^{\lambda} dx^{\alpha} \left( \frac{ \partial }{ \partial x^{\lambda }} \right) = \omega_{\alpha} U^{\lambda} \frac{ \partial x^{\alpha }}{ \partial x^{\lambda }} = \omega_{\alpha} U^{\alpha } = \\
  = v^{\beta} U^{\lambda} g_{\beta \lambda} = v^{\beta} g_{\beta \alpha } U^{\alpha } \\
  \Longrightarrow \omega_{\alpha} = g_{\alpha \beta} v^{\beta}
\end{gathered}
\]

\[
g_p = g_{\mu \nu}(p) dx^{\mu} \otimes dx^{\nu} \quad \quad \, (7.1a)
\]
\[
g_{\mu \nu}(p) = g_p \left( \frac{ \partial }{ \partial x^{\mu }} , \frac{ \partial }{ \partial x^{\nu }} \right) = g_{\nu \mu}(p) \quad \, (p \in M) \quad \quad \, (7.1b)
\]

since $(g_{\mu \nu})$ has maximal rank (if $g_p(U,U) = 0$, $U=0$, so kernel is $0$), $g_{\mu \nu}$ has inverse $g^{\mu \nu}$

isomorphism between $T_p M$ and $T_p^*M$ expressed as $\omega_{\mu} = g_{\mu \nu} U^{\nu}$, $U^{\mu} = g^{\mu \nu} \omega_{\nu}$ \quad \, (7.2)

Take an infinitesimal displacement $dx^{\mu} \frac{ \partial }{ \partial x^{\mu} } \in T_p M$ 
\[
ds^2 = g \left( dx^{\mu} \frac{ \partial }{ \partial x^{\mu} }, dx^{\nu} \frac{ \partial }{ \partial x^{\nu} }\right) = g_{\mu \nu} dx^{\mu} dx^{\nu} \quad \quad \, (7.3)
\]

\exercisehead{7.1} 

\[
\left( \begin{matrix} - \frac{1}{ \sqrt{2} } & - \frac{1}{ \sqrt{2 }} & & \\ 
   - \frac{1}{ \sqrt{2}} &  \frac{1}{ \sqrt{2}} & & \\
  & & 1 & \\
  & & & 1 \end{matrix} \right) \left( \begin{matrix} & 1 & & \\
  1 & & & \\
  & & 1 &  \\
  & & & 1 \end{matrix} \right)  \left( \begin{matrix}  \frac{1}{ \sqrt{2} } & \frac{1}{ \sqrt{2 }} & & \\ 
  \frac{1}{ \sqrt{2}} & - \frac{1}{ \sqrt{2}} & & \\
  & & 1 & \\
  & & & 1 \end{matrix} \right) = \left( \begin{matrix} -1 & & & \\ & 1 & & \\ & & 1 & \\ & & & 1 \end{matrix} \right)
\]

Consider the light-cone basis

\[
\lbrace e_+ , e_-, e_2, e_3 \rbrace
\]
\[
e_{\pm} \equiv \frac{e_1 \pm e_0 }{ \sqrt{2}}
\]
\[
\begin{aligned}
  & g(e_{\pm} , e_{\pm} ) = \frac{1}{2} g( e_1 \pm e_0 , e_1 \pm e_0 ) = \frac{1}{2} ( 1 + (-1) ) = 0 \\
  &  g(e_{\pm} , e_{\mp} ) = \frac{1}{2} g(e_1 \pm e_0 , e_1 \mp e_0 ) = \frac{1}{2} ( 1 - (-1) ) = 1 
\end{aligned} \quad \quad \, \begin{aligned}
  & g_{++} = g_{--} = 0 \\ 
  & g_{+-} = g_{-+} = 1 \end{aligned}
\]
Indeed the light cone metric is $\left( \begin{matrix} & 1  & & \\ 1 & & & \\ & & 1 & \\ & & & 1 \end{matrix} \right)$
\[
\begin{aligned}
  & \omega_{\mu} = g_{\mu \nu} U^{\nu} \\ 
  & \omega_+ = V^- \\ 
  & \omega_- = V^+ \\ 
  & \omega_2 = V^2 \\ 
  & \omega_3 = V^3
\end{aligned}
\]

\subsubsection{ Induced metric }

Let $M$ be $m$-dim. submanifold of $n$-dim. Riemannian manifold $N$ with metric $g_N$.  \\
If $f: M \to N$ embedding which induces the submanifold structure of $M$.  (Sec. 5.2).  \\
(recall, smooth $f:M\to N$, $\text{dim}{ M } \leq N$, $f$ immersion if $f_*: T_pM \to T_{f(p)}N$ injection, so rank $f_* = \text{dim}{M}$.  $f$ embedding if $f$ immersion and $f$ injection.  Also $f(M)$ submanifold of $N$).  

- pullback $f^*$ induces natural metric $g_M = f^* g_N$ on $M$
\begin{equation}
  g_{M \mu \nu}(x) = g_{ N \alpha \beta} \frac{ \partial f^{\alpha }}{ \partial x^{\mu }} \frac{ \partial f^{\beta} }{ \partial x^{\nu }} \quad \quad \quad \, (7.5)
\end{equation}

Recall the pullback: $\begin{aligned} & \quad \\ 
  & f^* : T^*_{f(p)}N \to T_p^*M \\
  & \omega \in T^*_{f(p)}N \\
  & V \in T_p M \end{aligned}$ \quad \quad \, $\begin{aligned} & \quad \\ 
  & \langle f^* \omega , V \rangle = \langle \omega, f_* V \rangle \\
  & \text{ and } \\
  & f_* V[g] = V[gf], \quad \, g\in \mathcal{F}(N) \end{aligned}$

Now $\begin{gathered} 
  g_N: T_{f(p) } N \otimes T_{f(p)} N \to \mathbb{R} \\
  g_N(U, \, ) \in T^*_{f(p)} N \end{gathered}$ 

\[
\langle f^* g_N(U, \, ) , V \rangle = \langle g_N(U, \, ), f_* V \rangle  = \langle g_N(U, \, ), V^{\mu} \frac{ \partial f^{\nu} }{ \partial x^{\mu} } \frac{ \partial }{ \partial y^{\nu}} \rangle 
\]
For 
\[
f_* V[g] = V[gf] = V^{\mu} \frac{ \partial }{ \partial x^{\mu} } [gf] = V^{\mu} \frac{ \partial g}{ \partial y^{\nu}} \frac{ \partial f^{\nu } }{ \partial x^{\mu }} = V^{\mu} \frac{ \partial f^{\nu }}{ \partial x^{\mu} } \frac{ \partial g}{ \partial y^{\nu} }
\]
with $gf(x^{\mu})$
so then (20121026)
\[
\begin{gathered}
  g_N\left(U, V^{\mu} \frac{ \partial }{ \partial y^{\nu }} \right) \frac{ \partial f^{\nu} }{ \partial x^{\mu }} = g_N(U, V^{\nu} \frac{ \partial }{ \partial y^{\beta} } ) \frac{ \partial f^{\beta} }{ \partial x^{\nu }} \\ 
  \Longrightarrow g_N(f(x)) \frac{ \partial f^{\alpha }}{ \partial x^{\mu} } \frac{ \partial f^{\beta}}{ \partial x^{\nu }}
\end{gathered}
\]
with $U \in T_{f(p)}N$

For example, 
\[
\begin{gathered}
  f: (\theta, \phi) \mapsto ( s_{\theta} c_{\varphi} , s_{\theta} s_{\varphi }, c_{\theta} ) \\ 
  \nabla f = \left| \begin{matrix} c_{\theta} c_{\varphi} & - s_{\theta} s_{\varphi} \\ 
    c_{\theta} s_{\varphi} & s_{\theta} c_{\varphi} \\ 
    -s_{\theta} & 0 \end{matrix} \right|
\end{gathered}
\]

\[
g_{\mu \nu} dx^{\mu} \otimes dx^{\nu} = \delta_{\alpha \beta} \frac{ \partial f^{\alpha }}{ \partial x^{\mu }} \frac{ \partial f^{\beta}}{ \partial x^{\nu} }dx^{\mu} \otimes dx^{\nu} = d\theta \otimes d\theta + s^2_{\theta} d\varphi \otimes d\varphi
\]

\exercisehead{7.2} Let $f: T^2 \to \mathbb{R}^3$.  Embedding of torus into $(\mathbb{R}^3, \delta)$ defined by 
\[
f:(\theta, \varphi) \mapsto ((R+\cos{\theta} ) \cos{\varphi}, ( R+ rc_{\theta} ) s_{\varphi}, rs_{\theta} ), \quad \, R>r 
\]
\[
\nabla f = \left| \begin{matrix} -r s_{\theta} c_{\varphi } & - (R+ rc_{\theta} ) s_{\varphi} \\ 
  -r s_{\theta} s_{\varphi} & (R+ rc_{\theta} ) c_{\varphi} \\ 
  rc_{\theta} & 0 \end{matrix} \right|
\]
\[
g_{\mu \nu} dx^{\mu} \otimes dx^{\nu} = \delta_{\alpha \beta} \frac{ \partial f^{\alpha }}{ \partial x^{\mu} } \frac{ \partial f^{\beta} }{ \partial x^{\nu} } dx^{\mu} \otimes dx^{\nu} = r^2 d\theta \otimes d\theta + (R+rc_{\theta})^2 d\varphi \otimes d\varphi = g_{\theta \theta} d\theta \otimes d\theta + g_{\varphi \varphi }d\varphi \otimes d\varphi 
\]




\subsection{ Parallel transport, connection and covariant derivative }

\subsubsection{ Heuristic introduction }

\subsubsection{ Affine connections }

\[
\nabla_X (fY) = X[f]Y + f\nabla_X Y \quad \quad \quad (7.13d)
\]

\[
\nabla_{\nu} e_{\mu} = \nabla_{e_{\nu}} e_{\mu} = e_{\lambda} \Gamma^{\lambda}_{ \, \, \nu \mu} \quad \quad \quad (7.14)
\]
\[
\nabla_V W = V^{\mu} \nabla_{\mu} ( W^{\nu} \partial_{\nu} ) = V^{\mu} ( \partial_{\mu} W^{\nu} \partial_{\nu} + W^{\nu} \nabla_{\mu} \partial_{\nu} ) = V^{\mu} ((\partial_{\mu} W^{\nu} ) \partial_{\nu} + W^{\nu} \Gamma^{\lambda}_{\, \, \mu \nu } \partial_{\lambda} ) = V^{\mu} ( \partial_{\mu} W^{\lambda} + W^{\nu} \Gamma^{\lambda}_{\, \, \mu \nu } ) \partial_{\lambda} = V^{\mu} \nabla_{\mu} W^{\lambda} \partial_{\lambda}
\]

\[
\nabla_{\mu} ( W^{\lambda} \partial_{\lambda} ) = ( \partial_{\mu} W^{\lambda} ) \partial_{\lambda} + W^{\lambda} ( \nabla_{\mu} \partial_{\lambda} ) = \partial_{\mu} W^{\lambda} \partial_{\lambda} + W^{\lambda} \Gamma^a_{\, \, \mu \lambda} \partial_a = (\partial_{\mu} W^{\lambda} + W^b \Gamma^{\lambda}_{ \, \, \mu b} ) \partial_{\lambda}
\]

\subsubsection{ Parallel transport and geodesics }

\[
\nabla_V X = 0 \quad \quad \quad (7.18a)
\]

$X$ parallel transported along $c(t)$ where $V = \frac{d}{dt} = \left. \frac{dx^{\mu}(c(t))}{ dt} \partial_{\mu} \right|_{c(t)}$

Now
\[
\dot{X} = \partial_{\mu}X \dot{c}^{\mu} = \partial_{\mu} X V^{\mu}
\]
where $X=X(c(t))$.  So

\[
\begin{gathered}
\nabla_V X = V^{\mu} ( \partial_{\mu} X^{\lambda} + X^{\nu} \Gamma^{\lambda}_{ \, \, \mu \nu } ) \partial_{\lambda} = 0  \\
\Longrightarrow V^{\mu} \partial_{\mu} X^{\lambda} + \Gamma^{\lambda}_{ \, \, \mu \nu} V^{\mu} X^{\nu} = 0 \text{ or } \dot{X} + \Gamma^{\lambda}_{\, \, \mu \nu } \frac{dx^{\mu}(c(t))}{ dt} X^{\nu} = 0 
\end{gathered}
\]

If 
\[
\nabla_V V =0 \quad \quad \quad (7.19a)
\]

tangent vector $V(t)$ itself is parallel transported along $c(t)$.  $c(t)$ \emph{geodesic}.  


\exercisehead{7.3} Recall what left-invariant means.  A left-invariant vector field is such that 
\[
L_{a*} \left. X\right|_g = \left. X\right|_{ag}
\]

Recall

\[
\frac{d^2 x^{\mu} }{dt^2} + \Gamma^{\mu}_{ \, \nu \lambda} \frac{dx^{\nu}}{ dt} \frac{dx^{\lambda}}{ dt} = 0 \quad \quad \, (7.19b)
\]

affine reparametrization

\[
t\to at +b \quad \quad \, (a,b \in \mathbb{R})
\]

Recall 
\[
\frac{dx^{\mu}}{dt} \to \frac{dt}{dt'} \frac{dx^{\mu} }{dt}
\]

In this case, $\frac{dt}{dt'} = \frac{1}{a}$.  So under this affine reparametrization
\[
\frac{d^2 x^{\mu}}{dt^2} = \frac{d}{dt} \left( \frac{dx^{\mu}}{dt} \right) \to \frac{1}{a} \frac{d}{dt} \left( \frac{dx^{\mu} }{ dt'} \right) = \frac{1}{a^2} \frac{d^2 x^{\mu} }{dt^2}
\]

So
\[
\begin{gathered}
  \frac{d^2 x^{\mu}}{dt^2} + \Gamma^{\mu}_{ \, \nu \lambda} \frac{dx^{\nu} }{dt} \frac{dx^{\lambda}}{dt} \to   \frac{1}{a^2} \frac{d^2 x^{\mu}}{dt^2} + \Gamma^{\mu}_{ \, \nu \lambda} \frac{1}{a} \frac{dx^{\nu} }{dt} \frac{1}{a} \frac{dx^{\lambda}}{dt} = \frac{1}{a^2} \left( \frac{d^2 x^{\mu} }{ dt^2} + \Gamma^{\mu }_{ \, \nu \lambda} \frac{dx^{\nu }}{dt} \frac{dx^{\lambda} }{dt} \right) = 0 
\end{gathered}
\]

\subsubsection{ The covariant derivative of tensor fields }

Define 
\[
\nabla_X f = X[f] \quad \quad \quad (7.21)
\]

\[
\begin{aligned}
  & \nabla_X (fY) = X[f] Y + f\nabla_X Y \quad \quad \quad (7.13d) \\ 
  & \nabla_X (fY) = (\nabla_X f) Y + f\nabla_X Y \quad \quad \quad (7.13d') \\ 
\end{aligned}
\]

Require this 
\begin{equation}
  \nabla_X (T_1 \otimes T_2 ) = (\nabla_X T_1) \otimes T_2 + T_1 \otimes (\nabla_X Y_2) \quad \quad \quad (7.22)
\end{equation}

$\langle \omega, Y \rangle \in \mathcal{C}^{\infty}M$, $Y \in \mathcal{X}(M)$

\[
X[\langle \omega, Y \rangle ] = \nabla_X[ \langle \omega, Y \rangle ] = \langle \nabla_X \omega, Y \rangle + \langle \omega, \nabla_X Y \rangle
\]

\[
\begin{aligned}
  & \langle \nabla_X \omega, Y \rangle = (\nabla_X \omega)_i Y^i \\ 
  & \langle \omega, \nabla_X Y \rangle = \omega_i X^{\mu} ( \partial_{\mu} Y^i + Y^{\nu} \Gamma^i_{\mu \nu } ) \\
  & X[ \langle \omega , Y \rangle ] = X(\omega_i Y^i ) = X^i \partial_j \omega_i Y^i + X^j \omega_i \partial_j Y^i 
\end{aligned}
\]

\[
X^j \partial_j \omega_i Y^i + X^j \omega_i \partial_j Y^i = (\nabla_X \omega)_i Y^i + \omega_i X^{\mu} ( \partial_{\mu} Y^i ) + \omega_j X^{\mu} Y^i \Gamma^j_{ \mu i }
\]
\[
(\nabla_X \omega)_{\nu} = X^{\mu} \partial_{\mu} \omega_{\nu} - X^{\mu} \Gamma^{\lambda}_{ \mu \nu} \omega_{\lambda} \quad \quad \quad (7.23)
\]

$\begin{aligned} & \quad \\ & X = \partial_{\mu} \\ & X^{\nu} = \delta^{\nu}_{ \, \, \mu } \end{aligned}$  

\[
(\nabla_{\mu} \omega)_{\nu} = \partial_{\mu} \omega_{\nu} - \Gamma_{\mu \nu}^{ \lambda}\omega_{\lambda} \quad \quad \quad (7.24)
\]

Recall $\nabla_{\nu} e_{\mu} \equiv \nabla_{e_{\nu}} e_{\mu} = e_{\lambda} \Gamma^{\lambda}_{\nu \mu} $ \quad \quad (7.14)
\[
\begin{aligned}
  & \omega = \delta_j^{ \, \, i } dx^j = dx^i \\ 
  & (\nabla_{\mu} dx^i )_{\nu} = \partial_{\mu} \delta_{\nu}^{ \, \, i } - \Gamma^{\lambda}_{\mu \nu} \delta_{\lambda}^{ \, \, i} = - \Gamma^i_{\mu \nu}
\end{aligned}
\]
\[
\nabla_{\mu} dx^{\nu} = - \Gamma^{\nu}_{\mu \lambda} dx^{\lambda}
\]

\[
\nabla_{\nu} t^{\lambda_1 \dots \lambda_p }_{ \mu_1 \dots \mu_q} = \partial_{\nu} t^{\lambda_1 \dots \lambda_p}_{ \mu_1 \dots \mu_q} + \Gamma^{\lambda_1}_{ \, \nu \kappa } t^{\kappa \nu_2 \dots \lambda_p }_{\mu_1 \dots \mu_q} + \dots + \Gamma^{\lambda_p}_{ \nu \kappa} t^{\lambda_1 \dots \lambda_{p-1} \kappa }_{ \mu_1 \dots \mu_q} - \Gamma^{\kappa}_{ \nu \mu_1} t^{\lambda_1 \dots \lambda_p }_{ \kappa \mu_2 \dots \mu_q} - \dots - \Gamma^{\kappa}_{ \nu \mu_q} t^{\lambda_1 \dots \lambda_p }_{\mu_1 \dots \mu_{q-1} \kappa } \quad \quad \quad (7.26)
\]

Metric tensor.  $g:TM \times_M TM \to \mathbb{R}$ restriction to a fiber, $g_p : T_p M \times T_pM \to \mathbb{R}$

\exercisehead{7.4}

\[
\nabla_{\lambda} g_{\mu \nu} = \partial_{\lambda} g_{\mu \nu} - \Gamma^{\kappa}_{\lambda \mu} g_{\kappa \nu } - \Gamma^{\kappa}_{\lambda \nu} g_{\mu \kappa }
\]

Notice that $g(X) : \mathcal{X}(M) \to C^{\infty}(M)$ \\
so $g(X)$ a 1-form. \\

as a 1-form
\[
\begin{gathered}
  \nabla_a (g(e_i)) = \nabla_a(g_{ik} \sigma^k) = (\nabla_a g_{ik})\sigma^k + g_{ik} \nabla_a \sigma^k = (\nabla_a g_{ik}) \sigma^k + g_{ik} (- \omega_{aj}^k \sigma^j)
\end{gathered}
\]


as tensor product, using rules.
\[
\nabla_a (g(e_i)) = (\nabla_a g)e_i + g\nabla_a e_i  = (\nabla_a g) e_i + g \omega_{ai}^j e_j 
\]

subtract the 2 facts above.

\[
\begin{gathered}
  (\nabla_a g) e_i = ( \nabla_a g_{ik})\sigma^k - g_{ij} \omega^j_{ak} \sigma^k - g_{jk} \omega^j_{ai} \sigma^k  \\
  \Longrightarrow \boxed{ (\nabla_a g)_{ij} =  (\nabla_a g_{ij} ) - g_{ik} \omega_{aj}^k - g_{kj} \omega_{ai}^k  }
\end{gathered}
\]





\subsubsection{ The transformation properties of connection coefficients }

another chart $(U, \psi)$, or $UV \neq \emptyset$.  $y = \psi(p)$

Let $\lbrace  \partial_{y_a} \rbrace$ \\
$\partial_{y^a} = \delta^b_{\, \, a} \partial_{y_b}$

Recall 
\[
\begin{gathered}
\nabla_VX = V^{\mu} ( \partial_{\mu} X^{\lambda} + X^{\nu} \Gamma^{\lambda}_{\mu \nu} ) \partial_{\lambda} = 0 \\
\nabla_{\nu} \partial_{\mu} = \Gamma^{\lambda}_{ \nu \mu} \partial_{\lambda}  \quad \quad \quad (7.14) \\
\nabla_a \partial_b = \widetilde{\Gamma}^c_{ab} \partial_c \quad \quad \quad (7.28)
\end{gathered}
\]

$\partial_a = \frac{ \partial x^{\nu} }{ \partial y^{\mu} } \partial_{\nu}$


\[
\begin{gathered}
  \nabla_a \partial_b = \nabla_a \left( \frac{ \partial x^{\mu }}{ \partial y^b } \partial_{\mu} \right) = \frac{ \partial^2 x^{\mu }}{ \partial y^a \partial y^b } \partial_{\mu} + \frac{ \partial x^{\mu }}{ \partial y^b} \nabla_a \partial_{\mu} = \frac{ \partial^2 x^{\mu }}{ \partial y^a \partial y^b } \partial_{\mu} + \frac{ \partial x^{\mu}}{ \partial y^b} \left( \frac{ \partial x^{\nu} }{ \partial y^a} \Gamma^{\lambda}_{\nu \mu } \partial_{\lambda} \right)  \\
\nabla_a \partial_{\mu} = \frac{ \partial x^{\nu } }{ \partial y^a} ( \partial_{\nu} \delta_{\mu}^{ \, \, \nu} + \delta_{\mu}^{ \, \, \rho } \Gamma^{\lambda}_{\nu \rho } ) \partial_{\lambda} = \frac{ \partial x^{\nu } }{ \partial y^a} \Gamma^{\lambda}_{\nu \mu} \partial_{\lambda}
\end{gathered}
\]

\[
\begin{gathered}
  \frac{ \partial^2 x^{\nu }}{ \partial y^a \partial y^b } \partial_{\nu} + \frac{ \partial x^{\mu}}{ \partial y^b} \frac{ \partial x^{\lambda} }{ \partial y^a } \Gamma^{\nu}_{ \lambda \mu } \partial_{\nu} = \widetilde{\Gamma}^c_{ab} \frac{ \partial x^{\nu }}{ \partial y^c} \partial_{\nu} \\  
  \widetilde{\Gamma}^c_{ab} = \frac{ \partial x^{\lambda}}{ \partial y^a } \frac{ \partial x^{\mu} }{ \partial y^b } \frac{ \partial y^c}{ \partial x^{\nu} } \Gamma^{\nu}_{\lambda \mu } + \frac{ \partial y^c}{ \partial x^{\nu }} \frac{ \partial^2 x^{\nu }}{ \partial y^a \partial y^b } \quad \quad \quad (7.29)
\end{gathered}
\]



\subsubsection{ The metric connection }

\subsection{ Curvature and torsion }

\subsubsection{ Definitions }

\subsubsection{ Geometrical meaning of the Riemann tensor and the torsion tensor }

\subsubsection{ The Ricci tensor and the scalar curvature }

\subsection{ Levi-Civita connections }

\subsubsection{ The fundamental theorem }

\subsubsection{ The Levi-Civita connection in the classical geometry of surfaces }

\subsubsection{ Geodesics }

\subsubsection{ The normal coordinate system }

\subsubsection{ Riemann curvature tensor with Levi-Civita connection}

\subsection{ Holonomy }

\subsection{ Isometries and conformal transformations }

\subsubsection{ Isometries }

\subsubsection{ Conformal transformations }

\subsection{ Killing vector fields and conformal Killing vector fields }

\subsubsection{ Killing vector fields }

\subsubsection{ Conformal Killing vector fields }

\subsection{ Non-coordinate bases }

\subsection{ Differential forms and Hodge theory }




\section{Complex Manifolds}

\subsection{Complex manifolds }

\subsubsection{}

\subsubsection{Examples}

\emph{Example 8.3}
Consider 
\[
\mathbb{C}P^n
\]

$z=(z^0 \dots z^n) \in \mathbb{C}^{n+1}$ defines complex line through $0$ if $z\neq 0$.

$z\sim w$ if $\exists \, a \neq 0$, $a\in \mathbb{C}$, s.t. $w=az$

Then $\mathbb{C}P^n := (\mathbb{C}^{n+1} - \lbrace 0 \rbrace ) /\sim$.

$z^0, z^1, \dots z^n$ are homogeneous coordinates, $\equiv [ z^0, z^1 \dots z^n ]$.

\[
(z^0 , \dots z^n) \sim (\lambda z^0, \dots , \lambda z^n) \qquad \, (\lambda \neq 0)
\]
chart $U_{\mu} \subset \mathbb{C}^{n+1}-\lbrace 0 \rbrace$ s.t. $z^{\mu} \neq 0$.

In chart $U_{\mu}$, \\
inhomogeneous coordinates defined by $\xi^{\nu}_{ (\mu)} := \frac{z^{\nu}}{ z^{\mu} } \xi^{\lambda}_{ (\nu) }$.

$\psi_{ \mu \nu}$ is multiplication by $z^{\nu}/ z^{\mu}$ which is, of course, holomorphic.

EY: In summary, 
\begin{equation}
\begin{gathered}
  \text{ chart atlas } \lbrace (U_{\mu}, (\xi^{\nu}_{ (\mu) } )_{ \nu \neq \mu } ) \rbrace \qquad \, \mu = 0 , 1 , \dots n, \, \nu \neq \mu , \, U_{\mu} \subset \mathbb{C}^{n+1}- \lbrace 0 \rbrace \text{ s.t. } z^{\mu} \neq 0 \\
\text{ inhomogeneous coordinates } \xi_{ (\mu) }^{\nu} = \frac{ z^{\nu} }{ z^{\mu}} , \qquad \, \nu \neq \mu \\
\text{ transition maps: } \text{ on } U_{\mu} \cap U_{\nu} \neq \emptyset , \\
\qquad \, \text{ holomorphic } \begin{aligned} & \quad \\
  &  \psi_{\mu \nu} : \mathbb{C}^n \to \mathbb{C}^n \\
  & \xi^{\lambda}_{ (\nu  ) } \mapsto \xi^{\lambda}_{( \mu ) } = \frac{z^{\nu} }{ z^{\mu} } \xi^{\lambda}_{(\nu) }
  \end{aligned}
  \end{gathered}
\end{equation}


e.g. $n=1$.

\[
\begin{aligned}
& U_0 \subset \mathbb{C}^2 - \lbrace 0 \rbrace , \, z^0 \neq 0 \, \qquad \,  & \xi^1_{(0)} = \frac{z^1}{z^0} \\ 
  & U_1 \subset \mathbb{C}^2 - \lbrace 0 \rbrace , \, z^1 \neq 0 \, \qquad \,  & \xi^0_{(1)} = \frac{z^0}{z^1}   
\end{aligned}
\]
\[
\begin{aligned}
& \psi_{10} : \mathbb{C} \to \mathbb{C} \\
& \xi_{(1)}^0 = \frac{z^0}{z^1} = 1/\xi^1_{(0)}
  \end{aligned}
\]
otherwise, trying to plug this in results in a senseless result, 1:
\[
 \xi_{(0)}^1 = \frac{z^1}{z^0} \mapsto \xi^1_{(1)} = \frac{z^0}{z^1} \xi^1_{ (0)} = 1 
\]
or, by changing, over simplifying notation, for $(z,w) \in \mathbb{C}^2$,
\[
\begin{aligned}
& U \subset \mathbb{C}^2 - \lbrace 0 \rbrace , \, z \neq 0 \, \qquad \,  & Z = w/z  \\ 
  & V \subset \mathbb{C}^2 - \lbrace 0 \rbrace , \, w \neq 0 \, \qquad \,  & W = z/w    
\end{aligned}
\]
and so, simply,
\[
W = 1/Z = W(Z)
\]

e.g. $n=2$  

\[
\begin{aligned}
& U_0 \subset \mathbb{C}^2 - \lbrace 0 \rbrace , \, z^0 \neq 0 \, \qquad \,  & \xi^1_{(0)} = \frac{z^1}{z^0},  \xi^2_{(0)} = \frac{z^2}{z^0}  \\ 
  & U_1 \subset \mathbb{C}^2 - \lbrace 0 \rbrace , \, z^1 \neq 0 \, \qquad \,  & \xi^0_{(1)} = \frac{z^0}{z^1} , \xi^2_{(1)} = \frac{z^2}{z^1}   \\
  & U_2 \subset \mathbb{C}^2 - \lbrace 0 \rbrace , \, z^2 \neq 0 \, \qquad \,  & \xi^0_{(2)} = \frac{z^0}{z^2} , \xi^1_{(2)} = \frac{z^1}{z^2}  
\end{aligned}
\]

\[
\begin{aligned}
& \begin{aligned}
& \psi_{10} : \mathbb{C}^2 \to \mathbb{C}^2 \\
    & \xi_{(1)}^0 = \frac{z^0}{z^1} = 1/\xi^1_{(0)} \\
    & \xi_{(0)}^2 \mapsto \xi_{(1)}^2 = \frac{z^0}{z^1} \xi^2_{(0)} = \xi^2_{(0)} / \xi^1_{(0)} 
  \end{aligned} \\
  &  \begin{aligned}
& \psi_{20} : \mathbb{C}^2 \to \mathbb{C}^2 \\
       & \xi^0_{(2)} = 1/ \xi^2_{(0)} \\
       & \xi^1_{(2)} =  \xi^1_{(0)} / \xi^2_{(0)} 
     \end{aligned} \\
    &  \begin{aligned}
& \psi_{21} : \mathbb{C}^2 \to \mathbb{C}^2 \\
       & \xi^0_{(2)} = \xi^0_{(1)} / \xi^2_{(1)}  \\ 
       & \xi^1_{(2)} =  1/ \xi^2_{(1)} \\
       \end{aligned} 
\end{aligned}
\]

I will try to write the ``embedding'' (not proved yet) into $\mathbb{C}^3$:
\[
\begin{aligned}
  & \varphi_0^{-1}, \varphi_1^{-1}, \varphi_2^{-1} : \mathbb{C}^2 \to U_0 \subset \mathbb{C}^3/0, z^0 \neq 0 ,  U_1 \subset \mathbb{C}^3/0, z^1 \neq 0, U_2 \subset \mathbb{C}^3/0, z^2 \neq 0 \\
  & \varphi_0^{-1}:(Z,W) = (1,Z,W) \\ 
  & \varphi_1^{-1}:(Z,W) = (Z,1,W) \\ 
  & \varphi_2^{-1}:(Z,W) = (Z,W,1) 
\end{aligned}
\]





\subsection{ Calculus on complex manifolds }



\section{Fibre bundles}

\subsection{ Tangent bundles }

$\pi^{-1}(p) = T_p M$, fibre at $p$

section of $TM$, \, $s: M \to TM$ s.t. $\pi \circ s = 1_M$

local section $s_i : U_i \to TU_i$ on chart $U_i$.  

\[
\Longrightarrow \begin{aligned} &  s(p) = X \equiv \left. X \right|_p \equiv U \in TM \\
 & \pi(u) = p \end{aligned}
\]





\subsection{ Fibre bundles }


\subsubsection{ Definitions}


\begin{definition}[9.1] (diff.) fibre bundle $(E, \pi , M, F , G)$ 
\begin{enumerate}
\item[(i)] diff. manifold $E$ total space.  
\item[(ii)] diff. manifold $M$ base space. 
\item[(iii)] diff. manifold $F$ fibre (or typical fibre)
\item[(iv)] surjection $\pi : E \to M$ projection.  \\
\phantom{surjection} $\pi^{-1}(p) = F_p \simeq F$ fibre at $p$.  
\item[(v)] Lie group $G$ structure group.  Left action on $F$.   
\item[(vi)] open cover $\lbrace U_i \rbrace $ of $M$, diffeomorphism $\Phi_i : U_i \times F \to \pi^{-1}(U_i)$ s.t. 
\[
\pi \Phi_i(p, f) = p 
\]
$\Phi_i$ local trivialization since $\Phi_i^{-1} : \pi^{-1}(U_i)$ onto direct product $U_i \times F$
\item[(vii)]  $\Phi_i(p,f) = \Phi_{i,p}(f)$, $\Phi_{i,p} : F \to F_p$ diffeomorphism.  \\
on $U_i U_j \neq \emptyset$, $t_{ij}(p) \equiv \Phi_{i,p}^{-1} \Phi_{j, p } : F \to F$, require $t_{ij}(p) \in G$ \\
Then $\Phi_i, \Phi_j$ related by smooth $t_{ij} : U_i U_j \to G$ as 
\[
\Phi_j(p,f ) = \Phi_i(p,t_{ij}(p) f ) \quad \quad \quad (9.4)
\]
$t_{ij}$ transition functions.  
\end{enumerate}
\end{definition}

coordinate bundle $(E, \pi, M , F, G, \lbrace U_i \rbrace, \lbrace \Phi_i \rbrace)$, $\lbrace U_i \rbrace$ specified covering of $M$.  

Take chart $U_i$ of $M$.  

$\pi^{-1}(U_i)$ diffeomorphic to $U_i \times F$.  

$\Phi_i^{-1} : \pi^{-1}(U_i) \to U_i \times F$ diffeomorphism.  

Let $u$ s.t. $\pi(u) = p \in U_i U_j$.  
\[
\begin{aligned}
  & \Phi_i^{-1}(u) = ( p , f_i ) \\ 
  & \Phi_j^{-1}(u) = ( p , f_j )
\end{aligned}
\]

$\exists \, t_{ij} : U_i U_j \to G$, s.t. $f_i = t_{ij}(p) f_j$


possible set of transition functions is far from unique.  

Let $\lbrace \phi_i \rbrace, \lbrace \widetilde{\phi}_i \rbrace$ be 2 sets of local trivializations giving rise to the same fiber bundle.  

\[
\begin{aligned}
  & t_{ij}(p) = \phi^{-1}_{i,p} \phi_{j,p}  \quad \quad \quad \, (9.7a) \\ 
  & \widetilde{t}_{ij}(p) = \widetilde{\phi}^{-1}_{i,p} \widetilde{\phi}_{j,p} \quad \quad \quad \, (9.7b)
\end{aligned}
\]

define $\begin{aligned} & \quad \quad \\ 
  & g_i(p) : F \to F \quad \quad \quad \, \forall \, p \in M \\ 
  & g_i(p) \equiv \phi^{-1}_{i,p} \circ \widetilde{\phi}_{i,p} \end{aligned}$

require $g_i(p)$ homeomorphism s.t. $g_i(p) \in G$.  



\subsubsection{ Reconstruction of fibre bundles }


\subsubsection{ Bundle maps }


\subsubsection{ Equivalent bundles }


\subsubsection{ Pullback bundles }



\subsubsection{ Homotopy axiom }





\subsection{ Vector bundles }


\subsubsection{ Definitions and examples }



\subsubsection{ Frames }


\subsubsection{ Cotangent bundles and dual bundles }



\subsubsection{ Sections of vector bundles }



\subsubsection{ The product bundle and Whitney sum bundle }



\subsubsection{ Tensor product bundles }




\subsection{ Principal bundles }




\subsubsection{ Definitions }

$\Phi_i: U_i \times G \to \pi^{-1}(U_i)$ local trivialization

right action

$\Phi_i^{-1}(ua) = (p,g_i a) $ \\
$ua = \Phi_i(p,g_i a)$ \\
$\forall \, a \in G, \, u \in \pi^{-1}(p)$ \\

Since the right action commutes with the left action  

EY : ??? Since the right action commutes with the left action   ??? \\
Theodore Frankel says: cf. pp. 455, Proof of Theorem 17.8, 
\[
g_ja = \tau_{ji}(p)g_i a = \tau_{ji} g_i a 
\]
i.e.
\[
ua = \Phi_j(p, g_ja) = \Phi_j(p,\tau_{ji}(p)g_ia ) = \Phi_i(p, g_ia)
\]



if $p \in U_i \bigcap U_j$
\[
ua = \Phi_j(p,g_j a) = \Phi_j(p, \tau_{ji}(p) g_i a) = \Phi_i(p,g_i a)
\]
since $\tau_{ji} = \Phi_j^{-1} \Phi_i$

Thus right multiplication defined without reference to local trivializations.  Notation of this : i.e. $P\times G \to P$ or $(u,a) \mapsto ua$




$\pi(ua) = \pi(u)$

EY : 
\[
\boxed{ \pi(ua) = \pi(u) }
\]

right action of $G$ on $\pi^{-1}(p)$ transitive since $G$ acts on $G$ transitively on the right and $F_p = \pi^{-1}(p)$ diffeomorphic to $G$ \\

cf. wikipedia Group action, types of action

action of $G$ on $X$, \\
transitive if $X \neq \emptyset $, $\forall \, x , y \in X, \, \exists \, g \in G$ s.t. $gx = y$ \\
free if given $g,h\in G$, $\exists \, x \in X$, with $gx  = hx$ implies $g=h$ i.e. if $gx = x$, (i.e. if $g$ has at least 1 fixed pt.), then $g=e$ \\

For $u \in \pi^{-1}(p)$, $p \in U_i$, $\exists \, !, \, g_U \in G$ s.t. $u = s_i(p)g_u$ \\
define $\Phi_i^{-1}(u) = (p, g_u)$   \\

$t_{ji} = \Phi_j^{-1}\Phi_i(p) \equiv \Phi_j^{-1} \Phi_i$

\[
\begin{aligned}
  & (p,e) = \Phi_i^{-1}(s_i(p)) = \Phi_i^{-1}(u) \\ 
  &  ug = \Phi_i(p,e)g = \Phi_i(p,g)
\end{aligned}
\]



\emph{Example 9.7}

Let $P$ be principal bundle with fiber $U(1) = S^1$, base space $S^2$ \\
This principal bundle represents the topological setting of the \textbf{magnetic monopole}.   \\

$G = U(1) = S^1$ \\
$M=S^2$ \\

\[
\begin{aligned}
  & \Phi_N : U_N \times S^1 \to \pi^{-1}{ (U_N) } \\ 
  & \Phi_S : U_S \times S^1 \to \pi^{-1}{ (U_S) } 
\end{aligned}
\]

Let $u\in \pi^{-1}{ (x)}$ 
\[
u = \Phi_N(x,e^{i \alpha_N} ) = \Phi_S( x, e^{ i \alpha_S } )
\]

\[
\begin{aligned}
  & \tau_{NS} = \Phi_N^{-1} \Phi_S \\ 
  & \tau_{NS}g_S = g_N
\end{aligned}
\]

then $\tau_{NS} = e^{ i (\alpha_N - \alpha_S)}$

Now $\tau_{NS}: U_S \times G \to U_N \times G$  

But on equator, $S^1$

uniquely define $\tau_{NS}(p)$ on equator 

\[
\tau_{NS}(p) = e^{i n \phi } \quad \, n \in \mathbb{Z}
\]

EY : why like this? but surely
\[
n\phi = \alpha_N - \alpha_S 
\]
(from EY)

Note that 

\[
e^{in\phi}  \to e^{ i n (\phi + 2 \pi m ) } = e^{in \phi } e^{i \pi 2mn }
\]
surely (EY)


\emph{Example 9.8}. If we identify all the infinite points of the Euclidean space $\mathbb{R}^m$, the 1-point compactification $S^m = \mathbb{R}^m \bigcup \lbrace \infty \rbrace$ is obtained.  \\
If a trivial $G$ bundle is defined over $\mathbb{R}^m$ we shall have a new $G$ bundle over $S^m$ after compactification, which is not necessarily trivial.  


\[
A = \left( \begin{matrix} u & v \\ 
  -v^* & u^* \end{matrix} \right)
\]

$\begin{aligned}
  & u = t+iz \\
  & v = y + ix \end{aligned}$

\[
\begin{gathered}
  \pi_3(SU(2)) \cong \pi_3(S^3) \cong \mathbb{Z} \\ 
  \pi_3(S^3) = \lbrace [f] | f: S^3 \to X = S^3 \rbrace
\end{gathered}
\]

$\pi_3(S^3)$ classifies maps from $S^3$ to $SU(2) \cong S^3$


\begin{equation}
  \begin{aligned}
    & f:S^3 \to S^3 \cong SU(2) \\ 
    & f(x,y,z,t) \mapsto \left( \begin{matrix} t + iz & y + ix \\ 
      -y +ix & t-iz \end{matrix} \right) = t1 + i (x \sigma_x + y \sigma_y + z \sigma_z ) \quad \quad \quad \, (9.47)
\end{aligned}
\end{equation}

$p = (x,y,z,t) \in U_N \bigcap U_S$ \\
$R = (x^2 + y^2 + z^2 + t^2)^{1/2}$ \\

$\tau_{NS}{(p)} = \frac{1}{R}(t1 + ix^i \sigma_i)$

\[
\begin{aligned}
  & \Phi_N^{-1}(u) = (p,g_N) \\ 
  & \Phi_S^{-1}(u) = (p,g_S)
\end{aligned}
\]
where $g_N, g_S \in SU(2)$



\subsubsection{ Associated bundles }

principal fiber bundle $P(M,G)$ \\
Let $G$ act on manifold $F$ on left  \\

define action of $g \in G$ on $P\times F$ by 
\[
(u,f) \mapsto (ug, g^{-1}f), \quad \, \begin{aligned} & \quad \\
  & u \in P \\
& f \in F \end{aligned}
\]

\textbf{associated fiber bundle } \, $(E, \pi, M, G, F, P)$ is equivalence class $P \times F/ G$ which 
\[
(u, f) \sim ( ug, g^{-1}f)
\]

e.g.  

Consider $F$ $k$-dim. vector space $V$ \\
\quad $\rho$ \, $k$-dim. representation of $G$

$P\times \rho V$ 
\[
(u,v) \sim (ug, \rho(g)^{-1} v) \text{ of } P \times V, \quad \, \begin{aligned} 
  & u \in P \\
& g \in G \\
  & v \in V \end{aligned}
\]

e.g. $P(M,GL(k, \mathbb{R}))$

associated vector bundle over $M$ with fiber $\mathbb{R}^k$

$E = P \times_{\rho} V$ \quad \, $\begin{aligned}
  & \quad \\
  & \pi_E : E \to M \\
  & \pi_E(u,v) = \pi(u) 
\end{aligned}$ \\

$\pi(u) = \pi(ug)$ implies 

\[
\pi_E(ug, \rho(g)^{-1}v) = \pi(ug) = \pi_E(u,v)
\]

local trivialization $\Phi_i : U_i \times V \to \pi^{-1}_E(U_i)$


transition function of $E$ given by $\rho(t_{ij}(p))$ where $t_{ij}(p)$ is that of $p$




\subsubsection{ Triviality of bundles }


\subsection*{ Problems }




\section{Connections on Fibre Bundles}

\subsection{Connections on principal bundles }



\subsubsection{}

\subsubsection{}

\subsubsection{ The local connection form and gauge potential }

Let $\lbrace U_i \rbrace$ open covering of $M$ \\
\phantom{ Let } $\sigma_i$ local section defined on each $U_i$   \\ 

introduce Lie algebra-valued 1-form $\mathcal{A}_i$ on $U_i$

\begin{equation}
  \mathcal{A}_i \equiv \sigma_i^* \omega \in \mathfrak{g} \otimes \Omega^1(U_i) \quad \quad \quad \, (10.6)
\end{equation}


$\omega \in \mathfrak{g} \otimes T^*P$ \\

Remember $\omega \in \mathfrak{g} \otimes T^*P$ \\
\phantom{Remember } $\pi \sigma = 1$ \\
\phantom{Remember } $\pi \sigma(p) = p$, \, $p \in U \subset M$ \\

given Lie algebra valued 1-form $A_i$ on $U_i$, we can reconstruct connection 1-form $\omega$ s.t. $\sigma_i^* \omega = A_i$ i.e.

\begin{theorem}[10.1]
  given $\mathfrak{g}$-valued 1-form $A_i$ on $U_i$, $A_i \in \mathfrak{g} \otimes \Omega^1(U_i)$, \\
\phantom{given} local section $\sigma_i:U_i \to \pi^{-1}(U_i)$ \\

$\exists \, $ connection 1-form $\omega$ s.t. $A_i = \sigma_i^* \omega$

\end{theorem}

\begin{proof}
define $\mathfrak{g}$ valued 1-form $\omega$ on $P$ 
\begin{equation}
  \omega_i \equiv g_i^{-1} \pi^* A_i g_i + g_i^{-1} d_P g_i \quad \quad \quad \, (10.7)
\end{equation}

$d_P$ exterior derivative on $P$ \\

$g_i$ canonical local trivialization $\Phi_i^{-1}(u) = (p,g_i)$ \quad \, $u = \sigma_i(p)g_i$ \\

for $X \in T_pM$, 

\[
\begin{gathered}
  \sigma_i^*\omega_i(X) = \omega_i(\sigma_{i*}X) = g_i^{-1} \pi^* A_i g_i(\sigma_{i*} X) + g_i^{-1} d_P g_i(\sigma_{i*}X) = \\
  = \pi^*A_i(\sigma_{i*} X) + d_Pg_i(\sigma_{i*}X) = A_i(\pi_* \sigma_{i*}X) + d_Pg_i(\sigma_{i*} X)
\end{gathered}
\]
\end{proof}






\subsubsection{ Horizontal lift and parallel transport }

\begin{theorem}[10.2] Let $\gamma:[0,1] \to M$, $u_0 \in \pi^{-1}(\gamma(0))$ \\
Then $\exists \, ! \, $ horizontal lift $\widetilde{\gamma}(t)$ in $P$ s.t. $\widetilde{\gamma}{(0)} = u_0$ 
\end{theorem}



\begin{equation}
  g_i(\gamma(t)) = g_i(t) = \mathcal{P}\exp{ \left( -\int_0^t A_{i \mu} \frac{dx^{\mu} }{dt} dt \right) } = \mathcal{P}\exp{ \left( - \int_{\gamma(0)}^{\gamma(t)} A_{i\mu}(\gamma(t))dx^{\mu} \right) } \quad \quad \quad \, (10.14)
\end{equation}

\subsection{Holonomy}

\subsubsection{Definitions}

loop $\gamma \subset M$ defines \emph{transformation} $\tau_{\gamma}: \pi^{-1}(p) \to \pi^{-1}(p)$ on fiber \\
following from (10.18), EY : ??? , transformation compatible with right action

\begin{equation}
  R_g \Gamma(\widetilde{\gamma}) = \Gamma(\widetilde{\gamma})R_g \quad \quad \quad \, (10.18)
\end{equation}

\begin{equation}
  \tau_{\gamma}(ug) = \tau_{\gamma}(u)g  \quad \quad \quad \, (10.21)
\end{equation}


\[
C_p(M) = \lbrace \gamma: [0,1] \to M | \gamma(0)=  \gamma(1) = p \rbrace
\]

subgroup of structure group $G$, \textbf{holonomy group} at $u$ is 
\begin{equation}
 \Phi_u \equiv \lbrace g \in G | \tau_{\gamma}(u) = ug, \, \gamma \in C_p(M) \rbrace   \quad \quad \quad \, (10.22)
\end{equation}



\exercisehead{10.6} 


solution already found!  Recall

\begin{equation}
  g_i(\gamma(t)) = g_i(t) = \mathcal{P}\exp{ \left( -\int_0^t A_{i \mu} \frac{dx^{\mu} }{dt} dt \right) } = \mathcal{P}\exp{ \left( - \int_{\gamma(0)}^{\gamma(t)} A_{i\mu}(\gamma(t))dx^{\mu} \right) } \quad \quad \quad \, (10.14)
\end{equation}

\[
g_i(\gamma(1)) = g_i(1) = \mathcal{P}\exp{ \left( -\int_0^1 A_{i \mu} \frac{dx^{\mu} }{dt} dt \right) } = \mathcal{P}\exp{ \left( - \int_{\gamma(0)}^{\gamma(1)} A_{i\mu}(\gamma(t)) dx^{\mu} \right) } = \mathcal{P}\exp{ \left( - \oint A_{i\mu} dx^{\mu} \right) }
\]


\begin{equation}
  g_{\gamma} = \mathcal{P} \exp{ \left( - \oint_{\gamma} A_{i \mu } dx^{\mu}  \right) } \quad \quad \quad \, (10.26)
\end{equation}

\hrulefill


\section{Characteristic Classes}

% e.g. Sec. 10.5, $SU(2)$ bundle over $S^4$ classified by homotopy group $\pi_3(SU(2)) \cong \mathbb{Z}$ \\
% also, $\pi_3(SU(2))$ evaluated by integrating $\text{tr}{ \mathcal{F}^2} \in H^4(S^4)$ over $S^4$ (Thm. 10.7)

e.g. cf. Sec. 10.5, $SU(2)$ bundle over $S^4$ classified by $\pi_3(SU(2)) \cong \mathbb{Z}$ \\
number $n \in \mathbb{Z}$ tells us how transition functions twist local pieces of the bundle when glued together.  \\
cf. Thm. 10.7, $\pi_3(SU(2))$ evaluated by integrating $\text{tr}{ F^2} \in H^4(S^4)$ over $S^4$


\subsection{Invariant polynomials and the Chern-Weil homomorphism }





\subsubsection{Invariant polynomials}

$M(k,\mathbb{C})$ - set of complex $k\times k$ matrices  \\

Let $S^r(M(k,\mathbb{C}))$ denote vector space of symmetric $r$-linear $\mathbb{C}$-valued functions on $M(k,\mathbb{C})$, i.e. \\
\quad \quad \, $\widetilde{P}: \bigotimes^r M(k,\mathbb{C}) \to \mathbb{C}$ \\
\quad \quad \quad \, $\widetilde{P} \in S^k(M(k,\mathbb{C}))$ 

\begin{equation}
  \widetilde{P}{ (a_1 \dots a_i \dots a_j \dots a_r ) } =\widetilde{ P}{ (a_1 \dots a_j \dots a_i \dots a_r) } \quad \quad \quad \, 1\leq i , j \leq r \quad \quad \quad \, (11.1)
\end{equation}
where $a_p \in GL(k,\mathbb{C})$ \\

$\widetilde{P} \in S^r(\mathfrak{g})$ invariant if $\forall \, g \in G$, $A_i \in \mathfrak{g}$, 

\begin{equation}
  \widetilde{P}{ (\text{Ad}_g{A_1} \dots \text{Ad}_g{A_r})} = \widetilde{P}(A_1 \dots A_r) \quad \quad \quad \, (11.3)
\end{equation}

where

\[
\text{Ad}_gA_i = g^{-1} A_i g
\]

e.g. 
\begin{equation}
  \widetilde{P}(A_1 \dots A_r) = \text{str}{ (A_1 \dots A_r)}  \equiv \frac{1}{r!} \sum_P \text{tr}{ (A_{ P(1)} \dots A_{P(r)} ) }
\end{equation}


invariant polynomial $P$ of degree $r$

\begin{equation}
  P(A) \equiv \widetilde{P}( \underbrace{ A \dots A }_{ r } ) \quad \quad \, A \in \mathfrak{g}, \quad \, \widetilde{P} \in I^r(G) \quad \quad \quad \, (11.6)
\end{equation}

Conversely $P$ defines invariant and symmetric $r$-linear form $\widetilde{P}$ by 
\[
\begin{gathered}
  P(t_1A_1 + \dots + t_r A_r ) \\ 
  \widetilde{P}(t_1 A_1 + \dots + t_r A_r \dots t_1 A_1 + \dots + t_r A_r ) \Longrightarrow t_1 \dots t_r \widetilde{P}(A_1 \dots A_r) \text{ term }
\end{gathered}
\]
$\frac{1}{r!} \widetilde{P}(A_1 \dots A_r)$ term the polarization of $P$


In the previous chapter, introduced local gauge potential $A$, field strength $F$ on a principal bundle.  \\
\quad We have shown these geometrical objects describe the associated vector bundles as well. \\
\quad Since the set of connections $A_i$ describes the twisting of a fiber bundle, the non-triviality of a principal bundle is equally shared by associated vector bundle \\
\quad \quad In fact, if (10.57) employed as definition of local connection in a vector bundle, it can be defined even without reference to the principal bundle with which it is originally associated \\
Later we encounter situations in which use of vector bundles is essential (the Whitney sum bundle, the splitting principle, etc.)

\begin{theorem}[11.1] (Chern-Weil theorem)

Let invariant polynomial $P$

\begin{enumerate}
\item[(a)] $dP(F) = 0$ 
\item[(b)] $F,F'$ curvature 2-forms corresponding to different connections $A, A'$.  Then $P(F')- P(F)$ exact.
\end{enumerate}
\end{theorem}


\begin{proof}
\begin{enumerate}
\item[(a)] Consider invariant polynomial $P_r(F)$ homogeneous of degree $r$, since any invariant polynomial can be decomposed into homogeneous polynomials 

\[
\begin{aligned}
  & \Omega_i = X_i \eta_i \\ 
  & d\Omega_i = X_i d\eta_i
\end{aligned}
\]

\begin{equation}
  \sum_{i=1}^r (-1)^{ p ( p_1 + \dots + p_i ) } \widetilde{P}_r( \Omega_1 \dots [\Omega_i, A ] \dots \Omega_r ) = 0 \quad \quad \quad \, (11.12)
\end{equation}



\item[(b)] Let $A, A'$ 2 connections on $E$, $F, F'$ respective field strength \\

Define interpolating gauge potential $A_t$ 

\begin{equation}
  A_t \equiv A + t \theta \quad \quad \, \theta \equiv ( A' - A) \quad \quad \, 0 \leq t \leq 1 \quad \quad \quad \, (11.15)
\end{equation}



\[
\begin{gathered}
  \begin{aligned}
    & F_1 = F+ D\theta + \theta^2 \\
    & F_0 = F \\ 
    & F' = dA' + (A')^2 \\ 
    & F = dA + A^2 
\end{aligned}  \quad \quad \quad \, \begin{gathered} F_1 - F_0 = D\theta + \theta^2 = d\theta + [A, \theta] +  \theta^2 = F' - F \end{gathered} \\
\begin{gathered}
  F' - F = d\theta + (A')^2 - A^2 \\ 
D\theta = d\theta + [A, \theta ] = d\theta + A \wedge \theta + \theta \wedge A \\ 
A \wedge \theta +  \theta \wedge A = A \wedge ( A' - A) + (A' - A) \wedge A = A\wedge A' - A^2 + A' \wedge A - A^2 \\
\theta^2 = (A' - A) \wedge (A' - A) = (A')^2 - A' \wedge A - A \wedge A' + A^2
\end{gathered}
\end{gathered}
\]

\begin{equation}
  P_r(F') - P_r(F) = P_r(F_1) - P_r(F_0) = \int_0^1 dt \frac{d}{dt}P_r(F_t)=  r\int_0^1 dt \widetilde{P}_r\left( \frac{dF_t}{dt}, F_t \dots F_t\right) \quad \quad \quad \, (11.17)
\end{equation}

\begin{equation}
  \frac{d}{dt} P_r(F_t) = r \widetilde{P}_r( D\theta, F_t \dots F_t) + 2rt \widetilde{P}_r(\theta^2, F_t \dots F_t) \quad \quad \quad \, (11.18)
\end{equation}


Use (11.12) and $\begin{aligned} & \quad \\ 
  & \Omega_1 = A = \theta \\ 
  & \Omega_2 = \dots = \Omega_m = F_t \end{aligned}$ \quad \quad \, $\begin{aligned} & \quad \\ 
  & p_1 = 1 \\
  & p_2 = \dots = p_r = 2 \end{aligned}$ 

\[
\begin{gathered}
  \sum_{i=1}^r (-1)^{ p ( p_1 + \dots + p_i ) } \widetilde{p}_r( \Omega_1 \dots [\Omega_i , A ] \dots \Omega_r ) = 0 \Longrightarrow \widetilde{P}_r( [\theta , A ] , F_t \dots F_t) + (r-1) \widetilde{P}_r( \theta, [F_t, \theta]  , F_t \dots F_t) = 0 
\end{gathered}
\]

From (11.18), (11.19) and the previous identity, we obtain
\[
\frac{d}{dt} P_r(F_t) = rd[ \widetilde{P}_r(\theta, F_t \dots F_t) ]
\]

\[
\begin{aligned}
  & P_r(F') - P_r(F) = \int_0^1 dt \frac{d}{dt} P_r(F_t) = \int_0^1 dt r d[\widetilde{P}_r(\theta, F_t \dots F_t) ] 
\end{aligned} 
\]

\begin{equation}
   P_r(F') - P_r(F) = d \left[ r \int_0^1 \widetilde{P}_r(A'-A, F_t \dots F_t) dt \right] \quad \quad \quad \, (11.20)
\end{equation}

$P_r(F')$ differs from $P_r(F)$ by an exact form.  

\end{enumerate}
\end{proof}


transgression $TP_r(A',A)$ of $P_r$


\begin{equation}
TP_r(A',A) \equiv r \int_0^1 dt \widetilde{P}_r{ (A' \dots A, F_t \dots F_t ) }  \quad \quad \quad \, (11.21)
\end{equation}

$\widetilde{P}_r$ is polarization of $P$. \\

Let $\text{dim}{M} = m$ \\
$P_m(F')$ differs from $P_m(F)$ by an exact form, integrals over $M$ without boundary, should be same

\begin{equation}
  \int_M P_m(F') - \int_M P_m(F) = \int_M dTP_m(A',A) = \int_{\partial M} P_m(A',A) = 0 \quad \quad \quad \, (11.22)
\end{equation}


As proved, invariant polynomial closed and in general nontrivial.  \\
Accordingly, defines cohomology class of $M$ \\

Thm. 11.1(b) ($F$, $F'$ curvature 2-forms corresponding to different connections $A$, $A'$, difference $P(F')- P(F)$ is exact) ensures that this cohomology class is independent of the gauge potential chosen. \\
The cohomology class thus defines is called the \textbf{characteristic class}. \\
The characteristic class defined by an invariant polynomial $P$ is denoted by $\chi_E(P)$ where $E$ is a fiber bundle on which connections and curvatures are defined.   \\

\emph{Remark}: Since a principal bundle and its associated bundles share the same gauge potentials and field strengths, the Chern-Weil theorem applies equally to both bundles.  Accordingly, $E$ can be either a principal bundle or a vector bundle.

\subsection{Chern classes}

\url{http://www.johno.dk/mathematics/fiberbundlestryk.pdf}

\begin{definition}[9.7]
\emph{characteristic class} $c$ (with $\mathbb{R}$ coefficients) for principal $G$-bundle associates to every principal $G$-bundle $(E,\pi,M)$ cohomology class $c(E) \in H^*_{dR}(M)$ s.t. $\forall \, $ bundle map ($\overline{f}, f$):$(E',\pi', M') \to (E,\pi,M)$ (EY $\overline{f}$ is just the induced $f$ when you go up to $E$ level, vs. $M$ to $M'$ level) we have
\[
c(E') = f^*(c(E))
\]
if $c(E) \in H^l_{dR}(M)$ then $c$ has degree $l$
\end{definition}




\subsection{}



\subsection{}

\subsection{Chern-Simons form}


\subsubsection{ Definition }

Let $P_j(F)$ be an arbitrary $2j$-form characteristic class. \\
\phantom{Let } $P_j(F)$ closed, so by Poincar\'{e}'s lemma, it's locally exact. \\
\begin{equation}
  P_j(F) = dQ_{2j-1}(A,F) \quad \quad \quad \, (11.100)
\end{equation}

\[
Q_{2j-1}(A,F) \in \mathfrak{g} \otimes \Omega^{2j-1}(M)
\]

This can't be true globally. \\
\quad otherwise, if $P_j=dQ_{2j-1}$ globally on manifold $M$ with no boundary, 
\[
\int_M P_{m/2} = \int_M dQ_{m-1} = \int_{\partial M} Q_{m-1} = 0 
\]
$m = \text{dim}{M}$

$2j-1$  from $Q_{2j-1}(A,F)$ is the Chern-Simons form of $P_j(F)$

From Pf. of Thm. 11.2(b),  \\
\quad $Q$ is given by the transgression of $P_j$,

\begin{equation}
  Q_{2j-1}(A,F) = TP_j(A,0) = j \int_0^1 \widetilde{P}_j(A, F_t \dots F_t)dt \quad \quad \quad \, (11.101)
\end{equation}

where $\widetilde{P}_j$ polarization of $P_j$, $F= dA + A^2$ \\
\quad set $A'=F' =0$

of course $A'=0$ only on local chart over which bundle is trivial.  

Suppose $\text{dim}{M} = m = 2l$ s.t. $\partial M \neq \emptyset$.  By Stoke's 

\begin{equation}
  \int_M P_l(F) = \int_M dQ_{m-1}{ (A, F)} = \int_{\partial M} Q_{m-1}{ (A, F) } \quad \quad \quad \, (11.102)
\end{equation}

$\int_M P_l(F) \in \mathbb{Z}$ and so does $\int_{ \partial M } Q_{m-1}{ (A, F) }$

Thus $Q_{m-1}$ is a characteristic class in its own right and describes the topology of boundary $\partial M$





\subsubsection{ The Chern-Simons form of the Chern character}

Chern character $\text{ch}_j{(F)}$ \\
\quad connection $A_t$ which interpolates between $0$ and $A$, 

\begin{equation}
  A_t = tA \quad \quad \quad \, (11.103)
\end{equation}

the corresponding curvature is 

\begin{equation}
  F_t = A_t^2 + dA_t = t dA + t^2 A^2 = tF + (t^2- t)A^2 \quad \quad \quad \, (11.104)
\end{equation}

(11.21)

\begin{equation}
  Q_{2j-1}(A, F) = \frac{1}{ (j-1)!} \left( \frac{i}{ 2\pi } \right)^j \int_0^1 dt \text{str}{ (A, F^{j-1}_t )}
\end{equation}

\begin{equation}
  Q_1(A,F) = \frac{i}{2\pi} \int_0^1 dt \text{str}{ (A, F_t^0) } = \frac{i}{2\pi} \text{tr}{A} \quad \quad \quad \, (11.106a)
\end{equation}

\begin{equation}
  \begin{gathered} 
    Q_3(A,F) = \left( \frac{i}{2\pi} \right)^2 \int_0^1 dt \text{str}{ (A, F_t) } = \left( \frac{i}{2\pi} \right)^2 \int_0^1 dt \text{str}{ (A, tF + (t^2 - t) A^2) } = \\
    = \left( \frac{i}{2\pi} \right)^2 \int_0^1 dt \text{str}{ (A, t dA + tA^2 + (t^2 - t) A^2 )}    = \frac{1}{2} \left( \frac{i}{2\pi  } \right)^2 \text{tr}{ ( A dA + \frac{2}{3} A^3 ) }
\end{gathered}
\end{equation}






\section{Index Theorems}




\section{Anomalies in Gauge Field Theories}




\section{Bosonic String Theory}












\end{document} 













