%
% Cox_UsingAG_poznamky.tex 
%
\documentclass[twoside]{amsart}
\usepackage{amssymb,latexsym}
\usepackage{MnSymbol}
\usepackage{times}
\usepackage{graphics}
\usepackage{tikz}
\usepackage{hyperref}
\hypersetup{colorlinks=true, urlcolor=blue}
%\usepackage{simpsons}
%\usepackage{epsdice}
\usepackage{staves} 

\usetikzlibrary{matrix,arrows}
%\usepackage{graphics}

\oddsidemargin-0.85cm
\evensidemargin-0.65cm
\topmargin-2.05cm     %I recommend adding these three lines to increase the 
\textwidth19.05cm   %amount of usable space on the page (and save trees)
\textheight25.05cm  
\parindent0.0em

%This next line (when uncommented) allow you to use encapsulated
%postscript files for figures in your document
%\usepackage{epsfig}

%plain makes sure that we have page numbers
\pagestyle{plain}

\theoremstyle{plain}
\newtheorem{theorem}{Theorem}
\newtheorem{axiom}{Axiom}
\newtheorem{lemma}{Lemma}
\newtheorem{proposition}{Proposition}
\newtheorem{corollary}{Corollary}

\theoremstyle{definition}
\newtheorem{definition}{Definition}

\title{Notes and Solutions to \textbf{Using Algebraic Geometry} by David A. Cox, John Little, Donal O'Shea. Springer, Second Edition.  }

\author{
  Ernest Yeung - Praha
       }
\date{zima 2014}

%This defines a new command \questionhead which takes one argument and
%prints out Question #. with some space.
\newcommand{\questionhead}[1]
  {\bigskip\bigskip
   \noindent{\small\bf Question #1.}
   \bigskip}

\newcommand{\problemhead}[1]
  {
   \noindent{\small\bf Problem #1.}
   }

\newcommand{\exercisehead}[1]
  { \smallskip
   \noindent{\small\bf Exercise #1.}
  }

\newcommand{\solutionhead}[1]
  {
   \noindent{\small\bf Solution #1.}
   }


%-----------------------------------
\begin{document}
%-----------------------------------

\maketitle

Fund Science! \& Help Ernest finish his Physics Research! : quantum super-A-polynomials - a thesis by Ernest Yeung \\

\url{http://igg.me/at/ernestyalumni2014} \\

\noindent Facebook  : ernestyalumni \\
gmail        : ernestyalumni \\
google       : ernestyalumni \\
linkedin     : ernestyalumni \\
tumblr       : ernestyalumni \\
twitter      : ernestyalumni \\
weibo        : ernestyalumni \\ 
youtube      : ernestyalumni \\
indiegogo    : ernestyalumni \\

\textbf{Using Algebraic Geometry}.  David A. Cox.  John Little. Donal O'Shea. Second Edition.  Springer.  2005.  ISBN 0-387-20706-6 QA564.C6883 2004

\section{ Introduction }

\subsection{ Polynomials and Ideals }

\emph{monomial } 

\begin{equation}
  (1.1) \quad \quad \, x_1^{\alpha_1} \dots x_n^{\alpha_n}
\end{equation}

total degree of $x^{\alpha}$ is $\alpha_1 + \dots + \alpha_n \equiv |\alpha|$ \\



field $k$, $k[x_1 \dots x_n]$ collection of all polynomials in $x_1 \dots x_n$ with coefficients $k$.   \\

polynomials in $k[x_1 \dots x_n]$ can be added and multiplied as usual, so $k[x_1 \dots x_n]$ has structure of commutative ring (with identity) \\
however, only nonzero constant polynomials have multiplicative inverses in $k[x_1 \dots x_n]$, so $k[x_1 \dots x_n]$ not a field \\
\quad however set of rational functions $\lbrace f/g | f,g \in k[x_1 \dots x_n], \, g\neq 0\rbrace$ is a field, denoted $k(x_1 \dots x_n)$ \\

so
\[
f = \sum_{\alpha} c_{\alpha}x^{\alpha}
\]
where $c_{\alpha} \in k$

so

\[
f \in k [x_1 \dots x_n ] = \lbrace f | f = \sum_{\alpha} c_{\alpha} x^{\alpha} , x^{\alpha} = x_1^{\alpha_1} \dots x_n^{\alpha_n}, c_{\alpha} \in k \rbrace
\]

$f$ homogeneous if all monomials have same total degrees

polynomial $f$ is homogeneous if all monomials have the \emph{same total degree} \\

Given a collection of polynomials $f_1 \dots f_s \in k[x_1 \dots x_n]$, we can consider all polynomials which can be built up from these by multiplication by arbitrary polynomials and by taking sums

\begin{definition}[1.3] Let $f_1 \dots f_s \in k[x_1 \dots x_n]$ \\
Let $\langle f_1 \dots f_s \rangle = \lbrace p_1 f_1  + \dots + p_s f_s | p_i \in k[x_1 \dots x_n] \text{ for } i = 1 \dots s \rbrace$
\end{definition}


\exercisehead{1} 
\begin{enumerate}
\item[(a)] $x^2 = x \cdot ( x  -y^2 ) + y \cdot ( xy )$  
\item[(b)] 
\[
 p \cdot ( x - y^2 ) = p x - p y^2
\]
and for $p xy = (py)x$
\item[(c)] 
\[
p(y) ( x - y^2) = p(y)x - p(y) y^2 \notin \langle x^2, xy \rangle
\]
\end{enumerate}

\exercisehead{2} 
\[
\begin{gathered}
  \sum_{i=1}^s p_i f_i  + \sum_{j=1}^s q_j f_j = \sum_{i=1}^s (p_i + q_i )f_i, \quad \, p_i + q_i \in k[x_1 \dots x_n] \end{gathered}
\]
$\langle f_1 \dots f_s \rangle$ closed under sums in $k[x_1 \dots x_n]$ \\

If $f\in \langle f_1 \dots f_s \rangle$, \\
\phantom{If }$ p \in k [x_1 \dots x_n]$

\[
\begin{gathered}
p\cdot f = p \sum_{i=1}^s q_j f_j = \sum_{i=1}^s pq_j f_j, \quad \, pq_j \in k[x_1 \dots x_n] \text{ so }  \\
p\cdot f \in \langle f_1 \dots f_s \rangle
\end{gathered}
\]

Done.  \\

The 2 properties in Ex. 2 are defining properties of ideals in the ring $k[x_1 \dots x_n]$

\begin{definition}[1.5]
Let $I \subset k[x_1 \dots x_n]$, \, $I \neq \emptyset$ \\
$I$ ideal if 
\begin{enumerate}
\item[(a)] $f+ g \in I$, \, $\forall \, f,g \in I$ 
\item[(b)] $pf \in I$, \, $\forall \, f \in I$, arbitrary $p \in k[x_1 \dots x_n]$
\end{enumerate}
\end{definition}

Thus $\langle f_1 \dots f_s\rangle$ is an ideal by Ex. 2.  \\

we call it the ideal generated by $f_1 \dots f_s$.  

\exercisehead{3} Suppose $\exists \, $ ideal $J$, $f_1 \dots f_s \in J$ s.t. $J \subset \langle f_1 \dots f_s \rangle$ \\
if $f\in \langle f_1 \dots f_s \rangle$, $f = \sum_{i=1}^s p_i f_i$, \, $p_i \in k[x_1 \dots x_n]$ \\

$\forall \, i = 1 \dots s$, $p_i f_i \in J$ and so $\sum_{i=1}^s p_i f_i \in J$, by def. of $J$ as an ideal.

\[
\langle f_1 \dots f_s \rangle \subseteq J \quad \quad \, \Longrightarrow J = \langle f_1 \dots f_s \rangle
\]

$\Longrightarrow \langle f_1 \dots f_s \rangle$ is smallest ideal in $k[x_1 \dots x_n]$ containing $f_1 \dots f_s$


\exercisehead{4} For $\begin{aligned} & \quad \\
  & I = \langle f_1 \dots f_s \rangle \\
  & J = \langle g_1 \dots g_t \rangle \end{aligned}$ \\

 $I = J$ iff $s=t$ and $\forall \, f \in I$, $f = \sum_{i=1}^t q_i g_i$ and if $ 0  = \sum_{i=1}^t q_i g_i$, $q_i =0$, \, $\forall \, i = 1 \dots t$, and if $0 = \sum_{i=1}^s p_i f_i$, \, $p_i = 0$, \, $\forall \,  i = 1 \dots s$


\begin{definition}[1.6]
\[
\sqrt{I} = \lbrace g \in k[x_1\dots x_n] | g^m \in I \text{ for some } m \geq 1 \rbrace
\]
\end{definition}

e.g. $x+y \in \sqrt{ \langle x^2 + 3 xy , 3xy + y^2 \rangle }$

in $\mathbb{Q}[x,y]$ since 

\[
(x+y)^3 =x(x^2 + 3xy) + y(3xy + y^2) \in \langle x^2 + 3xy, 3xy + y^2\rangle
\]


%\begin{definition}[1.6]
%\[
%\sqrt{I } = \lbrace g \in k[x_1 \dots x_n ] | g^m \in I \text{ for some m } \geq 1 \rbrace
%\]
%\end{definition}

%e.g. $x+y \in \sqrt{ \langle x^2 + 3xy, 3xy + y^2 \rangle }$

%in $\mathbb{Q}[x,y]$ since

%\[
%(x+y)^3 = x(x^2 + 3xy) + y(3xy + y^2) \in \langle x^2 + 3xy, 3xy + y^2 \rangle
%\]

\begin{itemize}
\item (Radical Ideal Property) $\forall \, $ ideal $I\subset k[x_1 \dots x_n]$, $\sqrt{I}$ ideal, $\sqrt{I} \supset I$
\item \textbf{(Hilbert basis Thm.)} $\forall \, $ ideal $I\subset k[x_1\dots x_n]$ \\
$\exists \, $ finite generating set, \\
i.e. $\exists \, \lbrace f_1 \dots f_2 \rbrace \subset k [x_1 \dots x_n]$ s.t. $I=\langle f_1 \dots f_s \rangle$
\item (Division Algorithm in $k[x]$) $\forall \, f,g \in  k[x]$ (EY : in 1 variable) \\
$\forall \, f, g \in k[x]$ (in 1 variable )\\
$f= qg + r$, $\exists \, !$ quotient $q$, $\exists \, $ remainder $r$
\end{itemize}

\subsection{}

\subsection{Gr\"obner Bases}

\begin{definition}[3.1]
  Gr\"obner basis for $I$ $\equiv G = \lbrace g_1 \dots g_k \rbrace \subset I$ s.t. $\forall \, f \in I$, $\text{LT}(f)$ divisible by $\text{LT}(g_i)$ for some $i$
\end{definition}

\begin{itemize}
\item (Uniqueness of Remainders) let ideal $I\subset k[x_1 \dots x_n]$ \\
division of $f\in k[x_1 \dots x_n]$ by Gr\"o bner basis for $I$, produces $f=g+r$, $g\in I$, and no term in $r$ divisible by any element of $\text{LT}(I)$
\end{itemize}





\subsection{Affine Varieties}

affine $n$-dim. space over $k$ \quad \, $k^n = \lbrace (a_1 \dots a_n ) | a_1 \dots a_n \in k \rbrace$

$\forall \, $ polynomial $f\in k[x_1 \dots x_n ]$, $(a_1 \dots a_n) \in k^n$ \\
\phantom{ \quad } $f: k^n \to k$ \\
\phantom{ \quad } $f(a_1 \dots a_n)$ s.t. $x_i = a_i$ i.e. \\

if $f= \sum_{\alpha} c_{\alpha} x^{\alpha}$ for $c_{\alpha} \in k$, then  \\
\phantom{ \quad } $f(a_1 \dots a_n) =\sum_{\alpha} c_{\alpha}a^{\alpha} \in k$, where $a^{\alpha} = a_1^{\alpha_1} \dots a_n^{\alpha_n}$

\begin{definition}[4.1]
affine variety $\mathbf{V}(f_1 \dots f_s) = \lbrace ( a_1 \dots a_n) | (a_1 \dots a_n) \in k^n, \, f_1(x_1 \dots x_n) = \dots = f_s(x_1 \dots x_n) = 0 \rbrace$ \\
subset $V\subset k^n$ is affine variety if $V = V(f_1 \dots f_s)$ for some $\lbrace f_i \rbrace$, polynomial $f_i \in k[x_1 \dots x_n]$
\end{definition}

\begin{itemize}
  \item (Equal Ideals Have Equal Varieties) If $\langle f_1 \dots f_s \rangle = \langle g_1 \dots g_t \rangle$ in $k[x_1 \dots x_n]$, then $\mathbf{V}(f_1 \dots f_s) = \mathbf{V}(g_1 \dots g_t)$
\end{itemize}

so, recap \\
if $\langle f_1 \dots f_s \rangle = \langle g_1 \dots g_t \rangle $ in $k[x_1 \dots x_n]$, \\
then $V(f_1 \dots f_s) = V(g_1 \dots g_t)$   \\

Recall Hilbert basis Thm. $\forall \, $ ideal $I \subset k[x_1 \dots x_n]$ 
\[
I= \langle f_1 \dots f_s \rangle
\]
$\Longrightarrow $ if $I=J$, then $V(I) = V(J)$

think of $V$ defined by $I$, rather than $f_1 = \dots = f_s =0$

\exercisehead{3}

Recall Def. 1.5 Let $I\subset k[x_1 \dots x_n]$ \\
$I$ ideal if $\begin{aligned} & \quad \\
  & f + g \in I  \quad \, \forall \, f,g \in I \\
  & pf \in I, \quad \, \forall \, f \in I \text{ arbitrary } p \in k[x_1 \dots x_n]
\end{aligned}$

Let $f,g \in I(V)$ 
\[
\begin{gathered}
  (f+g)(a_1 \dots a_n) = f(a_1 \dots a_n) + g(a_1 \dots a_n) = 0 + 0 = 0 \quad \quad \, f+g \in I(V) \\ 
  pf(a_1 \dots a_n) = p(a_1 \dots a_n) f(a_1 \dots a_n) = 0 \quad \quad \, pf \in I(V)
\end{gathered}
\]
Then $I(V)$ an ideal.

%\exercisehead{3} Recall Def. 1.5. Let $I \subset k[x_1 \dots x_n]$, $I$ ideal if $\begin{aligned} & \quad \\
%  & f+g \in I \quad \, \forall \, f,g\in I \\
%  & pf \in I , \quad \, \forall \, f \in I, \text{ arbitrary } p \in k[x_1 \dots x_n ] \end{aligned}$

%Let $f,g \in I(V)$ 
%\[

$V = V(x^2)$ in $\mathbb{R}^2$ \\
$I=\langle x^2 \rangle$ in $\mathbb{R}[x,y]$, \, $I= \lbrace px^2 | p \in k[x,y]\rbrace$ \\
\phantom{ \quad } $I \subset I(V)$, since $px^2 = 0$ for $x^2=0$, $(0,b)$, \, $b\in \mathbb{R}$ \\
But $p(x,y) = x\in I(V)$, as 
\[
I(V) = \lbrace f \in k[x_1 \dots x_n] | f(a_1 \dots a_n)=0, \, \forall \, (a_1\dots a_n) \in V\rbrace
\]
\phantom{ \quad \quad } $p(0,b) = x = 0$

But $x\notin I$

\exercisehead{4} $I\subset \sqrt{I}$

Recall Def. 1.6 $\sqrt{I} = \lbrace g \in k[x_1 \dots x_n] |g^m \in I \text{ for some } m\geq 1\rbrace$ \\
$\forall \, f \in I$, $f=f^1$, $m=1$, so $f\in \sqrt{I}$, \quad \, $I\subset \sqrt{I}$ \\
\phantom{\quad \quad } Hilbert basis thm., $\forall \, $ ideal $I\subset k[x_1 \dots x_n]$ s.t. $I=\langle f_1 \dots f_s \rangle$ \\
\phantom{\quad } $V(I) = \lbrace (a_1 \dots a_n) |(a_1 \dots a_n) \in k^n, \, f_1(a_1\dots a_n) = \dots = f_s(a_1\dots a_n)=0\brace$ \\
$\mathbf{I}(\mathbf{V}(I)) = \lbrace f \in k[x_1 \dots x_n] | f(a_1 \dots a_n) =0 \quad \, \forall \, (a_1 \dots a_n) \in V(I) \rbrace$ \\
Let $g\in \sqrt{I}$, \, $g^m \in I$, \, $g^m=g^{m-1}g$  \\
\phantom{\quad \quad \,} $g^m(a_1 \dots a_n) =0 = g^{m-1}(a_1 \dots a_n)g(a_1 \dots a_n) =0$.  Then $g(a_1 \dots a_n)=0$ or $g^{m-1}(a_1\dots a_m)=0$ \\
\phantom{\quad }as $g^m\in I$, and $V(I)$ is s.t. $f_1(a_1 \dots a_n) = \dots = f_s(a_1 \dots a_n)=0$ for $I=\langle f_1 \dots f_s \rangle$

\begin{itemize}
  \item (Strong Nullstellensatz) if $k$ algebraically closed (e.g. $\mathbb{C}$), $I$ ideal in $k[x_1 \dots x_n]$, then 
\[
\mathbf{I}(\mathbf{V}(I) = \sqrt{I}
\]
\item (Ideal-variety correspondence) Let $k$ arbitrary field
\[
\begin{aligned}
  & I \subset I(V(I)) \\ 
  & V(I(V)) = V \quad \, \forall \, V
\end{aligned}
\]
\end{itemize}

\subsection*{Additional Exercises for Sec.4}

\exercisehead{6}



\section{ Solving Polynomial Equations}

\subsection{}

\subsection{Finite-Dimensional Algebras}

Gr\"obner basis $G = \lbrace g_1 \dots g_t \rbrace$ of ideal $I\subset k[x_1\dots x_n]$, \\
recall def.: Gr\"obner basis $G = \lbrace g_1 \dots g_t\rbrace \subset I$ of ideal $I$, \, $\forall \, f \in I$, $\text{LT}(f)$ divisible by $\text{LT}(g_i)$ for some $i$ \\
\phantom{\quad \, } $f \in k[x_1\dots x_n]$ divide by $G$ produces $f=g+r$, $g\in I$, $r$ not divisible by any $\text{LT}(I)$ uniqueness of $r$ \\
$f\in k[x_1 \dots x_n]$ divide by $G$, 

Recall from Ch. 1, divide $f\in k[x_1 \dots x_n]$ by $G$, the division algorithm yields

\begin{equation}
  (2.1)  \quad \quad \quad \, f = h_1 g_1 + \dots + h_t g_t + \overline{f}^G
\end{equation}
where remainder $\overline{f}^G$ is a linear combination of monomials $x^{\alpha} \notin \langle \text{LT}(I) \rangle $ \\
\phantom{\quad } since Gr\"obner basis, $f\in I$ iff $\overline{f}^G=0$

$\forall \, f \in k[x_1\dots x_n]$, we have coset $[f] = f+I = \lbrace f +h|h\in I\rbrace$ s.t. $[f]=[g]$ iff $f- g \in I$

We have a 1-to-1 correspondence 
\[
\begin{gathered}
\text{remainders } \leftrightarrow \text{ cosets } \\
\overline{f}^G \leftrightarrow [f]
\end{gathered}
\]
algebraic
\[
\begin{aligned}
  & \overline{f}^G + \overline{g}^G \leftrightarrow [f] + [g] \\ 
  & \overline{ \overline{f}^G \cdot \overline{g}^G } \leftrightarrow [f]\cdot [g]
\end{aligned}
\]
$B = \lbrace x^{\alpha} | x^{\alpha} \notin \langle \text{LT}(I) \rangle \rbrace$ is a basis of $A$, basis monomials, standard monomials

20141023 EY's take

$\forall \, [f] \in A = k[x_1 \dots x_n]/I$, \, $[f] = p_ib_i$; \, $b_i \in B = \lbrace x^{\alpha} | x^{\alpha} \notin \langle \text{LT}(I) \rangle \rbrace$ \\
For $I = \langle G \rangle$ \\
\phantom{\quad } e.g. $G=\lbrace x^2 + \frac{3}{2} xy + \frac{1}{2} y^2 - \frac{3}{2} x - \frac{3}{2} y, xy^2-x, y^3-y \rbrace$ \\
$\langle \text{LT}(I) \rangle = \langle x^2, xy^2,y^3 \rangle$ \\
e.g. $B=\lbrace 1,x,y,xy,y^2\rbrace$ \\
\phantom{\quad } $[f]\cdot[g] = [fg]$ \\
e.g. $f=x, \, g=xy, \, [fg] = [x^2y]$ \\
now $f=h_1g_1 + \dots +h_tg_t+ \overline{f}^G$

\subsection{}

\subsection{Solving Equations via Eigenvalues and Eigenvectors}


\section{ Resultants }

\section{Computation in Local Rings}

\subsection{Local Rings}


\begin{definition}[1.1]
  \[
k[x_1 \dots x_n]_{\langle x_1 \dots x_n \rangle} \equiv \lbrace \frac{f}{g} | \text{ rational functions } \frac{f}{g} \text{ of } x_1 \dots x_n \text{ with } g(p) \neq 0 \text{ at } p \rbrace
\]
\end{definition}

main properties of $k[x_1 \dots x_n]_{\langle x_1 \dots x_n \rangle }$

\begin{proposition}[1.2]
  Let $R= k[x_1 \dots x_n]_{\langle x_1 \dots x_n \rangle }$.  Then
\begin{enumerate}
\item[(a)] $R$ subring of field of rational functions $k(x_1 \dots x_n) \supset k[x_1 \dots x_n]$
\item[(b)] Let $M=\langle x_1 \dots x_n \rangle \subset R$ (ideal generated by $x_1 \dots X_n$ in $R$) \\
Then $\forall \, \frac{f}{g} \in R \backslash M$, $\frac{f}{g}$ unit in $R$ (i.e. multiplicative inverse in $R$)
\item[(c)] $M$ maximal ideal in $R$
\end{enumerate}
\end{proposition}


\exercisehead{1} if $p=(a_1 \dots a_n) \in k^n$, $R = \lbrace \frac{f}{g} | f,g\in k[x_1 \dots x_n] , \, g(p) \neq 0 \rbrace$ 
\begin{enumerate}
\item[(a)] $R$ subring of field of rational functions $k(x_1 \dots x_n)$ 
\item[(b)] Let $M$ ideal generated by $x_1 - a_1 \dots x_n -a_n$ in $R$  \\
Then $\forall \, \frac{f}{g} \in R\backslash M$, $\frac{f}{g}$ unit in $R$ (i.e. multiplicative inverse in $R$)
\item[(c)]  $M$ maximal ideal in $R$
\end{enumerate}


\begin{proof}
let $p = (a_1 \dots a_n) \in k^n$ \\
let $g_1(p) \neq 0$, $g_2(p) \neq 0$ 
\[
\begin{gathered}
  \frac{f_1}{g_1 } + \frac{f_2}{g_2} = \frac{f_1 g_2 + f_2 g_1}{ g_1 g_2 } \quad \quad \,  g_1(p)g_2(p) \neq 0 \text{ so } \frac{f_1}{g_1} + \frac{f_2}{g_2} \in R \\
 \frac{f_1}{g_1} \cdot \frac{f_2}{g_2} = \frac{f_1 f_2}{g_1 g_2} \quad \quad \, g_1(p) g_2(p) \neq 0 \text{ so } \frac{f_1}{g_1}\frac{f_2}{g_2} \in R
\end{gathered}
\]
$f= \frac{f}{I} \in R$, \quad \, $\forall \, f\in k[x_1 \dots x_n]$, so $k[x_1 \dots x_n]\subset R$

\end{proof}

EY : 20141027, to recap, 

Let $V = k^n$ \\
Let $p = (a_1 \dots a_n)$ \\
single pt. $\lbrace p \rbrace$ is (an example of) a variety \\
$I(\lbrace p \rbrace) = \lbrace x_1 -a_1 \dots x_n -a_n \rangle \subset k[x_1 \dots x_n]$ \\

$R \equiv k[x_1 \dots x_n]_{\langle x_1 - a_1 \dots x_n-a_n \rangle }$ 
\[
R = \lbrace \frac{f}{g} | \text{ rational function $\frac{f}{g}$ of $x_1 \dots x_n$, $g(p) \neq 0$, $p=(a_1 \dots a_n) $ } \rbrace
\]

Prop. 1.2. properties 

\begin{enumerate}
\item[(a)] $R$ subring of field of rational functions $k(x_1 \dots x_n)$ \quad \, $k(x_1 \dots x_n) \subset R$ 
\item[(b)] $M = \langle x_1 \dots a_1 \dots x_n -a_n \rangle \subset R$.  ideal generated by $x_1 - a_1 \dots x_n-a_n$ \\
Then $\forall \, \frac{f}{g} \in R\backslash M$, $\frac{f}{g}$ unit in $R$ ( $\exists \, $ multiplicative inverse in $R$ )
\item[(c)] $M$ maximal ideal in $R$. \\
in $R$ we allow denominators that are not elements of this ideal $I(\lbrace p \rbrace)$ 
\end{enumerate}

\begin{definition}[1.3] local ring is a ring that has exactly 1 maximal ideal \end{definition}

\begin{proposition}[1.4] ring $R$ with proper ideal $M\subset R$ is local ring if $\forall \, \frac{f}{g} \in R\backslash M$ is unit in $R$
\end{proposition}

localization Ex. 8, Ex. 9 \\
parametrization

\exercisehead{2} \[
\begin{aligned}
  & x = x(t) = \frac{-2t^2 }{1+t^2} \\ 
 &  y = y(t) = \frac{2t}{1+t^2}
\end{aligned}
\]
$k[t]_{\langle t \rangle}$ \quad \, $\frac{-2t^2}{1+t^2}$ rational function of $t$.  $1+t^2 \neq 0$

if $k = \mathbb{C}$ or $\mathbb{R}$ \\

Consider set of convergent power series in $n$ variables \\

\begin{equation}
(1.5) \quad \quad \,   k\lbrace x_1 \dots x_n \rbrace = \lbrace \sum_{\alpha \in \mathbb{Z}^n_{\geq 0}} c_{\alpha} x^{\alpha} | c_{\alpha} \in k, \text{ series converges in some open $U\ni 0 \in k^n $ } \rbrace
\end{equation}

Consider set $k[[x_1 \dots x_n]]$ of formal power series

\begin{equation}
  (1.6) \quad \quad \, k[[x_1 \dots x_n]] = \lbrace \sum_{\alpha \in \mathbb{Z}^n_{\geq 0}} c_{\alpha} x^{\alpha} | c_{\alpha} \in k \rbrace \text{ series need not converge }
\end{equation} 


variety $V$ \\

$k[x_1\dots x_n]/\mathbf{I}(V)$ \phantom{ \quad \quad \quad } variety $V$


\subsection{Multiplicities and Milnor Numbers}


if $I$ ideal in $k[x_1\dots x_n]$, then denote $Ik[x_1\dots x_n]_{\langle x_1 \dots x_n \rangle}$ ideal generated by $I$ in larger ring $k[x_1\dots x_n]_{\langle x_1 \dots x_n \rangle}$

\begin{definition}[2.1] Let $I$ $0$-dim. ideal in $k[x_1 \dots x_n]$, so $V(I)$ consists of finitely many pts. in $k^n$.  \\
Assume $(0 \dots 0) \in V(I)$ \\
multiplicity of $(0\dots 0)\in V(I)$ is 
\[
\text{dim}_k{ k[x_1\dots x_n]_{\langle x_1\dots x_n \rangle}} / Ik[x_!\dots x_n]_{\langle x_1 \dots x_n \rangle}
\]
\end{definition}


generally, if $p=(a_1 \dots a_n) \in V(I)$ \\
multiplicity of $p$, $m(p) = \text{dim}{ k[x_1 \dots x_n]_M } / Ik[x_1 \dots x_n]_M$

\[
\text{dim}{ k[x_1 \dots x_n]_M } / Ik[x_1 \dots x_n]_M
\]

localizing $k[x_1 \dots x_n]$ at maximal ideal $M = I(\lbrace p \rbrace) = \langle x_1 - a_1 \dots x_n-a_n \rangle$


\section{}

\section{}

\section{ Polytopes, Resultants, and Equations }

\section{ Polyhedral Regions and Polynomials }

\subsection{ Integer Programming }

Prop. 1.12. \\

Suppose 2 customers $A, B$ ship to same location \\
\quad A: ship 400 kg pallet taking up $2 \, m^3$ volume \\
\quad B: ship 500 kg pallet taking up $3 \, m^3$ volume \\

shipping firm trucks carry up to 3700 \, kg, up to $20 \, m^3$ \\

B's product more perishable, paying \$ 15 per pallet \\

A pays \$ 11 per pallet

How many pallets from A, B each in truck to maximize revenues?

\begin{equation}
(1.1) \quad \quad \, \begin{gathered}
    4A + 5B \leq 37 \\
    2A  + 3B \leq 20 \\
    A, B \in \mathbb{Z}^*_{ \geq 0 } \end{gathered}
\end{equation}

maximize $11 A + 15 B$ \\

integer programming. \\
max. or min. value of some linear function 

\[
l(A_1 \dots A_n) = \sum_{i=1}^n c_i A_i 
\]

on set $(A_1 \dots A_n) \in \mathbb{Z}^n_{ \geq 0}$ s.t. 


3. Finally, by introducing additional variables; rewrite linear constraint inequalities as equalities. The new variables are called ``slack variables''

\begin{equation}
(1.4) \quad \quad \, a_{ij} A_j = b_i, \quad \, A_j \in \mathbb{Z}_{\geq 0}
\end{equation}

introduce indeterminate $z_i$, \, $\forall \, $ equation in (1.4)

\[
z_i^{a_{ij} A_j} = z_i^{b_i}
\]

$m$ constraints

\[
\prod_{i=1}^m z_i^{a_{ij}A_j} = \prod_{i=1}^m z_i^{b_i} = \left( \prod_{i=1}^m z_i^{a_{ij}} \right)^{ A_j}
\]

\begin{proposition}[1.6]
  Let $k$ field, define $\varphi: k[w_1 \dots w_n] \to k[z_1 \dots z_m]$ by 
\[
\varphi(w_j) = \prod_{i=1}^m z_i^{a_{ij}} \quad \quad \quad \, \forall \, j = 1 \dots n 
\]

and 

\[
\varphi(g(w_1 \dots w_n) ) = g(\varphi(w_1) \dots \varphi(w_n))
\]
$\forall \, $ general polynomial $g\in k[w_1 \dots w_n]$

Then $(A_1 \dots A_n)$ integer pt. in feasible region iff $\varphi: w_1^{A_1} \dots w_n^{A_n} \mapsto z_1^{b_1} \dots z_m^{b_m}$



\end{proposition}

\exercisehead{3}

Now 

\[
\begin{gathered}
\varphi(w_j) = \prod_{i=1}^m z_i^{a_{ij}} \\
z_i^{a_{ij} A_j} = z_i^{b_i}
\end{gathered}
\]

If $(A_1 \dots A_n)$ an integer pt. in feasible region, $a_{ij} A_j = b_i$

\[
\begin{gathered}
z_i^{a_{ij}A_j } = z_i^{b_i} = \prod_{j=1}^n z_i^{a_{ij} A_j} \Longrightarrow \prod_{j=1}^n \prod_{i=1}^m (z_i^{a_{ij} })^{A_j} = \prod_{i=1}^m z_i^{b_i} = \prod_{j=1}^n \varphi(w_j)^{ A_j} = \prod_{j=1}^n \varphi(w_j)^{A_j} = \varphi\left( \prod_{j=1}^n w_j^{ A_j } \right) = \prod_{i=1}^m z_i^{b_i}
\end{gathered}
\]
since $\varphi(g(w_1 \dots w_n)) = g(\varphi(w_1) \dots \varphi(w_n))$ \\

If $\varphi: \prod_{j=1}^n w_j^{A_j} \mapsto \prod_{i=1}^m z_i^{b_i}$

\[
\varphi\left( \prod_{j=1}^n w_j^{A_j} \right) = \prod_{j=1}^n (\varphi(w_j))^{A_j} = \prod_{i=1}^m z_i^{b_i} = \prod_{j=1}^n \left( \prod_{i=1}^m z_i^{a_{ij}} \right)^{ A_j} \Longrightarrow \prod_{j=1}^n z_i^{a_{ij} A_j} = z_i^{b_i}
\]
or $a_{ij}A_j = b_i$.  So $(A_1\dots A_n)$ integer pt.  




\exercisehead{4} 
\[
\prod_{i=1}^m z_i^{b_i} = \prod_{i=1}^m \prod_{j=1}^n z_i^{ a_{ij} A_j } = \prod_{j=1}^n \left( \prod_{i=1}^m z_i^{a_{ij}} \right)^{A_j} = \prod_{j=1}^n \varphi(w_j)^{A_j} = \varphi\left( \prod_{j=1}^n w_j^{A_j} \right)
\]
So if given $(b_1 \dots b_m) \in \mathbb{Z}^m$, and for a given $a_{ij}$, $a_{ij}A_j = b_i$ \\

For $m\leq n$, then $a_{ij}$ is surjective, so $\exists \, A_j$ s.t. $\prod_{i=1}^m z_i^{b_i} = \varphi\left( \prod_{j=1}^n w_j^{A_j} \right)$



\begin{proposition}[1.8]
Suppose $f_1 \dots f_n \in k[z_1 \dots z_m]$ given \\
Fix monomial order in $k[z_1 \dots z_n, w_1 \dots w_n ]$ with elimination property: \\
$\forall \, $ monomial containing 1 of $z_i$ greater than any monomial containing only $w_j$ \\

Let $\mathcal{G}$ Gr\"{o}bner basis for ideal
\[
I = \langle f_1 - w_1 \dots f_n - w_n \rangle \subset k[z_1 \dots z_m, w_1 \dots w_n]
\]
$\forall \, f \in k[z_1 \dots z_m]$, let $\overline{f}^{ \mathcal{G}}$ be remainder on division of $f$ by $\mathcal{G}$ \\
Then
\begin{enumerate}
\item[(a)] polynomial $f$ s.t. $f\in k[f_1 \dots f_n]$ iff $g= \overline{f}^{ \mathcal{G}} \in k[w_1 \dots w_n]$
\item[(b)] if $\begin{aligned} & \quad \\
  & f \in k [f_1 \dots f_n ] \\
  & g = \overline{f}^{\mathcal{G}}\in k[ w_1 \dots w_n] \end{aligned}$ \quad as in part (a), \\

then $f = g(f_1 \dots f_n)$ , giving an expression for $f$ as polynomial in $f_j$
\item[(c)] if $\forall \, f_i, f$ monomials, $f\in k[f_1 \dots f_n]$, \\
then $g$ also a monomial.  
\end{enumerate}
\end{proposition}



\subsection{ Integer Programming and Combinatorics }



\section{ Algebraic Coding Theory }


\section{ The Berlekamp-Massey-Sakata Decoding Algorithm }


\end{document}

