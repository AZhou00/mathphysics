%
% Cox_Ideals_notes.tex
%
\documentclass[twoside]{amsart}
\usepackage{amssymb,latexsym}
\usepackage{MnSymbol}
\usepackage{times}
\usepackage{graphics}
\usepackage{tikz}
\usepackage{hyperref}
\hypersetup{colorlinks=true, urlcolor=blue}
%\usepackage{simpsons}
%\usepackage{epsdice}
\usepackage{staves} 

\usetikzlibrary{matrix,arrows}
%\usepackage{graphics}

\oddsidemargin-0.85cm
\evensidemargin-0.65cm
\topmargin-2.05cm     %I recommend adding these three lines to increase the 
\textwidth19.05cm   %amount of usable space on the page (and save trees)
\textheight25.05cm  
\parindent0.0em

%This next line (when uncommented) allow you to use encapsulated
%postscript files for figures in your document
%\usepackage{epsfig}

%plain makes sure that we have page numbers
\pagestyle{plain}

\theoremstyle{plain}
\newtheorem{theorem}{Theorem}
\newtheorem{axiom}{Axiom}
\newtheorem{lemma}{Lemma}
\newtheorem{proposition}{Proposition}
\newtheorem{corollary}{Corollary}

\theoremstyle{definition}
\newtheorem{definition}{Definition}

\title{Notes and Solutions to \emph{Ideals, Varieties, and Algorithms} by David A. Cox, John Little, Donal O'Shea  }

\author{
  Ernest Yeung - M\"{u}nchen
       }
\date{Marz 2014, jaro, primavera}

%This defines a new command \questionhead which takes one argument and
%prints out Question #. with some space.
\newcommand{\questionhead}[1]
  {\bigskip\bigskip
   \noindent{\small\bf Question #1.}
   \bigskip}

\newcommand{\problemhead}[1]
  {
   \noindent{\small\bf Problem #1.}
   }

\newcommand{\exercisehead}[1]
  { \smallskip
   \noindent{\small\bf Exercise #1.}
  }

\newcommand{\solutionhead}[1]
  {
   \noindent{\small\bf Solution #1.}
   }


%-----------------------------------
\begin{document}
%-----------------------------------

\maketitle

From the beginning of 2016, I decided to cease all explicit crowdfunding for any of my materials on physics, math.  I failed to raise \emph{any} funds from previous crowdfunding efforts.  I decided that if I was going to live in \emph{abundance}, I must lose a scarcity attitude.  I am committed to keeping all of my material \textbf{open-sourced}.  I give all my stuff \emph{for free}.   

In the beginning of 2017, I received a very generous donation from a reader from Norway who found these notes useful, through \emph{PayPal}.  If you find these notes useful, feel free to donate directly and easily through \href{https://www.paypal.com/cgi-bin/webscr?cmd=_donations&business=ernestsaveschristmas%2bpaypal%40gmail%2ecom&lc=US&item_name=ernestyalumni&currency_code=USD&bn=PP%2dDonationsBF%3abtn_donateCC_LG%2egif%3aNonHosted}{PayPal}, which won't go through a 3rd. party such as indiegogo, kickstarter, patreon.  Otherwise, under the \emph{open-source MIT license}, feel free to copy, edit, paste, make your own versions, share, use as you wish.    

\noindent gmail        : ernestyalumni \\
linkedin     : ernestyalumni \\
tumblr       : ernestyalumni \\
twitter      : ernestyalumni \\
youtube      : ernestyalumni \\



David Cox, John Little, Donal O'Shea. \textbf{Ideals, Varieties, and Algorithms: An Introduction to Computational Algebraic Geometry and Commutative Algebra}, Third Edition, Springer

\section{Geometry, Algebra, and Algorithms}

\subsection{Polynomials and Affine Space}

fields are important is that linear algebra works over \emph{any} field

\begin{definition}[2] set of all polynomials in $x_1 , \dots , x_n$ with coefficients in $k$, denoted $k[x_1, \dots , x_n]$

\end{definition}

polynomial $f$ \emph{divides} polynomial $g$ provided $g= fh$ for some $h \in k[x_1, \dots , x_n ]$

$k[x_1, \dots, x_n]$ satisfies all field axioms except for existence of multiplicative inverses; commutative ring, $k[x_1, \dots , x_n]$ \emph{polynomial ring}


\subsubsection*{Exercises for 1 }

\exercisehead{1}
$\mathbb{F}_2$ commutative ring since it's an abelian group under addition, commutative in multiplication, and multiplicative identity exists, namely $1$.  It is a field since for $1\neq 0$, the multiplicative identity is $1$.  

\exercisehead{2}
\begin{enumerate}
\item[(a)]
\item[(b)]
\item[(c)]
\end{enumerate}




\subsection{Affine Varieties }


\subsection{Parametrizations of Affine Varieties}


\subsection{Ideals}



\subsection{Polynomials of One Variable}




\section{Groebner Bases}

\subsection{Introduction}




\subsection{Orderings on the Monomials in $k[x_1, \dots , x_n]$ }




\subsection{A Division Algorithm in $k[x_1, \dots , x_n ]$ }



\subsection{Monomial Ideals and Dickson's Lemma }


\subsection{The Hilbert Basis Theorem and Groebner Bases}


\subsection{Properties of Groebner Bases}



\subsection{Buchberger's Algorithm}






\section{Elimination Theory}



\subsection{The Elimination and Extension Theorems}


\subsection{The Geometry of Elimination}



\section{The Algebra-Geometry Dictionary}


\subsection{Hilbert's Nullstellensatz}


\subsection{Radical Ideals and the Ideal-Variety Correspondence}



\section{Polynomial and Rational Functions on a Variety}


\subsection{Polynomial Mappings }


\section{Robotics and Automatic Geometric Theorem Proving}



\subsection{Geometric Description of Robots }








\end{document}

